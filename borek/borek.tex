\begin{newrevplenv}{Michał Borek}
	{Pułapka na ludzi myślących}
	{Pułapka na ludzi myślących}
	{Pułapka na ludzi myślących}
	{Afiliacja}
	{Andrzej Dragan, \textit{Kwantechizm czyli klatka na ludzi}, Wydawnictwo Fabuła Fraza, Warszawa 2019, ss.~xx?.}


\lettrine[loversize=0.13,lines=2,lraise=-0.03,nindent=0em,findent=0.2pt]%
{A}{}ndrzej Dragan jest absolwentem i~wykładowcą na Wydziale Fizyki Uniwersytetu Warszawskiego, gdzie habilitował się na podstawie pracy pt. \textit{Relatywistyczna informacja kwantowa}. Autor wykładał również na Imperial College London oraz w~University of Nottingham, a~także gościnnie na uniwersytecie w~Singapurze. Jego prace naukowe poruszają głównie elementy z~zakresu optyki kwantowej, relatywistycznej teorii informacji kwantowej, teorii względności oraz kwantowej teorii pola w~zakrzywionych czasoprzestrzeniach. Autor ceniony jest również za swoją działalność artystyczną. Jest znanym fotografem, producentem filmów krótkometrażowych, a~także kompozytorem muzyki elektronicznej.

Już ta krótka nota biograficzna o~autorze ukazuje, jak szerokie jest spektrum jego zainteresowań i~talentów. To widoczne jest również w~jego najnowszej książce
%\label{ref:RND4ypVr8msHU}(Dragan, 2019).
\parencite[][]{dragan_kwantechizm_2019}. %
 Już w~samym podtytule autor ujawnia swoje inspiracje eksperymentami Skinnera, przeprowadzanymi na szczurach. Zwierzęta te umieszczane były w~klatkach wyposażonych w~dźwignie podające pokarm. I~tylko na tym się koncentrowały, i~to im w~zupełności wystarczyło. Bezrefleksyjność szczurów wobec otaczającej ich klatki zestawiona została przez autora z~bezrefleksyjnością ludzi do przyjmowania otaczającej nas natury i~funkcjonowania rzeczywistości. Jako ludzie koncentrujemy się na zaspokojeniu podstawowych potrzeb, a~głębsza refleksja nad całą resztą otaczającej nas rzeczywistości jakoś nam umyka. Na szczęście naszą klatką interesują się fizycy, których rozważania okazują się być całkowicie zaskakujące, a~przez to trudne do przyjęcia. Autor, jako fizyk, stara się w~przystępny i~ciekawy sposób przyswoić ogółowi podstawowe mechanizmy rządzące otaczającą nas rzeczywistością na podstawie najlepiej opisujących ją na dzień dzisiejszy teorii. W~kwestii doboru analizowanych teorii fizycznych autor faktycznie stara się być ortodoksyjny, ograniczając się do analizy tylko tych najlepiej potwierdzonych i~powszechnie uznanych teorii ostatniego stulecia. W~sposób widoczny stroni od mnożących się rozwiązań kontrowersyjnych i~spekulatywnych. Frapujące jest więc, że już we wstępie stara się od absolutnie wszelakiego rodzaju ortodoksyjności odcinać. Autor mimo potężnej wiedzy we wspomnianych dziedzinach unika matematycznego formalizmu i~wzorów. Widać, że za cel nadrzędny obrał sobie, by książka był zrozumiana i~przystępna każdemu czytelnikowi, bez względu na stopień znajomości fizyki. Zdaje się, że oddają to nawet wykresy i~rysunki poglądowe, przedrukowane jako odręczne i~jakby wykonane pośpiesznie. Wszystko to, by nie spłoszyć czytelnika-laika atmosferą nazbyt akademicką i~poważną.

Kolejną z~bardzo widocznych inspiracji autora jest Richard Feynman i~jego słynna autobiografia \textit{Pan raczy żartować, Panie Feynman}
%\label{ref:RNDDEMYTSdM6Y}(1996).
\parencite*[][]{feynman_pan_1996}. %
 Ten jeden z~najgenialniejszych fizyków w~swojej autobiografii zdołał opisać i~kulisy prac nad budową bomby atomowej, i~tajniki otwierania zamków szyfrowych, i~radość gry na bongosach, i~spotkania z~Einsteinem, i~warunki życia w~Los Alamos, i~nawet flirty fizyków z~tancerkami w~Las Vegas. Podobnie autor \textit{Kwantechizmu} pisze ze swadą o~swoich podróżach i~przygodach, opisuje, jak to uczył się hipnotyzować, jak wyprowadzał w~pole mrówki czy biednych użytkowników budek telefonicznych, a~nawet uchyla rąbka tajemnicy ze swojej kariery fotograficznej. Co chwila można wyczuć, jak puszcza oko do czytelnika, żartuje, cytuje znane teksty piosenek, czy kultowe już dialogi filmowe. Całości dopełnia niezwykle potoczny język używany przez autora. Gdzie więc jest miejsce na wszystkie te wielkie, fizyczne zagadnienia, o~których wspominaliśmy wcześniej? Ano tak jak w~przypadku autobiografii Feynmana niejako przy okazji. \textit{Kwantechizm} jawi się przez to jako zbiór luźnych, miejscami zabawnych opowieści o~fizyce kwantowej. W~konsekwencji całość jest bardzo przystępna i~łatwa do przyswojenia. Książka nie pretenduje do miana podręcznika akademickiego mającego nauczyć teorii względności i~mechaniki kwantowej, ale chce przekazać czytelnikowi ich podstawowe intuicje. I~robi to bardzo dobrze w~sposób porywający, obrazowy i~bardzo ciekawy. Używane przykłady zapadają w~pamięć, świetnie przedstawiają wielkie fizyczne idee i~niejednokrotnie wywołują uśmiech na twarzy.

Wielką zasługą autora jest rzetelność na płaszczyźnie fizycznej. Autor to wysokiej klasy specjalista w~swojej dziedzinie, który przekazuje bardzo trudne i~nieintuicyjne zagadnienia w~takim stopniu i~w~taki sposób, że stają się dostępne każdemu. Jednocześnie autor nie ma złudzeń: dopóki nie nauczymy się rozwiązywać równań, to w~pełni i~tak nie zrozumiemy fizyki. Dostrzec jednak możemy, że odpowiedzi udzielane nam przez rzeczywistość w~perspektywie teorii względności i~mechaniki kwantowej okazują się absurdalne i~bardzo trudne do przyjęcia. I~to właśnie autor stara się uświadomić czytelnikowi. Jednoznacznie wykazuje, że większość naszych wyobrażeń o~świecie okazała się albo przybliżeniami, albo przesądami. Ba, nawet nasz codzienny język okazuje się pełen uproszczeń, które tylko utwierdzają nas w~naszych błędnych intuicjach. Nasze oczekiwania w~stosunku do rzeczywistości okazują się naiwne, a~ona sama zdaje się być o~wiele bardziej złożona.

W~całej książce autor często akcentuje swoje antyfilozoficzne i~antyreligijne stanowisko. Aspekty te aż nad wyraz wybrzmiewają w~jego wypowiedziach i~rozważaniach. Niestety nie ustrzegł się tu rażących uproszczeń, a~filozofię i~religię przedstawił w~sposób skrajnie jednostronny, czy wręcz infantylny. Autor dokonuje skrajnej polaryzacji przeciwstawiając wszelką wiarę nauce. Utożsamia myślenie religijne z~najbardziej skrajną formą fideizmu, podważającą znaczenie wszelkiej władzy poznawczej rozumu. W~swoich wywodach autor nie dostrzega, że religia była często stymulatorem rozwoju nauki, a~historia wielkich odkryć naukowych była także pisana przez ludzi głęboko wierzących. Przestrzenie te w~żaden sposób się nie wykluczają. Konflikt ten może pojawić się między nieoświeconą wiarą a~prymitywną nauką. Między wiarą otwartą a~nauką o~szerokich horyzontach nie ma żadnego konfliktu. Co więcej, w~ich relacji następuje wzajemne uzupełnianie się
%\label{ref:RND8rxbwBiNxF}(np. Heller, 2016, 2019; zob. także Rodzeń, 2021).
\parencites[np.][]{heller_czlowiek_2016}[][]{heller_nauka_2019}[zob. także][]{rodzen_teologia_2021}. %
 Jeszcze bardziej niezrozumiała wydaje się niechęć autora do filozofii. Jest to o~tyle dziwne, że nie sposób odmówić filozoficznych inspiracji autorom teorii względności i~mechaniki kwantowej – dziedzinami, którymi autor \textit{Kwantechizmu} zajmuje się na co dzień. Fizyka zmuszona jest do odwoływania się do filozofii chociażby na przestrzeni metodologii i~właściwej interpretacji 
%\label{ref:RNDHLWfCuH21I}(więcej zob. np. Heller, 2019a; zob. także Polak, 2019).
\parencites[więcej zob. np.][]{heller_how_2019}[zob. także][]{polak_philosophy_2019}. %
 Nawet sam Dragan wielokrotnie powołując się i~odnosząc do metodologii nauk, zupełnie nieświadomie wkracza nieświadomie na grunt rozważań filozoficznych. Innym gorzkim faktem jest, że autor nawołujący do ostrożności w~bezkrytycznym przyjmowaniu i~powielaniu niesprawdzonych informacji, sam takowych się nie ustrzegł. O~ile poprawność opisu zjawisk, doświadczeń fizycznych i~ich analizy przez znanego profesora akademickiego w~swojej dziedzinie nie podlega dyskusji, o~tyle w~warstwie historycznej Dragan dopuścił się bardzo poważnych błędów merytorycznych! Najpoważniejsze to nieprawdziwe przedstawienie słynnej sprawy Galileusza, przypisywanie scholastykom kosmologii egipskich, błędy w~cytowaniu Einsteina, czy wypaczenia historii Gaussa i~Faradaya. I~chociaż na wstępie swojej książki autor zwraca uwagę, że jako ludzie zbyt często poddajemy się całej serii uproszczeń, ideologii i~racjonalizacji, to niestety nie sposób oprzeć się wrażeniu, że sam stał się ich ofiarą. Tak oto nawołując do weryfikowania wszelkich informacji stworzył książkę, która sama staje się przestrzenią do takowych ćwiczeń.

Książka Andrzeja Dragana \textit{Kwantechizm czyli klatka na ludzi} na płaszczyźnie fizyki to świetna książka popularnonaukowa na początek fascynującej przygody z~teorią względności i~mechaniką kwantową. Pisana lekkim piórem, niewymagająca od czytelnika praktycznie żadnej wiedzy o~współczesnych zagadnieniach fizycznych, a~jednocześnie przystępnie je podająca, ilustrująca i~wyjaśniająca. Wymaga za to szerszej wiedzy historycznej i~chociaż podstawowej filozoficznej dla dokonania weryfikacji pomagającej wydobyć to, co miało być w~niej naprawdę wartościowe. Niestety, książka przy okazji popularyzacji nauki wprowadza ogromne zamieszanie i~ostatecznie robi wiele szkód dla głębszego rozumienia fizyki, a~także znaczenia i~funkcji filozofii i~religii. I~tak oto może stać się atrakcyjną i~pociągającą, ale jednak pułapką-klatką na ludzi myślących.



%-------------------------------


\selectlanguage{english}
\vspace{5mm}%
\begin{flushright}
{\chaptitleeng\color{black!50}{A cage for thinking people}}
\end{flushright}

%\vspace{10mm}%
{\subsubsectit{\hfill Abstract}}\\
{Lorem ipsum dolor sit amet, consectetur adipiscing elit. Proin a blandit augue. Suspendisse tortor sem, sollicitudin id sollicitudin eget, bibendum et augue. Etiam bibendum at sem sed sollicitudin. Vestibulum quis sagittis neque. Orci varius natoque penatibus et magnis dis parturient montes, nascetur ridiculus mus. Suspendisse ac augue vitae felis eleifend egestas non vel ante. Aliquam mauris lacus, vestibulum eget pretium suscipit, laoreet sit amet orci. Pellentesque id dolor non orci pellentesque suscipit. Sed tincidunt luctus ultrices. Pellentesque auctor molestie tempor. Nunc in commodo arcu. Pellentesque nec suscipit sem. Suspendisse venenatis, ex eu fringilla aliquam, urna augue auctor diam, sit amet blandit mi nisl id elit. Donec libero nunc, venenatis ac ultrices eget, hendrerit eget nunc. Donec laoreet lacus sit amet velit interdum, quis viverra nibh fermentum.}\par%
\vspace{2mm}%
{\subsubsectit{\hfill Keywords}}\\%
{lorem, ipsum, dolor, sit, amet.}%

\selectlanguage{polish}

\end{newrevplenv}
