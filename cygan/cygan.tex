\begin{newrevplenv}{Milena Cygan}
	{Pomiędzy tradycją a~współczesnością}
	{Pomiędzy tradycją a~współczesnością}
	{Pomiędzy tradycją a~współczesnością}
	{Uniwersytet Papieski Jana Pawła II w Krakowie}
	{Kamil Trombik, \textit{Koncepcje filozofii przyrody w~Papieskiej Akademii Teologicznej w~Krakowie w~latach 1978--1993. Studium historyczno-filozoficzne}, Wydawnictwo «scriptum», Kraków 2021, ss.~263.}




\lettrine[loversize=0.13,lines=2,lraise=-0.03,nindent=0em,findent=0.2pt]%
{``I}{}stnieje wiele różnych mniej lub bardziej trafnych określeń filozofii przyrody. Jedno z~takich określeń brzmi, że jest to dyscyplina należąca do nauk humanistycznych, w~ramach których prowadzi się filozoficzną refleksję nad wynikami badań nauk przyrodniczych […]. Nawet jeśli ograniczyć się do tego typu definicji i~pominąć wszystkie inne określenia tej dziedziny, to okazuje się, że tak pojmowaną filozofię przyrody można uprawiać --~o~czym przekonują liczne przykłady z~historii nauki --~na wiele rożnych sposobów''
%\label{ref:RNDfbMKXH4vg6}(Heller, Pabjan, 2007, s.~12).
\parencite[][s.~12]{heller_elementy_2007}. %
 Jednym z~takich właśnie sposobów jest ,,filozofia w~nauce''. Pierwszym i~podstawowym zadaniem tej dziedziny jest odkrywanie w~naukach przyrodniczych wątków i~zagadnień o~charakterze filozoficznym oraz ich analiza w~oparciu o~ogólnie akceptowalne zasady logiki i~metodologii 
%\label{ref:RNDlEY2dXdfb5}(Heller, 2019; zob. także Polak, 2019).
\parencites[][]{heller_how_2019}[zob. także][]{polak_philosophy_2019}. %
%Polak, 2019b
Obecnie jest to wiodąca forma uprawiania filozofii przyrody na Uniwersytecie Papieskim Jana Pawła II w~Krakowie i~w~krakowskim środowisku ,,filozofujących naukowców''. Dla każdego, kto przeszedł formację akademicką w~tym ośrodku naukowym wydaje się to być czymś oczywistym. Czy jednak sama obecność tego stylu uprawiania filozofii na tejże uczelni jest tak oczywista, zwłaszcza jeśli wziąć pod uwagę, iż jest to instytucja katolicka, a~większość placówek tego typu preferowała raczej ujęcia neoscholastyczne? Jak doszło do ukształtowania się tego podejścia na Wydziale Filozoficznym UPJPII? Jak to się stało, że dziś jest ono charakterystyczną cechą tego ośrodka naukowego? Na te pytania próbuje odpowiedzieć Kamil Trombik w~monografii \textit{Koncepcje filozofii przyrody w~Papieskiej Akademii Teologicznej w~Krakowie w~latach 1978--1993}
%\label{ref:RNDpoAeVncRUj}(2021),
\parencite*[][]{trombik_koncepcje_2021}, %
 dowodząc, że zalążki tego nowego podejścia do uprawiania filozofii przyrody znajdują się już u~samych początków ośrodka filozoficznego Papieskiej Akademii Teologicznej (PAT) i że jest ono częścią dłuższej tradycji uprawiania filozofii w~kontekście nauk, której stał się  spadkobiercą.

Książka Kamila Trombika składa się z~trzech głównych rozdziałów, odpowiadających trzem fazom formowania się Wydziału Filozoficznego PAT i~jego specyficznego nurtu filozofii przyrody. Pierwszy rozdział dotyczy źródeł filozoficzno-przyrodnicznych koncepcji rozwijanych na Papieskim Wydziale Teologicznym (przekształconym później w~PAT), a~także historycznych uwarunkowań, które zadecydowały o~tym, że filozofia przyrody cieszyła się w~tym ośrodku sporym zainteresowaniem. Przyrodnicze zorientowanie nie pojawiło się tam bowiem nagle i~przypadkowo, lecz było efektem splotu wielu historycznych czynników. Jeden z~nich wiąże się z~działalnością uczonych w~katedrze filozofii chrześcijańskiej Wydziału Teologicznego UJ, założonej w~1882 roku, której kierownikiem został Stefan Zachariasz Pawlicki. To właśnie on nadał kierunek, w~jakim zaczęła się rozwijać krakowska filozofia chrześcijańska i~przyczynił się do tego, iż nie działo się to całkowicie po linii neoscholastycznej. Trombik na kartach swojej książki polemizuje więc z~autorami, którzy chcieliby widzieć Pawlickiego jako kontynuatora czy modernizatora myśli tomistycznej. Analizując dorobek tego myśliciela, wykazuje, że był on raczej zwolennikiem budowania filozofii na bazie wyników badań nauk przyrodniczych (oraz introspekcji), nie zaś, jak filozofowie orientacji neotomistycznej, analizowania danych nauk szczegółowych w~oparciu o~założenia określonego systemu metafizycznego. Drogą, którą wyznaczył Pawlicki, częściowo podążyli jego następcy w~katedrze filozofii chrześcijańskiej, mianowicie Franciszek Gabryl, Konstanty Michalski i~Jan Salamucha\footnote{Wszystkich ich łączyło zainteresowanie filozofią przyrody oraz wniesienie istotnego wkładu w~kreowanie postawy uprawiania filozofii w~kontakcie z~naukami szczegółowymi. Oczywiście, warto podkreślić, że postawa taka była typowa dla tzw. ,,filozofii chrześcijańskiej'', gdyż polemizowała ona wówczas z~pozytywizmem, zatem zagadnienia naukowe były traktowane jako najważniejsze z~apologetycznego puntu widzenia.}.

Obok wspomnianych wyżej filozofów, autor omawia także wkład w~tworzenie się tradycji interdyscyplinarnej innych uczonych niezwiązanych ze wspomnianą katedrą i~neoscholastyką, nawiązujących raczej w~swoich pracach do empiriokrytycyzmu. Trombik prezentuje poglądy W. Heinricha i~M. Straszewskiego oraz innych filozofów i~,,filozofujących przyrodników'' (T. Garbowskiego, M. Smoluchowskiego, W. Natansona, J. Matellmana, Z. Zawirskiego, B. Gaweckiego), którzy podejmowali zagadnienia filozoficzne w~kontekście nauk formalnych i~przyrodniczych. Myśliciele ci przejawiali niechęć wobec spekulacji filozoficznych niezakorzenionych w~empirii, a~filozofię traktowali jako dyscyplinę związaną genetycznie (a po części także przedmiotowo) z~przyrodoznawstwem
%\label{ref:RNDtPC7ghio2D}(Trombik, 2021, s.~86).
\parencite[][s.~86]{trombik_koncepcje_2021}. %
 Niestety, rozwój krakowskiej filozofii przyrody został zahamowany przez wybuch drugiej II wojny światowej. W~dalszej części rozdziału autor skupia się więc na okresie pomiędzy rokiem 1945 a~1978 ukazując, jak prace na polu badań interdyscyplinarnych najpierw podjęli katoliccy myślicie związani z~Wydziałem Teologicznym UJ, a~następnie z~nowo powołanymi ośrodkami naukowymi, takimi jak Papieski Wydział Teologiczny czy Ośrodek Badań Interdyscyplinarnych (OBI). Niebagatelną rolę odegrała w~tym okresie działalność Kazimierza Kłósaka i~Tadeusza Wojciechowskiego --~neotomistów pragnących modyfikować tradycyjne rozwiązania scholastycznej filozofii przyrody, godząc je z~wynikami nauk przyrodniczych. Bardzo ważne okazały się też inicjatywy Karola Wojtyły i kolejnego pokolenia filozofów, wśród których najważniejszą role odegrali Józef Życiński i~Michał Heller.

Drugi rozdział oscyluje już wokół pytania o~to, w~jaki sposób rodziła się i~jak była uprawiana filozofia przyrody na PWT i~PAT w~latach 1978--1986. W~pierwszej kolejności autor przedstawia historię formowania się nowych jednostek naukowych, mianowicie Wydziału Filozoficznego PAT i~OBI, a~także powstania czasopism: \textit{Zagadnienia Filozoficzne w~Nauce} i \textit{Philosophy in Science.} Następnie przybliża poglądy K. Kłósaka i~T. Wojciechowskiego, związanych w~tym czasie z~krakowską uczelnią, a~hołdujących tradycji neotomistycznej. Analizując dorobek naukowy tych dwóch uczonych, Trombik ukazuje ich nowatorskie --~w~ramach neoscholastycznych rozwiązań --~propozycje filozoficzne, mające na celu zharmonizowanie tradycyjnej filozofii z~wynikami nauk. Autor prezentuje sposób, w~jaki uczeni ci budowali pomosty łączące filozofię i~naukę, podkreślając, że taka postawa niekoniecznie była standardem pośród myślicieli tej opcji filozoficznej. Z~drugiej strony obnaża niektóre niekonsekwencje w~ich podejściu, a~także ignorowanie oraz brak odniesień do filozofii nauki i~kwestii metodologicznych. Podejście to, jak wynika z~analiz Trombika, nie cieszyło się w~Krakowie większym zainteresowaniem i~nie doczekało się kontynuacji.

Trzecią część swojej pracy poświęca autor rozwojowi filozofii przyrody na PAT w~latach 1987--1993. Bada, jak przyjęła się nowa filozofia przyrody (,,filozofia w~nauce'') i~jak wchodziła w~stadium swojego rozkwitu. Do roku bowiem 1987 filozofia przyrody rozwijała się tam dwutorowo: z~jednej strony oddziaływała tradycyjna, neotomistyczna filozofia przyrody, z~drugiej zaś coraz większym zainteresowaniem i~uznaniem cieszyło się podejście badawcze zapoczątkowane przez Hellera i~Życińskiego. W~rozdziale tym Trombik referuje, w~jaki sposób następował rozwój i~recepcja ,,filozofii w~nauce'' wśród uczonych związanych z~PAT na przestrzeni wspomnianych lat, zwłaszcza w~okresie, gdy dziekanem Wydziału Filozoficznego został J. Życiński. Autor dowodzi, że na przełomie dekad to właśnie filozofowie przyrody, zwłaszcza ci związani z~koncepcją ,,filozofii w~nauce'' byli, obok środowiska Józefa Tischnera, jedną z~najbardziej aktywnych grup na PAT. O~tym, że tak się działo, świadczą organizowane wówczas sympozja, konferencje, aktywność wydawnicza czy współpraca z~licznymi, zagranicznymi ośrodkami naukowymi. O~istotnej roli filozofii przyrody w~tej nowej odsłonie świadczy też jej znacząca obecność w~ówczesnych programach nauczania, a~także zanikanie filozofii przyrody uprawianej w~duchu tomizmu, która traciła na znaczeniu już pod koniec lat 80., a~na przełomie kolejnej dekady już praktycznie nie oddziaływała w~obrębie PAT. Ów rodzaj filozofii reprezentował jeszcze T. Wojciechowski, którego rozwiązania filozoficzne Trombik poddaje krytycznej ocenie, pokazując, że pomimo podejmowania prób uwspółcześnienia tej tradycji, nie powiodło mu się przezwyciężenie zależności niektórych ustaleń metodologicznych i~terminologicznych typowych dla filozofii tomistycznej. Niemniej autor podkreśla, że mimo niepowodzeń, Wojciechowski wpisuje się w~poczet myślicieli tamtego okresu, którzy dostrzegali wagę problemów pojawiających się na styku filozofii i~nauki (Wojciechowski interesował się biologią i~psychologią). Innym uczonym, o~którym wspomina Trombik, zajmującym się wówczas filozofią przyrody był jezuita Piotr Lenartowicz. W~jego podejściu uwidaczniały się próby pogodzenia metafizyki arystotelesowsko-tomistycznej z~nowoczesną biologią. Niestety, ze względu na swoją anachroniczność okazały się one niezadowalające i~niesatysfakcjonujące w~środowisku PAT, w~którym rozwijała się coraz prężniej ,,filozofia w~nauce'', promowana już nie tylko przez Hellera i~Życińskiego, ale również przez krąg ich uczniów i~współpracowników (m.in. W.~Skoczny, W.~Wójcik, Z.~Wolak, Z.~Liana). Badania prowadzone przez Trombika skłaniają go ostatecznie do sformułowania hipotezy, iż w~przypadku ,,filozofii w~nauce'' mamy do czynienia z~nową szkołą filozoficzną. Autor dostrzega też pewne analogie w~rozwoju krakowskiego ośrodka filozoficznego w~stosunku do wczesnego etapu kształtowania się środowiska skupionego wokół Kazimierza Twardowskiego, z~którego potem wyłoniła się szkoła lwowsko-warszawska.

Recenzowana pozycja w~bardzo klarowny sposób pokazuje, iż filozofia przyrody, a~zwłaszcza noszący znamiona szkoły filozoficznej projekt ,,filozofii w~nauce'', odegrały istotną rolę w~historii Papieskiej Akademii Teologicznej w~Krakowie. Z~tego względu należy stwierdzić, że i~monografia K. Trombika, będąca historyczno-filozoficznym studium fragmentu dziejów tego ośrodka naukowego, wpisuje się w~poczet ważniejszych publikacji o~doniosłym znaczeniu dla dzisiejszego Uniwersytetu Papieskiego Jana Pawła II oraz rozwijającego się w~jego strukturach środowiska filozoficznego. Jest też cennym wkładem do badań nad historią filozofii polskiej, a~szczególnie krakowskiej, która jeszcze nie jest dostatecznie opracowana. Praca Trombika jest tym bardziej ważna, że krakowska uczelnia wraz ze swoim Wydziałem Filozoficznym ma już prawie pół wieku, zatem ustalenie jej ideowych korzeni i~filozoficznej tożsamości wydaje się istotnym zadaniem. Autor stawia więc pytanie o~to, jakie koncepcje filozoficzne były reprezentowane w~środowisku PAT w~latach 1978--1993, a~zatem w~pierwszym i~decydującym okresie formowania się Wydziału Filozoficznego tejże uczelni. Oczywiście Trombik nie tylko przedstawia poszczególne koncepcje, ale również próbuje zbadać, jakie czynniki zadecydowały o~tym, że problematyka związana z~filozofią przyrody stała się wiodącym i~reprezentatywnym dla tego ośrodka kierunkiem, a~także dlaczego taką popularność zyskała właśnie ,,filozofia w~nauce'' rozwijana pierwotnie przez Hellera i~Życińskiego. Odpowiedź na powyższe pytanie została poniekąd zawarta już w~pierwszym rozdziale książki, który, jak referowano uprzednio, ukazuje długą tradycję krakowskiego uprawiania filozofii w~dialogu z~nauką. Początki praktykowania postawy interdyscyplinarnej, otwartej na naukę, autor lokalizuje w~latach 80. XIX wieku. Takie określenie źródeł omawianego stylu uprawiania filozofii nie jest zbyt nowatorskie. Podąża tu Trombik za ujęciem, które dużo wcześniej zaproponowali inni historycy filozofii polskiej, przede wszystkim zaś Paweł Polak w~cyklu swoich artykułów poświęconych początkom krakowskiej filozofii przyrody, stawiając pytania o~to, kiedy, w~jakich okolicznościach i~z~jakich filozoficznych pobudek kierunek ten rozwinął się w~Krakowie oraz na ile wpłynęły one na ukształtowanie się specyficznego stylu uprawiania filozofii przyrody. Między innymi w~swoim artykule ,,U~źródeł krakowskiej filozofii przyrody''
%\label{ref:RNDxgs9OfhWYQ}(Polak, 2011, s.~135)
\parencite[][s.~135]{polak_u_2011} %
 początki te wiąże z~takimi wydarzeniami, jak powołanie sekcji filozoficznej w~ramach Polskiego Towarzystwa Przyrodników im. Kopernika oraz działalnością W. Heinricha, postulującego stworzenia nowej filozofii przekraczającej empiriokrytycyzm i~M. Straszewskiego, inicjującego refleksję nad filozoficznymi implikacjami teorii naukowych. Z~kolei w~artykule ,,Rola Stefana Z. Pawlickiego w~kształtowaniu się krakowskiego ośrodka przyrody''
% \label{ref:RND16cC9AfO2l}\textcolor{black}{(Polak, 2017)}
\parencite{polak_rola_2017} %
  ukazuje go jako myśliciela, który mógł mieć wpływ na wytworzenie się w~Krakowie podejścia określanego mianem ,,klimatu współpracy interdyscyplinarnej''. Zatem ta część monografii Trombika, chociaż bardzo interesująca, jest jednocześnie najmniej oryginalna, choć daje najbardziej kompletny obraz tego procesu rozproszony wcześniej w~wielu publikacjach. 

Wprawdzie rozdział dotyczący filozoficzno-historycznych źródeł ,,filozofii w~nauce'', uzasadniający jej uprzywilejowane miejsce na PAT, nie jest z~perspektywy całości monografii częścią najbardziej w~tej pozycji istotną, to jednak warto jeszcze w~tym kontekście wspomnieć, że pewne elementy referowanego w~książce interdyscyplinarnego podejścia są w~środowisku krakowskim dużo starsze. Można je już dostrzec u~Józefa Kremera 
%\label{ref:RNDeSqGgpTA2p}(Polak, 2019a, s.~255)
\parencite[][s.~255]{polak_miedzy_2019} %
 oraz wśród uczonych skupionych wokół Towarzystwa Naukowego Krakowskiego, gdzie preferowano model interdyscyplinarności i~związki filozofii z~nauką. A~nawet jeszcze wcześniej, bowiem inicjatywa spotkań elity intelektualnej Krakowa, podjęta przez Walentego Litwińskiego i~Samuela Bandtkiego, i~zorganizowana w~Towarzystwo Naukowe Krakowskie, była w~rzeczywistości pomysłem jeszcze starszym. Krakowscy profesorowie wcielili bowiem w~życie idee promowane przez Hugona Kołłątaja, który jako reformator i~rektor Akademii Krakowskiej domagał się powstania ośrodka, który miałby na celu rozwijanie badań naukowych 
%\label{ref:RNDDDnXn0y2sP}(Rederowa, 1998, s.~16).
\parencite[][s.~16]{rederowa_z_1998}. %
 Wśród jego licznych postulatów (oprócz zgromadzenia uczonych w~towarzystwo naukowe wychodzące poza struktury akademickie, choć ściśle z~nimi związane) znajdują się i~takie, które zalecają całkowite zerwanie ze scholastyką i~nakazują, by badania nad poszczególnymi problemami naukowymi, zwłaszcza tymi o~doniosłym znaczeniu praktycznym, prowadziły szerokie kolektywy uczonych o~różnych pokrewnych specjalnościach. Owocem tych studiów miało być kompleksowe rozwiązywanie problemów. Oznaczało to wówczas wielki przełom, o~którym Kołłątaj wiedział, że będzie trudny do zaakceptowania przez uczonych i~profesorów przyzwyczajonych do tradycji scholastycznej w~nauce, którzy nie będą w~stanie tych rozwiązań zaakceptować i~zrealizować 
%\label{ref:RNDgBj3ZJQNg9}(Hinz, 1962, s.~84).
\parencite[][s.~84]{hinz_refleksja_1962}. %
 Chociaż Kołłątajowi nie udało się wcielić w~życie wszystkich swoich pomysłów, jego idee jeszcze długo rezonowały w~środowisku krakowskim, którego charakterystyczną cechą postawy filozoficznej stała się wyniesiona z~Oświecenia niechęć do maksymalistycznych i~systemotwórczych filozofii, jak i~spekulatywnej metodologii\footnote{Nic więc dziwnego, że podobnie jak nie darzono większą sympatią scholastyki, tak i~z dystansem podchodzono do romantyzmu (oczywiście były wyjątki, np. Józef Kremer). Ufundowaną na niemieckich dialektycznych koncepcjach filozofię romantyczną krytykowano za brak odniesienia do doświadczenia i~ujmowania wszystkiego jedynie z~perspektywy metafizyki 
%\label{ref:RNDLzC5B8dE9O}(Wiszniewski, 1839, s.~254).
\parencite[][s.~254]{wiszniewski_uwagi_1839}.%
}.
O~ile więc można przyjąć, że filozofia przyrody rozwijana na PAT była źródłowo połączona z~działalnością filozofów ostatnich dwóch dekad wieku XIX, o~tyle postawa tych uczonych okazuje się być częścią dłuższej tradycji, sięgającej co najmniej końca wieku XVIII. Sugestia autora, że filozofia przyrody na PAT, wraz ze swoim nurtem ,,filozofii w~nauce'', charakteryzującym się specyficzną, interdyscyplinarną postawą badawczą jest mocno zakorzeniona w~krakowskich filozoficznych tradycjach, wydaje się więc uzasadniona.

Innym ważnym elementem ukazanym w~książce Trombika jest rola, jaką odegrał w~kształtowaniu się filozofii przyrody na PAT krakowski kardynał Karol Wojtyła. O~ile jego zaangażowanie w~powstanie Papieskiej Akademii Teologicznej jest zasadniczo znane, o~tyle niekoniecznie wiadomy szerszemu gronu może pozostawać fakt, że wśród jego licznych zasług można wymienić także inspirowanie praktycznego rozwijania ,,filozofii w~nauce''. Co prawda, zazwyczaj Wojtyłę kojarzy się z~problematyką antropologiczną i~etyczną, jednak ważne jest podkreślenie tego, że żywo interesował się zagadnieniami pochodzącymi z~nauk szczegółowych. Źródła, które przytacza autor, ukazują Wojtyłę jako jednego z~głównych animatorów ruchu interdyscyplinarnego i~promotora filozofii uprawianej w~dialogu z~nauką i~wiarą. Opierając się na materiałach archiwalnych, Trombik prezentuje zaangażowanie późniejszego papieża w~rozwój tego typu refleksji, podkreślając jego związek z~krakowską tradycją, do której sam Wojtyła się przyznaje, wspominając, że jego stosunek do nauki formowali profesorowie w~Krakowie 
%\label{ref:RNDecVbmURly0}(Trombik, 2021, s.~71).
\parencite[][s.~71]{trombik_koncepcje_2021}. %
 Wojtyła inicjował lokalny ośrodek poprzez organizowanie szeregu wydarzeń o~charakterze interdyscyplinarnym (sympozja, Studium Myśli Współczesnej), podczas których wyłoniło się pytanie o~sposób uprawiania filozofii w~kontekście współczesnej nauki i~współpracy filozofów i~naukowców. Ponadto dzięki jego staraniom powołano Instytut Filozoficzny na PWT (przekształcony później w~Wydział Filozoficzny PAT). W~świetle źródeł i~wypowiedzi filozofów związanych z~Krakowem, do których odwołuje się autor, okazuje się, że klimat interdyscyplinarnych badań i~dyskusji nie byłby możliwy, gdyby nie działalność Wojtyły na tym polu. Oczywiście, nie byłby on także możliwy, gdyby nie praca wielu innych filozofów i~uczonych, może nie tak ,,sławnych'' jak Wojtyła, Heller i~Życiński, którzy przyczynili się do rozwoju filozofii przyrody na PAT oraz promowania ,,filozofii w~nauce'' w~jej strukturach, a~których wkład został doceniony i~upamiętniony na kartach recenzowanej monografii. I~to właśnie ten aspekt pracy Trombika stanowi najbardziej oryginalną i~wartościową część całego opracowania.

Podsumowując, w~książce Trombika pogłębionych analiz doczekał się ważny okres historii rozwoju filozofii przyrody na obecnym UPJPII. Badania autora, prowadzone w~ramach pracy doktorskiej, których owocem jest recenzowana monografia, nie pozostawiają wątpliwości, że filozofia przyrody stanowiła ważny element zainteresowań filozofów związanych z~PAT i~jej Wydziałem Filozoficznym, a~pierwsze piętnaście lat jego istnienia znacząco wpłynęły na uformowanie się jej specyficznego i~wyróżniającego stylu uprawiania filozofii. Od początku swego istnienia Wydział Filozoficzny PAT był areną, na której ścierały się różne koncepcje i~filozoficzne podejścia, od tych w~duchu klasycznych arystotelesowsko-tomistycznych, do zupełnie nowego nurtu ,,filozofii w~nauce''. To, co wartościowe w~książce i~co udało się znakomicie wydobyć autorowi to pokazanie, że pomimo tego pluralizmu podejść badawczych istniejących w~obrębie Wydziału Filozoficznego, cechą wspólną i~wyróżniającą wszystkie opcje filozoficzne była otwartość na współczesne im wyzwania płynące z~rozwoju nauk oraz podejmowanie próby zmierzenia się z~ówczesnymi problemami (również z~tymi pochodzącymi ze strony wrogo nastawionego obozu ideologii marksistowskiej). Pośród tych różnych tradycji ostatecznie ,,zwyciężyła'' ,,filozofia w~nauce'', oddziałując także daleko poza ośrodek krakowski. To pierwsza monografia poświęcona w~całości zagadnieniu, które do pory nie zostało gruntownie zbadane. Stanowi też dobry punkt wyjścia do kontynuowania badań nad rozwojem filozoficzno-przyrodniczej refleksji w~kolejnych latach działalności tej jednostki naukowej. Ponadto ze względu na ukazanie kształtowania się koncepcji filozofii przyrody na PAT na tle szerszego historycznego kontekstu rozwoju filozoficznych tradycji w~Krakowie, jest bodźcem do podjęcia badań nad rolą interdyscyplinarności w~pierwszej połowie dziewiętnastego wieku. Co więcej, autor stawia też w~swojej pracy sporo hipotez, jak choćby wspomniane pytanie: czy ,,filozofię w~nauce'' można traktować jako szkołę filozoficzną? Udzielenie jednoznacznej odpowiedzi wymaga bardziej zaawansowanych badań w~tym zakresie. Te hipotezy to przyczynki do podejmowania kolejnych studiów, zatem można stwierdzić, że książka obfituje w~perspektywy badawcze, a~jako taka ukazuje znaczenie tego ośrodka naukowego, podkreślając jego wartość w~dziejach rozwoju polskiej nauki i~filozofii.


%-------------------------------


\selectlanguage{english}
\vspace{5mm}%
\begin{flushright}
{\chaptitleeng\color{black!50}{Between tradition and modernity}}
\end{flushright}

%\vspace{10mm}%
{\subsubsectit{\hfill Abstract}}\\
{The article is a review of Kamil Trombik's book, in which he presents particular concepts of the philosophy of nature at the Pontifical Academy of Theology in Kraków in the years 1978 to 1993. It was the first and decisive period in the formation of the Faculty of Philosophy at the Academy.
The goal of the monograph was to demonstrate the factors that contributed to philosophy of nature becoming one of the most prominent and representative trends in this academic center, as well as to attempt to answer the question of why ``philosophy in science,'' developed initially by Michał Heller and Józef Życiski, became the main style of doing philosophy of nature there.
In the reporting part of the review main problems that the author raises are presented. They are collected in three chapters of his work, which corresponds to three initial phases of the formation of the philosophical department
at the Pontifical Academy of Theology.
Then, in the critical part, some aspects of Trombik's work are assessed. First of all, the attention is paid to the part concerning the determination of the sources of ``philosophy in science'' which---although it seems to be the most interesting---is also the least original part of the work.
Next, the contribution of Karol Wojtya and many other lesser-known scientists and philosophers to the formation of an interdisciplinary atmosphere and the promotion of ``philosophy in science'' in the structures of The Pontifical Academy of Theology and the Krakow intellectual milieu is also appreciated.
Many hypotheses and research perspectives in Trombik's book are highlighted in the review, demonstrating the importance of this Krakow research center (Philosophy Department at the Pontifical Academy of Theology) for the history of Polish science and philosophy.
}\par%
\vspace{2mm}%
{\subsubsectit{\hfill Keywords}}\\%
{philosophy in science, philosophy of nature, interdisciplinarity, Faculty of Philosophy at the Pontifical Academy of Theology in Krakow, Michał Heller, Józef Życiński.}%

\selectlanguage{polish}

\end{newrevplenv}
