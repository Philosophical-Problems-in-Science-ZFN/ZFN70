\ramkaart{Imię Nazwisko}
{Tytuł rozdziału.\\\chapsubtit{Podtytuł rozdziału}}
{Tytuł rozdziału. Podtytuł rozdziału}
{Tytuł rozdziału. Podtytuł rozdziału}

\lettrine[loversize=0.13,lines=2,lraise=0.00,nindent=0em,findent=0.2pt]%
{W}{}ielu filozofów i językoznawców twierdziło, że gdy identyfikujemy przedmioty, zarówno w języku, jak i w postrzeganiu, kierujemy się względami praktycznymi, potrzebami, wolą -- krótko mówiąc, swego rodzaju interesem, czy to gatunkowym (a więc ogólnoludzkim), czy to swoistym dla danej społeczności, cywilizacji czy jednostki. Wydaje się to kłócić z przeświadczeniem, że te przedmioty muszą istnieć, chyba że traktujemy to przeświadczenie jako jedną z wielu równouprawnionych wizji świata. Ta sama zasada wielości równosilnych perspektyw stosuje się a fortiori do języka metafizyki.

Tak więc powracamy tam, gdzie nasz horror bierze swój początek. Jakże mam trwać przy danym języku (czy jakimś szczególnym punkcie widzenia, z którego patrzę na świat, albo regule interpretacji całości doświadczenia), nie przyznając mu uprzywilejowanej poznawczej mocy? A jeśli pretenduję do dysponowania wyższym czy może nawet absolutnym językiem, to albo nadawałby się on tylko do mówienia o innych językach, nie zaś o rzeczywistości, do której się odnoszą, albo byłby standardowym językiem, a inne byłyby jego niekompletnymi dialektami. W tym drugim przypadku byłby to rzeczywiście boski język, absolutny i zawierający wszelkie wyobrażalne punkty widzenia. Lecz język taki jest niemożliwy; nawet Bóg, przemawiając ustami proroka, musiał się przełożyć na język ludzki; przekład jest niechybnie zniekształcony, a nam brak dostępu do oryginału. W przypadku pierwszym język mój (język pierwszego stopnia, język rzeczy) nie może wprawdzie rościć sobie pretensji do jakiejkolwiek pozycji uprzywilejowanej, lecz w tym języku nie sposób byłoby nieobecność owej pozycji wyrazić: by to uczynić, musiałbym swój język porzucić i przejść do super(czy meta-) języka -- ale w takim języku moje stanowisko, jako że szczególne, nie dałoby się wyrazić.

\myquote{
Gdy więc wielkodusznie powiadam: „wszystkie stanowiska metafizyczne są równie dobre”, sam żadnego stanowiska nie zajmuję, po prostu wyrażam zasadę tolerancji, która, jakkolwiek chwalebna, ma charakter formalny i nigdy nie wyda, czy choćby zainspiruje, żadnej metafizycznej idei. Lecz próbując zachować tę zasadę i obstawać zarazem przy swym szczególnym punkcie widzenia, popadam w niekonsekwencję, jako że twierdzę wtedy, iż „stanowisko moje jest tak samo dobre jak każde inne, mimo że jest nie do pogodzenia z żadnym innym”.

A jeśli tak mówię, nie mogę w zrozumiały sposób wyjaśnić, w jakim sensie to stanowisko jest moje, w przeciwieństwie do innych. Niestety, tolerancyjna wspaniałomyślność nie pozwala uciec od paradoksu samoodniesienia.
}

\noindent Leon Chwistek, logik i malarz, w wydanej w 1921 roku książce Wielość rzeczywistości sugerował istnienie czterech rodzajów wzajemnie niezależnych (a zatem przypuszczalnie nieinterferujących z sobą wzajem) rzeczywistości: rzeczy, takie jak je postrzega zdrowy rozsądek; rzeczy nauk fizycznych; wrażeń; i wyobraźni. Znajdują one swój artystyczny wyraz w malarstwie – odpowiednio: prymitywistycznym, naturalistycznym, impresjonistycznym, futurystycznym\footnote{To jest przypis dolny. Jeśli niezależne od siebie pokłady rzeczywistości wymagają dla swego opisu niezależnych języków, sugeruje to, że języki te są całkowicie nieprzekładalne; skoro tak, to w istocie można sądzić, że różne wizje świata współistnieć mogą w doskonałej wzajemnej obojętności}. Lecz wielość ta, dowodził, pozwala na dowolną liczbę równoprawnych poglądów na świat, z których żadnego nie można dowieść, lecz każdy jest do przyjęcia pod warunkiem, że nie próbuje zmonopolizować prawdy. Dopuszczalne stają się różne odpowiedzi na tradycyjne pytania, jak te o wolność woli, relację między ciałem a duchem, obiektywność wartości, jeśli zakres odniesienia ogranicza się do jednej lub niektórych spośród owych czterech rzeczywistości.

\noindent Leon Chwistek, logik i malarz, w wydanej w 1921 roku książce Wielość rzeczywistości sugerował istnienie czterech rodzajów wzajemnie niezależnych (a zatem przypuszczalnie
nieinterferujących z sobą wzajem) rzeczywistości: rzeczy, takie jak je postrzega zdrowy rozsądek

\section{Podtytuł 1 stopnia}

\noindent Teoria wielu -- jakkolwiek wyodrębnianych -- rzeczywistości, jeśli nawet w tym przypadku stworzona dla metafizycznej interpretacji malarstwa, proponuje przekonujący i kuszący obraz świata. Jednakże jako propozycja epistemologiczna nie jest w stanie -- czy to w wersji Chwistka, czy Williama Jamesa -- poradzić sobie z wciąż tą samą trudnością: jak dowieść wyższości pewnej teorii bytu w tym samym języku, w którym została wypowiedziana? Mieliżbyśmy utrzymywać, że twierdzenie „wszystko jest konieczne” jest równie prawomocne co twierdzenie „poza związkami logicznymi nic nie jest konieczne” i że doktryna, według której słowo „ja” nie ma odniesienia, jest nie mniej prawdziwa niż ta, wedle której cokolwiek ma odniesienie, jest względne w stosunku do „ja”?

\subsection{Podtytuł 2 stopnia}

\noindent Jeśli niezależne od siebie pokłady rzeczywistości wymagają dla swego opisu niezależnych języków, sugeruje to, że języki te są całkowicie nieprzekładalne; skoro tak, to w istocie można sądzić, że różne wizje świata współistnieć mogą w doskonałej wzajemnej obojętności: nie mogą być ze sobą konfrontowane ani między sobą sprzeczne. Ale twierdzenie, że nie mogą być konfrontowane, jest wyrażone w języku innym, wyższego stopnia, nie nadającym się do celów metafizycznych. I tu powraca ten sam kłopot: albo ograniczamy się do tego wyższego języka i wtedy werdykty nasze nie mają znaczenia dla rzeczywistych problemów, z których filozofia żyje, albo przyjmujemy pewną metafizyczną perspektywę i głosimy, że perspektywy tej, jako zamkniętej, nie da się zharmonizować z żadną inną ani też innej przeciwstawić -- i wtedy też, w wyniku tej samointerpretacji, perspektywa nasza jest bez znaczenia dla rzeczywistych problemów, z których filozofia żyje.

\sectionno{Podtytuł 1 stopnia}

\noindent Teoria wielu -- jakkolwiek wyodrębnianych -- rzeczywistości, jeśli nawet w tym przypadku stworzona dla metafizycznej interpretacji malarstwa, proponuje przekonujący i kuszący obraz świata. Jednakże jako propozycja epistemologiczna nie jest w stanie -- czy to w wersji Chwistka, czy Williama Jamesa -- poradzić sobie z wciąż tą samą trudnością: jak dowieść wyższości pewnej teorii bytu w tym samym języku, w którym została wypowiedziana? Mieliżbyśmy utrzymywać, że twierdzenie „wszystko jest konieczne” jest równie prawomocne co twierdzenie „poza związkami logicznymi nic nie jest konieczne” i że doktryna, według której słowo „ja” nie ma odniesienia, jest nie mniej prawdziwa niż ta, wedle której cokolwiek ma odniesienie, jest względne w stosunku do „ja”?

\subsection{Podtytuł 2 stopnia}

\noindent Jeśli niezależne od siebie pokłady rzeczywistości wymagają dla swego opisu niezależnych języków, sugeruje to, że języki te są całkowicie nieprzekładalne; skoro tak, to w istocie można sądzić, że różne wizje świata współistnieć mogą w doskonałej wzajemnej obojętności: nie mogą być ze sobą konfrontowane ani między sobą sprzeczne. Ale twierdzenie, że nie mogą być konfrontowane, jest wyrażone w języku innym, wyższego stopnia, nie nadającym się do celów metafizycznych. I tu powraca ten sam kłopot: albo ograniczamy się do tego wyższego języka i wtedy werdykty nasze nie mają znaczenia dla rzeczywistych problemów, z których filozofia żyje, albo przyjmujemy pewną metafizyczną perspektywę i głosimy, że perspektywy tej, jako zamkniętej, nie da się zharmonizować z żadną inną ani też innej przeciwstawić -- i wtedy też, w wyniku tej samointerpretacji, perspektywa nasza jest bez znaczenia dla rzeczywistych problemów, z których filozofia żyje.

\section{Podtytuł 1 stopnia}

\noindent Teoria wielu -- jakkolwiek wyodrębnianych -- rzeczywistości, jeśli nawet w tym przypadku stworzona dla metafizycznej interpretacji malarstwa, proponuje przekonujący i kuszący obraz świata. Jednakże jako propozycja epistemologiczna nie jest w stanie -- czy to w wersji Chwistka, czy Williama Jamesa -- poradzić sobie z wciąż tą samą trudnością: jak dowieść wyższości pewnej teorii bytu w tym samym języku, w którym została wypowiedziana? Mieliżbyśmy utrzymywać, że twierdzenie „wszystko jest konieczne” jest równie prawomocne co twierdzenie „poza związkami logicznymi nic nie jest konieczne” i że doktryna, według której słowo „ja” nie ma odniesienia, jest nie mniej prawdziwa niż ta, wedle której cokolwiek ma odniesienie, jest względne w stosunku do „ja”?

\subsection{Podtytuł 2 stopnia}

\noindent Jeśli niezależne od siebie pokłady rzeczywistości wymagają dla swego opisu niezależnych języków, sugeruje to, że języki te są całkowicie nieprzekładalne; skoro tak, to w istocie można sądzić, że różne wizje świata współistnieć mogą w doskonałej wzajemnej obojętności: nie mogą być ze sobą konfrontowane ani między sobą sprzeczne. Ale twierdzenie, że nie mogą być konfrontowane, jest wyrażone w języku innym, wyższego stopnia, nie nadającym się do celów metafizycznych. I tu powraca ten sam kłopot: albo ograniczamy się do tego wyższego języka i wtedy werdykty nasze nie mają znaczenia dla rzeczywistych problemów, z których filozofia żyje, albo przyjmujemy pewną metafizyczną perspektywę i głosimy, że perspektywy tej, jako zamkniętej, nie da się zharmonizować z żadną inną ani też innej przeciwstawić -- i wtedy też, w wyniku tej samointerpretacji, perspektywa nasza jest bez znaczenia dla rzeczywistych problemów, z których filozofia żyje.

\begin{thebibliography}{00}{Imię Nazwisko}
{Tytuł rozdziału. Podtytuł rozdziału}

\makeatletter
    \clubpenalty10000
    \@clubpenalty \clubpenalty
    \widowpenalty10000
\makeatother


\bibitem{Abc}
A.~Abc, \textit{Abc}...

\bibitem{Xyz}
A.~Abc, \textit{Abc}...

\end{thebibliography}