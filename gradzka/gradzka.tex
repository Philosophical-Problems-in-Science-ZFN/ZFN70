\begin{editorialeng}{Ewelina Grądzka}
	{Report from the philosophical workshop organized by The~Lvov--Warsaw School Research Center and ~Kazimierz Twardowski Philosophical Society of Lviv}
	{Report from the philosophical workshop$\ldots$}
	{Report from the philosophical workshop organized by The Lvov--Warsaw School Research Center and ~Kazimierz Twardowski Philosophical Society of Lviv}
	%	{Copernicus Center for Interdisciplinary Studies}
	{Report from the Philosophical Workshop organized by The Lvov--Warsaw School Research Center and ~Kazimierz Twardowski Philosophical Society of Lviv}
	%	{Abstrakt lorem ipsum}
	%	{słowo, słowo.}



Between 11--14 February 2021 the first international Philosophical Workshop organized by The Lvov--Warsaw School Research Center (LWSRC) and Kazimierz Twardowski Philosophical Society of Lviv (KTPSL) took place in the on--line version due to the ongoing COVID--19 pandemic. The working languages of the event were Polish, Ukrainian and English. The coordinators’ goal was to refer to the tradition of seminar of Kazimierz Twardowski, who was not only a~distinguished philosopher but also a~great educator, to stimulate interest and support for the young generation of researchers into the heritage of the Lvov--Warsaw School (LWS). It is claimed that due to Twardowski’s unprecedented didactical engagement he managed to upbring dozens of Professors like Kazimierz Ajdukiewicz, Stefan Baley, Leopold Blaustein, Tadeusz Czeżowski, Izydora Dąmbska, Tadeusz Kotarbiński, Stanisław Leśniewski, Jan Łukasiewicz, Władysław Witwicki.

The Workshop was welcomed in Polish language by Professor Bogdan Dziobkowski, vice--dean of the Philosophy Department at the University of Warsaw. He reminded of another similar event, the Lviv--Warsaw Seminar of Philosophy of Science that has been organized by Professor J.~Jadacki and Professor I.~Vakarchuk since June 2000 as a part of the East European School in the Humanities that has taken place in turns---once in Lviv and once in Warsaw. He emphasized also Twardowski’s effort to create comfortable scientific conditions for his students that ranged from organization of a~well--equipped library with a~reading area up to a~development of a~work environment based on hard work, training in clarity of thinking and expression, friendship as part of e.g. Philosophical Circle.

Stepan Ivanyk, the president of the KTPSL, while welcoming the participants in Ukrainian, reminded that the anniversary of Twardowski’s death (on 11th February 1938) was a~stimulus to organize the event. It was Twardowski’s students desire to commemorate their master by annual organizing meetings. The first meeting took place in February 1939. Unfortunately, the outbreak of the Second World War prevented any further events. Therefore, Mr Ivanyk asserted that it was KTPSL’s pleasure to join the initiative of the LWSRC to restore that idea. He also pointed that the LWS was a~school of critical thinking which is of great value nowadays, in times of information chaos and politics of post--truth. Secondly, a~lot of members of the School specialized first in other disciplines like biology, physics, mathematics, linguistics etc. and then philosophy. It was in line with Twardowski’s vision that to achieve proper understanding of philosophy, a~degree in another discipline is indispensable. Thirdly, the School had a~cosmopolitical character. Although Twardowski was a~great Polish patriot it did not dominate his vision of philosophy as universal, supranational. Among his students there were distinguished Polish Professors but also Jews like Leopold Blaustein, Salomon Igel or Jacob Avigdor and Ukrainians like Stefan Baley or Hawryił Kostelnik.

The event was opened in Polish language by Professor Anna Brożek, the Director of the LWSRC, who emphasized that around eighty people from various academic backgrounds and centers applied to participate. She introduced the LWS by presenting its most valuable aspects: the School, the method, analysis, logic and reason. The school followed antique tradition of master – apprentice. The master should be the epistemic authority for the student. Twardowski was considered even a~siege who managed to influence didactical style of his students. Moreover, there was an equally strong bonding between the students. They actively cooperated with each other, which included many creative disagreements. The method acknowledged clarity of thinking and expression, proper justification of ideas, decent exchange of thoughts. Unlike in other philosophical schools, Twardowski’s students did not follow his philosophical ideas. They were rather taught to think for themselves.

The LWS belongs to a~larger current – analytical philosophy. They rejected vague systems and non-scientific speculation. Instead, they investigated small problems that they were able to state clearly and language as it is an important tool of cognition. Something that distinguished the School among other analytical currents was its respect for philosophical tradition and some level of epistemic optimism. They believed that philosophical problems due to meticulous analysis are solvable or at least better stated. Kotarbiński called it \textit{small philosophy}. The LSW members considered logic to be divided into formal logic, logical semiotics and methodology of science. It was considered organon to help fulfill the postulate of clarity and proper justification. Twardowski, although influenced by psychologism which accepted descriptive psychology as foundation of philosophy, acknowledged also logic as second foundation of philosophy. His students, J. Łukasiewicz and S. Leśniewski, took it further and developed mathematical logic which stimulated the formation of the Warsaw School of Logic.

Professor Anna Brożek highlighted that their accomplishments had become ‘export products’ already during the interwar period. Mathematical logic was also used to solve philosophical problems. Descriptive psychology problems were not neglected but rather gradually reformulated into semantic issues. Moreover, logic was considered an indispensable part of general education and called \textit{the morality of thinking}. A~person with proper logic education is able to overcome one’s own biases, mistakes, prejudice, thinks independently and is resistant to manipulation. Finally, the emphasis on clarity refers to the postulate of reason which rejects any irrational aspects of thinking wherever possible. The anti-irrational program was against any dogmas or pseudoscientific mystification. They trusted that reasonable thinking naturally prolongs into rational decisions and effective actions. Professor Brożek stated that all the points mentioned above distinguish good philosophy in general and that anti-irrational program should rule in any aspect of life. She also reminded the background of the event and added that for the last thirty years there have been celebrations organized in Lviv. Firstly, those were lectures after K. Twardowski. For the last few years, Professor Ihor Karivets have held Round Table meetings related to the heritage of K. Twardowski and his School. This year, Warsaw joined the event. Professor Brożek affirmed that what really mattered for Twardowski was content of philosophy not the nationality, gender or the worldview of the researcher. Therefore, among his students there was an unprecedented number (even on the world level) of women but also a~large number of Jews and Ukrainians as well as priests or atheists. Professor Brożek indicated her joy that Warsaw and Lviv again reunite in the search for truth and that good philosophy does not have nationality or gender and does not carry any dominant worldview.

The event was divided into four types of meetings. There was a~Roundtable Session of nine Ukrainian scholars; twelve lectures given by distinguished researchers into the LWS heritage; a~seminar with five presentations given by young researchers and two workshops for undergraduate students.

The part held in Ukrainian language was divided into two sessions. In the beginning the Roundtable entitled ``Kasimir Twardowski and his Ukrainian pupils'' was organized by Professor Ihor Karivets in the morning on 11\textsuperscript{th} February 2021. The speakers presented the following papers: Svitlana Povtoreva (Lviv Polytechnic National University) ``Stepan Baley’s psychoanalytic researches into artistic creativity''; Oksana Chyrsinova (Lviv Polytechnic National University) ``Stepan Baley on the use of technical tools in teaching experimental psychology''; Olga Honcharenko (The National Academy of the State Border Guard of Ukraine named after Ivan Khmelnytskyi) ``On the influence of Kasimir Twardowski’s philosophy on Yakym Yarema’s views concerning the ``unconscious consciousness''; Yaryna Yurynets (National University of ``Kyiv--Mohyla Academy'') ``Lviv Philosophical School of Kasimir Twardowski and Volodymyr Yurynets: institutional connection or theoretical and methodological affiliation''; Ihor Karivets (Polytechnic National University) ``Hilarion Sventsitsky on periodization of Ruthenian philosophy and spiritual--physical monism''; Natalia Fanenshtel (Khmelnytsky Humanitarian and Pedagogical Academy) ``Kasimir Twardowski’s linguistic and didactic views''; Leonid Mazur (Precarpathian Institute after M. Hrushevsky of Interregional Academy of Personnel Management) ``Lviv-Warsaw philosophical school: semantical analysis of science’s antinomies and the essence of moral value'' and Andrii Synytsia (Ivan Franko Lviv National University) ``On the essence of historical--philosophical concepts (on the example of Lviv-Warsaw School\footnote{Due to historical change of the name of the city from Lwów (until the Second World War, name is commonly translated as ‘Lvov’) to \foreignlanguage{russian}{Львів} (after the Second World War, translated as Lviv) there is a~difference in translation of the name of the School into English used by various scholars. })''. Whereas on the last day of the event two more speakers presented in Ukrainian language: again Olga Honczarenko, ``The Lviv-Warsaw School and the idea of the university'' and Stepan Ivanyk, ``The Lvov-Warsaw School in the context of the development of Ukrainian philosophy''.

In the evening the second session took place. Maria van der Schaar, a~lecturer at Leiden University, who works on the history and philosophy of logic and history of analytic philosophy (especially Brentano’s school) presented a~speech entitled ``Philosophy as Critical Analysis; Twardowski’s Criticism of Russell’s Theory of Judgement''. Her speech was divided into four parts: an introduction of Twardowski’s method; Twardowski on judgement (1912); Russell on judgement (1912) and Twardowski’s criticism of Russell’s account of judgement and truth. Professor Schaar, who is the author of \textit{Kazimierz Twardowski: A~Grammar for Philosophy}
%\label{ref:RNDi7TwfDLLfn}(\textit{cf.} Schaar, 2016)
\parencite*[cf.][]{schaar_kazimierz_2016} %
 wondered what the relation between action and product is and concluded that it is an internal one and not a~relation of cause and effect. The example given was ``John’s death is the product, not the effect of him dying''. She also referred to the solution of the problem of psychologism and presented that Twardowski distinguished between judgement in a~psychological sense (\textit{ein Urteil fällen}) and judgement in a~logical sense (\textit{das gefällte Urteil}). The distinction is a~logical and not ontological. Next, the speaker presented Russell’s Multiple Relation Theory of Judgement (MRTJ). It states that the act of judging involves a~mind, the \textit{subject}, and the \textit{objects} about which we judge and which we bring into a~unity in a~certain \textit{order} as a~consequence of our judgement. The example given was: ``Othello believes that Desdemona loves Cassio''. The order here is J~(o,d,l,c) and it is true if a~belief corresponds to a~certain complex in the world and is false when it does not. In his lecture series published in 1925 
%\label{ref:RNDNqPyg3NfBq}(\textit{cf.} 1999b)
 Twardowski \parencite*[cf.][]{brandl_theory_1999} %
 criticized Russell’s theory of truth which means he also criticized MRTJ. ``The definition of truth presupposes a~particular perspective on the essence of judgement.'' 
%\label{ref:RND4e9zfwEXTJ}(Twardowski, 1999b)
\parencite[][]{brandl_theory_1999} %
 For the Polish philosopher Russell’s MRTJ is just another variant of the traditional idea of judgement and the only difference is that judging connects objects but not ideas and there is no act of denial. Schaar pointed to three other arguments of Twardowski. The first one is that judging is not a~relation. It is mind’s judging that brings about a~relation and it is not itself a~relation. Here Schaar offered an evaluation of Twardowski’s first point against Russell based on the example of J~(o,d,l,c). Secondly, the founder of the LWS was dissatisfied with too many ambiguities as the terms like \textit{belief, statement, judging} and \textit{judgement} were used interchangeably. Thirdly, the act of judging is not an act of unifying but rather of affirming or denying the existence of \textit{A}~(that can be simple or complex). The speaker emphasized the relevance of Twardowski’s accounts as we can observe the revival of Russell’s MRTJ like in case of Peter Hanks or others. They focus on the act of judging as the only truth--bearer and do not want to acknowledge objective propositions. Professor Schaar mentioned also Friederile Moltmann due to her proposition to use the distinction between act and product to account for a~less psychological bearer of truth and falsity. Schaar had already presented a~speech \textit{Twardowski on Judgement and Inference} with evaluation on Moltmand’s and Twardowski’s position in October 2019 in Warsaw. In the Q\&A session Professor Bernard Linsky expressed his admiration for Twardowski’s argumentation and interest in reading his text. Linsky also wondered if Russell was aware of that critique. Schaar denied as Twardowski’s lecture was translated into English only in the 1990’s. Next, Professor Woleński raised a~question that generated a~vivid discussion related to Brentano’s critique of the correspondence theory of truth.

The second speaker of the evening was Dr Guillaume Frechette from Université de Genève who works mainly on 19\textsuperscript{th} and early 20\textsuperscript{th} century Austrian and German philosophy. He presented a~talk, that was aimed to suit those who do not have expert knowledge in the field, entitled \textit{Psychology and Philosophy at the turn of the century: Twardowski and the School of} \textit{Brentano}. It was divided into four parts: From Kant to Fechner; Quo vadis psychology?; Brentano on philosophy and psychology as a~science; Varieties of descriptive psychology: the case of Twardowski. In the first part Kant’s rejection of empirical psychology as scientific endeavor was presented and Herbart and Fechner opposition to Kant. Finally, around 1900 psychology slowly got independent from philosophy and a~discussion on place of psychology at university started. There were also two main radical options of psychologism (Mach, Jerusalem) and anti--psychologism (Husserl, Frege). However, the most important for the talk, and often neglected in discussions, was the more nuanced attitude of Brentano and his school that treated psychology as a~philosophical discipline. And as philosophy is science for Brentano, therefore psychology is science. Philosophy is speculatively exact like mathematics and psychology’s exactness does not come from its use of quantitative measurements (like Wundt wanted) but from that speculative exactness. In his unfinished \textit{Psychology from an Empirical Standpoint} Brentano stated that the field of investigations are the contents of consciousness (objects of sensation, original association, superposed presentations, presentations of inner consciousness). The goal is to discover the mental laws (ontology of the mental). Later, Dr Frechette claimed that although the famous distinction between action, content and product is attributed to Twardowski (even Meinong stated that), it had been already present in Brentano’s works like in his lectures on logic. However, more interesting than paternity of the idea, is the idea itself. Although the example of a~\textit{name} is used it should not be understood that the distinction is purely semantic (like in Frege). Dr Frechette advocated that it is simply one good way to illustrate the distinction and claimed that it is what Twardowski wanted to say in his \textit{Action and Product}.

On 12\textsuperscript{th} February 2021 the first session was started by a~lecture of Professor Jan Woleński, philosopher, lawyer, emeritus Professor of Jagiellonian University, entitled ``The Lvov--Warsaw School as an example of a~historical catching up by the Polish philosophical community''. Professor Woleński’s intention was to present the LWS engagement in collaboration with international community of philosophers. Although Polish philosophical tradition is around six hundred fifty years old, Poland has never belonged to philosophical superpowers. Jagiellonian University (JU), the first in Poland, was established only in 1364. The most significant accomplishment was that of Paweł Włodkowic, a~co--founder of the international law. In the 18\textsuperscript{th} century \textit{Komisja Edukacji Narodowej} (The Commission of National Education) related to Kołłątaj’s antischolastic educational reforms was intended to relate to modern Western philosophy. It failed due to the partitions of Poland at the end of 18\textsuperscript{th} century and partition powers were not interested in the development of education. Therefore Polish philosophy was developed outside academia. Great poets took the job of philosophers like in late romantic period but except of patriotic thought it was not very original. Most of the ideas we observed at that time were eclectic. Finally, Professor Woleński reached Kazimierz Twardowski’s arrival to Lvov. The situation of philosophy in Lvov was poor until Twardowski’s didactical and organizational efforts that led to creation of the Lvov--Warsaw School (and preparation of at least thirty Professors which is a~world record). Without that didactical -- organizational success there would be no international success. Twardowski was against ``Polish national philosophy'' (unlike Henryk Struve) or Warsaw positivism (due to eclecticism and its devotion to French and British philosophy) and opted rather for doing ``good philosophy in Poland'' (relatively independent from other countries but at the same time in a~constant contact with international community). For example, history textbooks should rather be written by national historians and not translated. Professor Woleński gave an example of Tatarkiewicz’s textbook to philosophy which is fair with all the currents (countries) in philosophy and at the same time includes Polish philosophers. The arrival of the independence of Poland stimulated works to build developed academic environment. Twardowski’s program was compared to Janiszewski’s ideas of development of mathematics. Janiszewski likewise claimed that Polish scientists have to come up with something original to be internationally recognized. Therefore, on the one hand Russell’s books were read while on the other hand Ajdukiewicz developed his own conventionalism. There were at that time some complains from Roman Ingarden that Polish philosophy was delayed as phenomenology had been neglected. The LWS maintained contact with the Vienna Circle. For example, when their Manifest \textit{Wissenschaftliche Weltauffassung: Der Wiener Kreis} was published it was instantly commented in \textit{Ruch Filozoficzny} (Philosophical Movement). Rudolf Carnap and Karl Menger (who later praised the LWS in his diaries) visited Poland and later Tarski went to Vienna, where he impressed Gödel and Carnap. There was a~conference in Vienna in 1934, where a~lot of LWS philosophers were invited and presented important investigations. That started an intensive cooperation which continued through the 1930’s. There was even a~plan to organize an international philosophical congress in 1940 but due to eruption of the Second World War that plan failed.

Next, a~workshop was held by Dr. Marcin Będkowski entitled ``Text analysis based on the example of Twardowski’s work \textit{O~jasnym i~niejasnym stylu filozoficznym} [\textit{On Clear and Unclear Philosophical Style}].''
%\label{ref:RNDNDvuP1mzCr}(\textit{cf.} Twardowski, 1927, eng. transl. 1999a)
\parencites[cf.][]{twardowski_o_1927}[eng. transl.][]{brandl_clear_1999} %
 He works at the Institute of Polish Language and is interested in pragmatic logic, linguistic pragmatics, methodology of humanities, history of the LWS. His goal was to show general idea of the analysis: ``reconstruct the reasoning process'', ``distinguishing significant concepts and their understanding'', ``pointing to divisions and typologies''. He used the text by Brożek, Jadacki \textit{Analiza ``analizy}'' [Analysis of ``analysis''] 
%\label{ref:RNDb6gqa9mzFx}(2006)
\parencite*[][]{brozek_analiza_2006} %
 and Szymanek \textit{O~logicznej analizie tekstu} [On the logical analysis of the text] 
%\label{ref:RNDlnSLvjsare}(2010).
\parencite*[][]{szymanek_o_2010}. %
 Dr. Będkowski admitted that although most of the participants had surely taken course in logic, the challenge is to apply the skills to text analysis. As an exercise a~piece from the text by I. Dąmbska \textit{Lęk przed śmiercią} [Fear of death] was used. He reminded also Kotarbiński’s distinction between method of creative interpretation vs the global imitative method 
%\label{ref:RND10GHKcRUBb}(\textit{cf.} Kotarbiński, 1965).
\parencite[cf.][]{kotarbinski_praxiology_1965}. %
 The later method proposes to follow the way the author thinks and the goal is to imitate that thinking. However, it is not putting the problem any further. On the other hand, the first method motivates to develop the problem further and its other possible solutions. After reading Dąmbska’s text, Dr. Będkowski investigated what the thesis of Plato was. Later, the structure of justification in that text was analyzed with the use of a~modern tool of rationaleonline.com. It was claimed to be a~very effective and clear, like a~map, that shows the relations between premises and thesis. The speaker also presented a~more developed diagram based on the text of Ajdukiewicz. The final step was to refer all that knowledge to the main text, that of Twardowski. The participants were encouraged to look for the reference conjunctions, next to state what the main thesis is, what the supportive theses are and if the main thesis requires some additional assumptions. At the end, Dr. Będkowski presented a~very developed diagram for Twardowski’s text. Due to engagement of the participants in the discussion, the session significantly exceed the assumed time.

In the afternoon a~session dedicated to young researchers’ investigations took place. It was the main reason the whole event was organized. The idea referred to Twardowski’s broad didactical work that focused on direct contact with students during the meetings of a~Philosophical Circle, proseminar or seminar. That attitude is considered to be a~key to his success and the fundament of the LWS. The young researchers presented as followed: Ewelina Grądzka (Ph.D. candidate, supervisor: Professor Paweł Polak, Pontifical University of John Paul II) ``Are Kazimierz Twardowski’s ideas on philosophy teaching valuable recommendations for the contemporary philosophy for/with children movement?''; Karolina Kantorowicz (MA, student, supervisor: Professor Anna Brożek, Warsaw University) ``Determinism and criminal liability in the works of Kazimierz Twardowski and Leon Petrażycki''; Joanna Frydrych (student, supervisor: Professor Marcin Tkaczyk, John Paul II Catholic University of Lublin) ``Boolean algebra and Leśniewski’s mereology''; Mateusz Lisowski (student, supervisor: Professor Marcin Tkaczyk, John Paul II Catholic University of Lublin) ``Review of the axiomatization of Ł--modal logic by Jan Łukasiewicz''; Antoni Torzewski (student, supervisor: Professor Dariusz Łukasiewicz, Kazimierz Wielki University) ``Criticism of certain aspects of the metaphilosophical program of the Lvov--Warsaw School''.

The next session was started by Professor Jacek Jadacki, with another workshop on ``How to prepare good summaries?'' on 13\textsuperscript{th} February 2021. In the beginning Martin Heidegger’s text from \textit{Kant and the Problem of Metaphysics} and Jan Łukasiewicz’s text \textit{Two-valued logic} were presented as two examples of text improper for summary. In the case of the first work it is too vague and it is almost impossible to distinguish clear segments, although Professor intended to expose some thesis, argumentation, even definition and later agreed that he used a~translation that is already some form of interpretation (in the LWS it was required to read texts in their original language). Whereas Łukasiewicz’s work represents a~text that either gives a~summary in the beginning or it is impossible to take part of the theorem out. Finally, Professor Jadacki analyzed I. Dąmbska’s text \textit{O~pojęciu rozumienia} [\textit{On the notion of understanding}]
%\label{ref:RNDoRP5VPKeyi}(Dąmbska, 2016)
\parencite[][]{dambska_notion_2016} %
 first presented at the International Congress of Philosophy in Florence in 1958. He distinguished eleven segments in the text and offered his shorter and more detailed summary. The reason was to show that a~properly developed summary can be reduced to a~shorter version if necessary without any loss in understanding. Professor Jadacki emphasized that text comprehension is always some form of interpretation and it requires openness to discussion and revision. At the end, the organizers of the event announced a~competition for the best summary of the text by Maria Ossowska \textit{O~pojęciu godności} [On the notion of dignity], another LWS member.

Next, Professor Ryszard Kleszcz from University of Lodz presented ``Philosophy and a~worldview. In the context of the metaphilosophical proposal of the Lvov--Warsaw School''. He is interested in epistemology, methodology, analytical philosophy, Polish philosophy of the 20\textsuperscript{th} century and metaphilosophy. First, Professor specified understanding of the concept of ``metaphilosophy'' and ``worldview''. For ``metaphilosophy'' he proposed a~definition from \textit{The Cambridge Dictionary of Philosophy} by Paul Moser. Its relation to the problem of the worldview is interesting, especially in the context of the LWS. Next, a~``worldview'' was defined. It generally does not have such a~strong justification like science or philosophy and can always be questioned
%\label{ref:RNDU3g7M561UU}(\textit{cf.} Bocheński, 2016).
\parencite[cf.][]{bochenski_logika_2016}. %
 For the LWS, its synthetic character and interest in the purpose of the Universe discredited it from philosophy. Finally, Twardowski’s ideas were analyzed. In the beginning, he followed Brentano’s attitude to metaphysics as an integral part of philosophy. However, later he decided it is impossible to solve metaphysical problems with scientific tools and rejected them as part of the worldview 
%\label{ref:RND4msF9Zg8dc}(\textit{cf.} Twardowski, 1965).
\parencite[cf.][]{twardowski_przemowienie_1965}. %
 Twardowski distinguished three types of beliefs: rational (scientific), irrational (beyond rationality, not necessarily against science), unrational/non--rational (against science). Although Twardowski considered philosophy as part of scientific investigation that cannot justify any worldview over the other, he did not underestimate the problem of a~worldview. He understood its importance for practical life and its helpful, informative task of how to refer to the world, others and ourselves. However, even a~worldview should follow some rules: ``lack of internal contradictions, being in accordance with science, being comprehensible.'' Professor Kleszcz admitted such expectations can still be considered ambiguous. He also described the attitude of J. Łukasiewicz, T. Kotarbiński and T. Czeżowski to show that opinions in that subjected varied in the LWS. Łukasiewicz referred to the problem in two texts: \textit{O~nauce i~filozofii} [On science and philosophy] 
%\label{ref:RNDENs6je3j0Z}(1915)
\parencite*[][]{lukasiewicz_o_1915} %
 and \textit{O~metodę w~filozofii} [For a~method in philosophy] 
%\label{ref:RNDKzKKZWsQxX}(1927).
\parencite*[][]{lukasiewicz_o_1927}. %
 In the first text he distinguished two types of philosophy – one as a~set of disciplines and the other as a~general worldview. In the second text requirements for the first type of philosophy were established offering a~program of radical axiomatization. Kotarbiński’s ideas evolved over time. In his text \textit{Filozof} [A philosopher] he admitted that although metaphysics is full of ambiguity this does not exclude it. Moreover, his reism may be treated as part of metaphysics. In his work he did not separate strictly those problems. Whereas Czeżowski was a~declared Brentanist and accepted metaphysics as a~fully-fledged part of philosophy as science cannot grasp the whole reality. But there is also ``faith'' understood as a~worldview and it has another irreplaceable role. To sum up, Professor Kleszcz postulated that although philosophy and a~worldview are obviously connected, their different methodological status should be considered on metaphilosophical level and therefore LWS’s attitude is recommendable.

Professor Paweł Polak, Pontifical University of John Paul II, whose interest focuses on philosophical problems of computer science and the history of science and Polish philosophy of nature, presented his research entitled: ``A landscape of Lvov philosophy. Twardowski’s School and the others -- selected examples.'' The cases of philosophizing scientists like two physicists Marian Smoluchowski and Stanisław Loria, Maksymilian Tytus Huber (Professor of theory of mechanical engineering); engineers like Stanisław Szczepanowski and Wacław Wolski (oil traders and thinkers), Bronisław Biegeleisen (psychologist and engineer); and writers’ circles like Ostap Ortwin or Julian Edwin Zachariewicz were analyzed. Professor Polak observed that the LWS is a~fascinating subject for historians of philosophy as it is difficult to make generalizations about it, it causes problems with setting its frames and it raises a~question whether schools should be conceived as mechanisms or organisms. It is also important to investigate how the environment influenced the development of the LWS to better understand why some solutions were accepted over the others. As an example, the speaker had chosen a~controversy over Einstein’s relativity theory. The reconstruction of the discussion (in the form of a~presentation of successive papers) explained a~lot of interpretation difficulties and helped to comprehend the differences in the influences of various works. Like in the case of Zawirski who actively participated in the discussion or Ajdukiewicz who although inspired by the dispute, did not engage in it. In the 19\textsuperscript{th} century Polish philosophy developed mostly outside the universities and, interestingly, the polemics took place in the local press or on various social meetings. After a~short presentation of positions in the discussion, Professor Polak concluded Zachariewicz and Wolski played a~significant role in the beginning and development of the polemic as people like Zawirski, Huber, Loria tried to respond to the problems set by them
%\label{ref:RNDR0j8RhVqT0}(see also Polak, 2016).
\parencite[see also][]{polak_philosophy_2016}. %
 Secondly, it was acknowledged that in general the LWS did not engage with other philosophical currents existing in Lvov in the same time. However, there were members who sought outside relationships (like Ortwin or Zawirski). Additionally, Twardowski inspired many non--collaborators and encouraged their investigations. \textit{Polskie Towarzystwo Filozoficzne} [Polish Philosophical Society] also locally stimulated interest in philosophy. Finally, it was emphasized that the LWS was not particularly involved in the philosophy of science or nature. This is the reason why Lvov scientists as well as Zawirski (LWS) tried to establish cooperation with Cracow philosophers.\footnote{More about historical links of mentioned group to this periodical see 
%\label{ref:RNDlLWSTRZEdF}(Trombik, 2019).
\parencite[][]{trombik_origin_2019}.%
}

The afternoon session that was dedicated to logic was initiated by Professor Kordula Świętorzecka, Cardinal Stefan Wyszyński University, with a~talk on ``More seriously about possible worlds. A~little introduction to modal logics''. Professor Świętorzecka began with some general introduction of the issue and emphasized that the notion of possible worlds is present in pop culture (movies, computer games) and it draws attention of many (like S. Lem, IT specialists), not only philosophers. It was reminded that the works of the most influential figures of the LWS, especially J. Łukasiewicz’s and A. Tarski’s, were related to modal logic. Although they were not aware of the semantics of possible worlds. Next, some problems with big notions (actual, possible world) were presented and a~small solution and an example was proposed, all with logic as a~tool. At the end of the speech the Professor generalized her example to modal logic interpreted in semantics of possible worlds and reminded that in the beginning the semantics of possible worlds was not related to modal logic. The meaning of necessity was also considered – in alethic way or to know something (epistemic notion that allows to formulate logics). Considering that different people have different kinds of modalities of necessities, logical tools are used e.g. in thinking about structures of voting. Professor Świętorzecka concluded that although the idea of possible worlds seems ridiculous, the tools she presented can be used for a~more down-to-earth investigations like informatics or in predictions of voting.

The next speaker was Zuzana Rybařiková, Palacký University Olomouc, who presented a~paper entitled ``Łukasiewicz’s Concept of Anti – psychologism.'' Łukasiewicz discussed the issue firstly in an article \textit{O~stosunku logiki do psychologii} [On the relation of logic to psychology], later in his book \textit{Analiza i~konstrukcja pojęcia przyczyny} [Analysis and construction of the concept of reason]
%\label{ref:RNDOwczfD1EKS}(Łukasiewicz, 1906)
\parencite[][]{lukasiewicz_analiza_1906} %
 and finally in \textit{Logika i~psychologia} [Logic and psychology] as well as in correspondence with Twardowski 
%\label{ref:RNDkajjAuQcq0}(Łukasiewicz, 1998).
\parencite[][]{lukasiewicz_logika_1998}. %
 In one of the letters he proclaimed a~war against psychologism in philosophy (later a~war against determinism) that he continued his whole life, although his ideas evolved. The problem of psychologism was widely discussed at the beginning of the 20\textsuperscript{th} century but the term is not well defined and it was differently understood by various users. Łukasiewicz defined \textit{anti-psychologism} in \textit{Logika i~psychologia} and Dr. Rybařiková divided her presentation mostly in accordance with his proposal: 1) laws of logic are not grounded in laws of psychology; 2) logic is \textit{a~priori} science but psychology is empirical science; 3) logic as a~science does not concern human reasoning but truth and falsehood and 4) logical terms differ from psychological terms. In conclusion, Dr. Rybařiková emphasized Łukasiewicz rejected the idea that laws of logic are certain and consequently the differentiation between empirical and \textit{a~priori} sciences. Importantly, he also favored the use of terms like \textit{zdanie} (a sentence) rather than \textit{sąd} (judgement) which was oriented on clear separation of logic from psychology.

The last session of the day was Sébastien Richard, Université Libre de Bruxelles, who presented ``Kotarbiński’s Reism.'' In his \textit{Elementy} [The Elements]Kotarbiński, following Brentano’s ideas, claimed that only objects exist. His doctrine, called reism, consisted of a~semantic part as well as an ontological one. From the ontological perspective only things are accepted whereas from the semantic perspective singular, general and empty terms. The three fundamentals of reism are: ``every object is a~thing; no object is a~property, a~relation, an event; the expressions like ‘property’, ‘relation’, ‘event’ are apparent terms''. To avoid the problem of events Kotarbiński used the strategy of paraphrase (common in the LWS). Importantly, the turning point was K. Ajdukiewicz review with strong criticism of reism which forced Kotarbiński to reformulate some of his ideas. Reism is a~truism in a~reistic language and it is either meaningless or false. Although Kotarbiński accepted Ajdukiewicz’s arguments and instead of being only a~‘substantial ontological and semantic doctrine’ it became a~methodological program. Lejewski opposed such a~development. For him such a~program is not justified on semantic level unless it is justified on an ontological level. He offered an argument against Ajdukiewicz’s claim that ``reism is a~truism in a~reistic language.'' At the end, Dr. Richard referred to Leśniewski’s Ontology (a logical system) and claimed, after Woleński, that it is neutral as it does not talk about objects but about the way we talk about objects.

On the last day of the event Professor Marek Rembierz, University of Silesia, whose interests include philosophy, history of philosophy, pedagogy, theory of upbringing, and the LWS traditions, presented ``Scientific critique, knowledge--forming discussion and logical culture as an affirmation of cognitive values (with reference to Professor Tadeusz Czeżowski’s lectures on general methodology of sciences in the Department of Pedagogy of Nicolaus Copernicus University in the academic year 1953/1954).'' The Professor noticed that although Czeżowski’s lectures were related to philosophy they were held at the Department of Pedagogy as there was no Department of Philosophy due to communist’s restrictions of that time in Poland. He also emphasized that it is an opportunity to attract attention to the pedagogical branch of the LWS and its achievements (the Director of the Department was K. Sośnicki who wrote \textit{Zarys logiki} [Outline of logic] and \textit{Zarys dydaktyki} [Outline of didactics], like Twardowski). It was symptomatic that philosophical analysis was applied to pedagogical inquiry. Professor Rembierz began with introduction of Czeżowski, who is considered an underestimated member of the LWS, yet the closest student of Twardowski. It referred to coherence between his personality and his didactical practice (Socratic style, according to I. Dąmbska), and on the other hand to his philosophical ideas, the closest to Brentano’s among the LWS. As discussion was one of the key fundamentals of the LWS method it is not surprising that one of the lectures was \textit{O~dyskusji i~dyskutowaniu} [On discussion and discussing]
%\label{ref:RNDvO97sb3cle}(Czeżowski, 1969b).
\parencite[][]{czezowski_o_1969}. %
 The subject of the second lecture was entitled \textit{Kultura logiczna} [Logic culture] 
%\label{ref:RND34NtttzANN}(Czeżowski, 1969a)
\parencite[][]{czezowski_kultura_1969} %
 and referred to its importance for the LWS. Interestingly, Czeżowski’s texts are unbelievably concise and clear. He introduced there a~term \textit{logical conscience} which goes beyond treating logic simply as a~tool or skill and forms part of the logical culture. \textit{Logical conscience} transcendences dogmatism and our lack of objectivity, stimulates criticism towards ourselves and the others. Therefore, it is important part of upbringing. Additionally, publishing a~textbook to logic is considered a~cultural activity. The second text on discussion refers to Twardowski’s ideas and considers discussion a~knowledge--making method. Professor Rembierz reminded also that Pope John Paul II in his book \textit{Memory and identity: personal reflections} 
%\label{ref:RND16WYASHxP0}(John Paul II, 2005)
\parencite[][]{john_paul_ii_memory_2005} %
 mentioned T. Kotarbiński, M. Ossowska and T. Czeżowski as those who managed to maintain criticism against dialectical materialism.

This conference was an important event for LWS researchers and I~hope to continue it soon.

\section*{Bibliography}%
		\printbibliography[heading=none]\nopagebreak[4]

\end{editorialeng}