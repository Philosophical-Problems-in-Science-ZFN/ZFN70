\begin{artengenv}{Wojciech Grygiel}
	{Cognitive aspects of the philosophical and theological coherence of the concept of a~miracle within the contemporary scientific world view}
	{Cognitive aspects of the philosophical and theological coherence\ldots}
	{Cognitive aspects of the philosophical and theological coherence\\of the concept of a~miracle within the contemporary scientific world\\
	view}
	{Pontifical University of John Paul II in Krakow}
	{The purpose of the article is to investigate the philosophical and theological validity and coherence of the classical concept of a~miracle within the contemporary scientific world view. The main tool in this process will be the cognitive standard model of the formation of religious beliefs operative in the cognitive science of religion. The application of %
	this model shows why an intentional agent is assigned as responsible for the occurrence of events with no visible cause such as a~miracle: miracles are events that violate the intuitively expected behaviors observable in the physical reality. It will become evident that much of the conceptual content of the classical understanding of miracles can be retained despite of the ontological and epistemological challenges of the contemporary science. In particular, this concerns the semantic view of miracles in which a~miracle does not occur as an objective Divine intervention but qualifies as religious interpretation of the natural course of events always in reference to a~cultural and personal context that is unique to those who directly experience these events either as direct recipients or as observers.}
	{miracles, divine action, panentheism, cognitive science, intuition.}
	
	



\section*{Introduction}
\lettrine[loversize=0.13,lines=2,lraise=-0.01,nindent=0em,findent=0.2pt]%
{S}{}ubjecting miracles to a~scientific study may sound like a~violation of the main principle of science, that is, the principle of methodological naturalism. The principle stipulates that science should rely on natural explanations only: nature should be explained by nature
%\label{ref:RNDt4AiRSjYlg}(e.g. Plantinga, 2001).
\parencite[e.g.][]{plantinga_methodological_2001}. %
 Many events commonly considered as miraculous manifest themselves in the physical realm suggesting that natural causes must be at least partially responsible for their occurrence. This is indeed the case when a~purely natural phenomenon is qualified as miraculous without reference to any supernatural agency. It turns out that the common sense classification of a~given phenomenon as miraculous very quickly raises serious concerns as to whether it is something extraordinary indeed or it is just the perception of its observer that prompts him or her to designate it as miraculous. In the English language the term \textit{miracle} clearly stems from the Latin \textit{mirari} which means to wonder. We wonder at things both natural and supernatural: we wonder at the discoveries of science and we wonder when we experience a~sudden healing from a~deadly cancer. In the first case we are astonished at what science can do and in the second case we are astonished at what science (at least for now) cannot do and we rush to explain it as the workings of the supernatural forces.

Miracles play a~decisive role in the Christian fundamental theology for they serve to confirm the divinity of Jesus Christ
%\label{ref:RNDWeJfFVcw2K}(Rusecki, 2006, pp.330–359).
\parencite[][pp.330–359]{rusecki_traktat_2006}. %
 Only God can heal the sick and raise from the dead. Miracles are also taken into account in the processes of beatification and canonization of confessors, that is, individuals that gave witness to the faith by the way they lived. A~miracle must occur through the intercession of a~candidate whereby God is believed to grant a~sign that he or she is enjoying the glory of heaven by performing a~miracle for which he or she interceded 
%\label{ref:RNDR85eQQSTKN}(John Paul II, 1983).
\parencite[][]{john_paul_ii_apostolic_1983}. %
 At this point a~fundamental question arises: if the progress of science reveals that certain diseases can be cured by purely natural means, does this invalidate the approved beatifications and canonizations? Worse yet, does this undermine the divinity of Christ as related by the gospels? The credibility of miracles has been challenged on the scientific grounds by in the 18\textsuperscript{th} century by David Hume who claimed the impossibility of their occurrence due to the inductive character of the laws of nature as exemplified by the Newtonian mechanics 
%\label{ref:RNDfxOdcfEQjo}(Hume, 2008, pp.79–95).
\parencite[][pp.79–95]{hume_enquiry_2008}.%
 The nature and the credibility of miracles remains a~topic of vivid discussions until the present day 
%\label{ref:RNDjupXhPwfba}(e.g. McGrew, 2019).
\parencite[e.g.][]{mcgrew_miracles_2019}.%


The inquiry carried out in this paper involves the application of the cognitive standard model of the formation of religious beliefs to establish the degree to what the classical philosophical and theological understanding of miracles retains its coherence and validity within the contemporary scientific world view. The task is conceptually complex for it hinges upon the understanding of one of the key issues in the contemporary natural theology, namely, that of the mechanism of the Divine action in the Universe
%\label{ref:RNDxt2AO6wyec}(e.g. Murphy, 1995; Słomka, 2021).
\parencites[e.g.][]{russell_divine_1995}[][]{slomka_gods_2021}. %
 The cognitive mechanisms that assign an intentional agent as responsible for the occurrence of events with no visible cause will allow to view miracles as events that violate the intuitively expected behaviors observable in the physical reality. This approach has been already implemented by De Cruz and De Smedt 
%\label{ref:RNDKLAJvbuymi}(De Cruz and De Smedt, 2015, pp.155–178)
\parencite[][pp.155–178]{de_cruz_natural_2015} %
 and the investigative efforts related in this paper take this approach as their point of departure. The task will be carried out in the following steps. Firstly, the classical philosophical and theological significance of miracles will be summarized with the particular emphasis on how hidden causes are invoked to explain these events. Secondly, the cognitive standard model of the formation of religious beliefs will be briefly introduced with the justification in what sense it pertains to the study of miraculous events. Thirdly, the specificity of the dynamics of the scientific growth will be surveyed in to see how the cognitive mechanisms may respond to the events that fall outside of the current knowledge of the nature’s trajectories. Fourthly and lastly, it will be shown how the standard model of the formation of religious beliefs secures many of the components of the classical concept of miracles in a~theological perspective that is consistent with the scientific world view. This coherence is achieved within the semantic view of miracles in which a~miracle does not occur as an objective Divine intervention but qualifies as religious interpretation of the natural course of events always in reference to a~cultural and personal context that is unique to those who directly experience these events either as direct recipients or as observers.

\section*{Miracles classically}
Miraculous events have been reported to occur since times immemorial both in religious and non-religious contexts. The attempts to unveil their nature have been undertaken by many philosophers, theologians and scientists in the entire history of the human intellectual endeavors
%\label{ref:RNDGgRl28nlOY}(e.g. Basinger, 2018).
\parencite[e.g.][]{basinger_miracles_2018}. %
 The contemporary understanding of what event qualifies as miraculous draws from two main classical sources: (1)~the religious thinking of medieval Christian philosophers such as St. Augustine and St. Thomas Aquinas and (2) the modern approach that rests largely on the critical works of David Hume. The major difference in these two sources stems from a~different concept of the fundamental ontology of nature that they assume. While the medieval view relied the ontology of things as individual substances, following the onset of the scientific method in the 17\textsuperscript{th} century the modern view shifted to see the fabric of the Universe as ordered by a~set of physical laws governing its dynamics. What remained intact of the medieval view, however, are the two distinct approaches to miraculous events with Augustine stressing the \textit{semantic} (subjective) and Aquinas the \textit{ontological} (objective) character of miracles.

As Rusecki points out, the works of Augustine impact the understanding of the nature and significance of miracles in all generations of thinkers to come
%\label{ref:RNDd2U3YsBIgH}(Rusecki, 2006, pp.35–36).
\parencite[][pp.35–36]{rusecki_traktat_2006}. %
 Augustine has not only dealt at length with ontological, epistemological and theological aspects of miracles but he has also developed his own view on how these aspects interplay in a~miraculous event. Rusecki argues that Augustine’s approach to miracles emphasizes their theological meaning, namely, that their function is primarily \textit{symbolic} to communicate the Divine plan of salvation of mankind and to strengthen its credibility. Consequently, miracles can be properly interpreted when considered in the religious context. A~detailed analysis of what is implied by the religious meaning of miracles is offered by Świeżyński 
%\label{ref:RNDxQidJQrVKK}(2012, pp.225–273).
\parencite*[][pp.225–273]{swiezynski_filozofia_2012}. %
 Augustine opines that since it is the will of God to maintain in existence all that He has created, God would never act contrary to nature. If miracles seem contrary to nature, it is due to the lack of knowledge of its laws 
%\label{ref:RNDXQPRm4qARv}(Augustine, 2003, \textit{The City of God}, XXI.8).
\parencite[][\textit{The City of God}, XXI.8]{augustine_city_2003}. %
 One of the most famous statements of Augustine concerning miracles, however, is voiced when he claims that the events considered as miraculous do not have to be any more exceptional than all other phenomena because all nature is a~great miracle in itself 
%\label{ref:RNDy8DYiAWkn3}(Augustine, 2003, \textit{The City of God}, X.12; X.16-18; XXI.7; XXI.8).
\parencite[][\textit{The City of God}, X.12; X.16-18; XXI.7; XXI.8]{augustine_city_2003}. %
 The reason some events are perceived as miraculous by the human mind is that they occur \textit{rarely} and as such they attract more attention and cause \textit{astonishment}. In a~strictly ontological sense all that occurs in the Universe according to the laws of nature deserves to be called miraculous because it precisely follows the order instituted by God and it is God Himself who acts in all that takes place in nature. According to Augustine miracles bear primarily epistemological (subjective) character for they arise on the grounds of the lack of the proper knowledge of nature and they acquire their due significance when they are interpreted in the domain of religion.

In contradistinction to the Augustinian conception of miracles were the emphasis falls on the experience of their recipient or observer, the approach adopted by Aquinas focuses on the objective properties of a~miraculous event. He starts out with the consideration of the specificity of the natural order to establish domains of possible phenomena reserved to the Divine action only. In this regard Aquinas clearly implements the Aristotelian methodology which commences from the things natural and based on a~chain of inferences arrives at the knowledge of the things pertaining to God. Aquinas’ understanding of the natural order relies the Aristotelian ontology of substances and represents the totality of the common sense knowledge of the nature of things available to a~scientist of the day before the onset of the contemporary scientific method. The precise meaning of Aquinas’ concept of the natural order has been succinctly summarized by Etienne Gilson
%\label{ref:RND9M2nUbHp3l}(1991, pp.376–377).
\parencite*[][pp.376–377]{gilson_spirit_1991}. %
 Gilson brings up the scholastic understanding of the Divine action in the world by means of the \textit{primary} and \textit{secondary causes}. Since the natural order as the network of the secondary causes is instituted by the act of the free will of God, He can always bypass the workings of the secondary causes and produce the desired effect directly by His power as the primary cause. In light of this, the following statement of Aquinas on miracles acquires its proper precision:

\myquote{
A~miracle properly so called is when something is done outside the order of nature. But it is not enough for a~miracle if something is done outside the order of any particular nature; for otherwise anyone would perform a~miracle by throwing a~stone upwards, as such a~thing is outside the order of the stone’s nature. So for a~miracle is required that it be against the order of the whole created nature. But God alone can do this, because, whatever an angel or any other creature does by its own power, is according to the order of created nature; and thus it is not a~miracle. Hence God alone can work miracles
%\label{ref:RNDAmS1NWBpzc}(Aquinas, 1981, \textit{Summa Theologiae} I.110.4).
\parencite[][\textit{Summa Theologiae} I.110.4]{aquinas_summa_1981}.%


}
The possibility of causal influences made directly by God outside the whole natural order in effecting a~miracle raises some difficulties thereby making the concept of a~miracle incoherent. In order to connect a~miracle with a~non-natural causation, it is necessary to know the boundaries of the natural order. Otherwise, the qualification of an event as a~miracle is ambiguous. Taking into account that the knowledge of the Universe in the 13\textsuperscript{th} century was confined to what was directly observable with a~naked eye and that the Universe in itself was regarded as a~static entity, it seems rational to assume that Aquinas could regard the Universe thus conceived as all that exist in the domain of nature. Moreover, the metaphysical principles derived by Aquinas as he modified those proposed by Aristotle provided the exhaustive explanation of the structure of the Universe and its relation to God whereby all that exists qualifies as the totality of the contingent order of being. Any operation that bypasses the natural order must have God alone as its cause.

It is evident from the above that what Aquinas calls a~miracle is a~sensually detectable event that lies outside the causal capacity of nature and its natural cause is unknown. In addition to this, Aquinas distinguished three degrees of a~miracle depending how far it is removed from the capacity of nature. They may totally exceed the power of nature but they may also engage the natural laws in a~manner that is inaccessible to the natural powers. These degrees are: \textit{supra naturam}, \textit{contra naturam} and \textit{praeter naturam}
%\label{ref:RNDJ5MZ0E98lK}(Aquinas, 1952, \textit{De Potentia Dei}, 6.2.3).
\parencite[][]{aquinas_power_1952}. %
 What seems most apparent from this account is that Aquinas admits of the objectivity of a~miracle, that is, it involves the direct activity of agents who are capable of exceeding the powers of nature and producing effects that could never occur naturally. It is not surprising that Aquinas points to God as the primary cause of miracles for God remains invisible in whatever He does.

In his \textit{Summa Contra Gentiles}, however, Aquinas introduces yet another qualification of a~miraculous event bearing a~more subjective character and referring directly to the etymology of the term \textit{miracle}, namely that of admiration and astonishment. Aquinas writes:

\myquote{
Things that are at times divinely accomplished, apart from the generally established order in things, are customarily called miracles; for we admire with some astonishment a~certain event when we observe the effect but do not know its cause. And since one and the same cause is at times known to some people and unknown to others, the result is that of several who see an effect at the same time, some are moved to admiring astonishment, while others are not. For instance, the astronomer is not astonished when he sees an eclipse of the sun, for he knows its cause, but the person who is ignorant of this science must be amazed, for he ignores the cause. And so, a~certain event is wondrous to one person, but not so to another. So, a~thing that has a~completely hidden cause is wondrous in an unqualified way, and this the name, miracle, suggests; namely, what is of itself filled with admirable wonder, not simply in relation to one person or another. Now, absolutely speaking, the cause hidden from every man is God
%\label{ref:RNDs2oHP9DjvD}(Aquinas, 1975, \textit{Summa Contra Gentiles}, III.101).
\parencite[][\textit{Summa Contra Gentiles}, III.101]{aquinas_summa_1975}.%


}
In this passage Aquinas supplements his treatment of miracles reported above by adverting not to the objective properties of such events but to the subjective response of the recipient or the observer. Unlike Augustine who ties the astonishment with the rarity of a~given event, Aquinas maintains that the astonishment takes place when the cause of event is unknown. An event that qualifies as miraculous only if this astonishment cannot be overcome by any future growth of the knowledge of the workings of the Universe. Aquinas concludes that these events have hidden causes in an absolute sense and only God can be such a~cause. Similarly to what has been addressed above, the proper discrimination of what may lead to a~surprise in an unqualified way calls for the exact knowledge of the boundaries of the natural order. Although these boundaries could have been intuitively clear within the pre-scientific world view, Aquinas’ understanding of miracles loses its coherence within the context of the contemporary science due to the constant expansion of what falls under the description of the scientific theories.

Although Aquinas admits that some natural causes may also remain hidden, the main force of his argument rests on considering God as totally imperceptible by the human sensation. With this being undeniably true, what seems surprising is that Aquinas strangely downplays the immanence of God in the contingent order in his treatment of miracles. It does not quite square with his metaphysics in which he treats every contingent being (\textit{ens}) as a~composite of \textit{esse} and \textit{essence}
%\label{ref:RNDYZ1p5pfEC6}(Aquinas, 1968).
\parencite[][]{aquinas_being_1968}. %
 Since the act of \textit{esse} is that by which God calls things into being and maintains them in existence, by this very act He makes Himself intimately present in creation. In his \textit{Summa Theologica} Aquinas clearly tied the Divine immanence to creatures’ existing by stating: `` \ldots a~thing’s existence is more interior and deep than anything else \ldots and hence it is necessary for God to exist in all things, and intimately so'' 
%\label{ref:RNDWFXxD4gqBz}(Aquinas, 1981, \textit{Summa Theologica}, I.8.1).
\parencite[][\textit{Summa Theologica}, I.8.1]{aquinas_summa_1981}. %
 In other words, the reason why anything whose existing is not its defining essence derives from and depends for its existence on the Creator. As a~result, there is profound closeness between creature as effect and Creator as the cause in this dependence. God is immanent in the natural order.

A~marked shift in the understanding of the nature of miraculous events occurred with the onset of the scientific method in the 17\textsuperscript{th} century when the fabric of the Universe ceased to be perceived substantially in favor of this fabric taking a~form of a~mathematical structure of the laws of nature. Contrary to the position of Aquinas where God would supplement the workings of nature with His direct intervention, the only possibility for God to act beyond nature is to expressly violate its laws. In such case miracles mean exceptional events that disrupt the uniformity of nature. According to David Hume, the scientific laws are discovered inductively based on repeated observations of regularities in nature indicating that the events observed occur with high probability due to the deterministic course that these events take. Since according to Hume miracles are rare events, that is, they have low probability, their empirical evidence can never outweigh the evidence of inductively accumulated data in support of the nature’s routine trajectories
%\label{ref:RNDjSnMuPU4Ii}(Hume, 2008, pp.79–95).
\parencite[][pp.79–95]{hume_enquiry_2008}. %
 Consequently, the violation of the laws of nature cannot not be established with certitude proper to the scientific method and miraculous events lack their credibility. De Cruz and De Smedt argue that such conceptualization of miracles is incoherent because if they occurred with higher probability, ``they would not be violating an established law of nature in the first place'' 
%\label{ref:RNDfsW7HT4T6e}(De Cruz and De Smedt, 2015, p.159).
\parencite[][p.159]{de_cruz_natural_2015}.%


\section*{The standard model}
The main purpose of applying the cognitive standard model of the formation of religious beliefs to the study of the miraculous events is to show why the human mind intuitively places an intentional agency as the cause of an event when its direct physical cause remains hidden. It turns out that the human mind has a~marked conceptual bias resulting in the content – specific cognition that makes the human mind handle different kinds of information with different weight
%\label{ref:RNDf8THkrOgwP}(Barrett, 2011, pp.35–39).
\parencite[][pp.35–39]{barrett_cognitive_2011}. %
 This type of cognition favors a~set of intuitive expectations on the nature of the world and what course of the natural phenomena is most probable. These expectations combine to what is designated as the \textit{folk ontology} 
%\label{ref:RNDkBjb7tiE0E}(e.g. Barrett, 2011, pp.58–95).
\parencite[e.g.][pp.58–95]{barrett_cognitive_2011}. %
 Since folk ontology amounts to the adaptively and developmentally acquired common sense knowledge of the Universe without the aid of the contemporary scientific method it may be to a~reasonable approximation viewed as coinciding with what constitutes the natural order according to Aquinas.

Pascal Boyer proposed that the religious beliefs feed predominantly on the intuitive (non-reflective) concepts to make these beliefs operative in the real-time activity
%\label{ref:RNDj0jiLjvhFh}(Boyer, 1994, 2001).
\parencite[][]{boyer_religion_2001}. %
 Also, he argued that the quickly disseminating religious concepts are those that reach beyond the folk ontology only to a~small degree. These concepts were subsequently named as \textit{minimally counterintuitive} (MCI) 
%\label{ref:RND7xQkMhKiEG}(Barrett, 2000).
\parencite[][]{barrett_exploring_2000}. %
 Minimal counterintuitivity means that only a~few beliefs of the entire intuitive conceptual equipment would not be satisfied thereby making a~given concept or event attractive and astonishing while with other intuitions unchallenged the concept or event in hand would be easily remembered and disseminated. Moreover, these concepts must have sufficient \textit{inferential potential} to produce reflective beliefs necessary to make sense of what is being observed and experienced in reality. It turns out the minimally counterintuitive intentional agents equipped with mental states are employed by the human mind as the principal meaning making tools. And, most importantly from the point of view of this study, De Cruz and De Smedt 
%\label{ref:RNDiJwT2xgVuZ}(2015, pp.161–165)
\parencite*[][pp.161–165]{de_cruz_natural_2015} %
 argue that miracles and the accounts of their stories qualify as the MCI events.

The reason why the human mind interprets natural events with no visible cause as the workings of intentional agents flows from two basic cognitive mechanisms. The first mechanism was originally suggested by Stephen Guthrie and its main task is to over-interpret a~self-perpelled motion as caused by an intentional agent equipped with mental states
%\label{ref:RNDVUX2puV7KX}(Guthrie, 1993).
\parencite[][]{guthrie_faces_1993}. %
 Barrett 
%\label{ref:RNDTmthp1Kbs6}(2000)
\parencite*[][]{barrett_exploring_2000} %
 termed this mechanism as the \textit{hyperactive agency detection device} (HADD). Since such a~motion cannot be explained as the action of a~visible mechanical cause, it does not satisfy the expectation of physicality whereby it activates the HADD so that the assault of a~predator can be prevented and chances for the reproductive success maintained. The second mechanism, namely the \textit{theory of mind} (ToM) otherwise called the \textit{folk psychology}, complements the workings of the HADD by positing mental states and processes that might have led to the predatory behaviors 
%\label{ref:RND8Swxjac1Er}(e.g. Barrett, 2011, pp.74–77).
\parencite[e.g.][pp.74–77]{barrett_cognitive_2011}. %
 This set of ideas combines into what in the cognitive science of religion is termed as \textit{the standard model of the formation of religious beliefs} 
%\label{ref:RNDY42Qvp7S2V}(Murray and Goldberg, 2010, pp.183–189).
\parencite[][pp.183–189]{murray_evolutionary_2010}. %


\section*{Counterintuitivity relativised}
The static Aristotelian Universe studied with qualitative pre-scientific methods had rather little place for surprise and unexpected discovery thereby assuring the relative stability and credibility of intuitions proper to the folk ontology. Contrary to this, however, the development of contemporary science results in a~dynamically changing picture of the world which gradually moves away from these intuitions towards pictures of more generalized and abstract character. The key question at this point is how the human mind responds to this development in its formation of beliefs on the causal activity of intentional agents. In his analysis on what happens as the human mind comprehends the outcomes of the theory of evolution Barrett pointed to two constituents of this response. Taking into account that the complexity of life in the Universe is now known to be the effect of the workings of the Darwinian law of natural selection and not a~purposeful activity of an intelligent designer Barrett asserts that ``we do not simply outgrow the tendency to see the purpose in the world but have to learn to override it''
%\label{ref:RNDCdyeHOl4EK}(Barrett, 2011, p.71).
\parencite[][p.71]{barrett_cognitive_2011}. %
 He comes to this conclusion based on the evidence of empirical studies showing that in this instance the folk ontology intuitions are not easily erased even in the conditions of the high level of scientific literacy 
%\label{ref:RND8GDICmL7C8}(e.g. Casler and Kelemen, 2008).
\parencite[e.g.][]{casler_developmental_2008}.%


A~somewhat different scenario was indicated by Grygiel
%\label{ref:RNDfdn9uA0397}(2020)
\parencite*[][]{grygiel_doctrine_2020} %
 who focused on the concept of a~field which is one of the most brilliant ideas of the contemporary physics. The picture of reality based on physical fields challenges the intuitive belief proper to the folk ontology that motion occurs through contact with a~visible cause. Since fields mediate the action of forces over the entire space in an invisible manner, their effects are perceived as having a~hidden cause. This means that while for the pre-scientific generations events such as moving iron strips with a~magnet could be interpreted as resulting from the action of an intentional agency and quickly acquire religious significance, this is no longer the case for those who are acquainted with the scientific world view based on the notion of the field as a~fundamental theoretical object. On this view, a~ringing cell phone is neither surprising nor an intentional agency is posited to explain the activation of the phone. It is the electromagnetic field of the cell phone network that causes it to ring and this event is no longer counterintuitive. The folk ontology seems to be in much greater recess in this instance as compared to a~purposeful designer invoked to account for the complexity of life in the Universe. This conclusion agrees with the findings of De Cruz and De Smedt 
%\label{ref:RNDJRHZhYVet6}(2012)
\parencite*[][]{de_cruz_evolved_2012} %
 who show the cognitive biases can be offset by the growth of the scientific knowledge and its subsequent cultural dissemination.

In order to reflect the dynamic growth of the scientific knowledge and its influence on the formation of religious beliefs, Grygiel introduced the concepts of the \textit{vincible} and \textit{invincible counterintuitivity}
%\label{ref:RNDQ2kDT75Fo5}(Grygiel, 2017; see also Van Eyghen, 2020).
\parencites[][]{grygiel_science_2017}[see also][]{van_eyghen_religious_2020}. %
 These concepts are useful in understanding a~boundary situation if counterintuitivity was ultimately overcome upon the formulation of a~scientific theory of everything capable of grasping the full meaning of reality as proposed by Hawking and Mlodinov 
%\label{ref:RND4sZujugWEq}(2010),
\parencite*[][]{hawking_grand_2010}, %
 for instance. It is commonly agreed, however, that such expectations are illusory for both practical and theoretical reasons 
%\label{ref:RNDPujZiFiVUt}(e.g. Heller, 2006).
\parencite[e.g.][]{heller_teorie_2006}. %
 The practical reasons were indicated by Albert Einstein who maintained that science discovers a~very small part of the depth of the physical reality only and most of it is hidden as a~profound mystery 
%\label{ref:RNDQ5wYdhMo27}(Einstein, 1931).
\parencite[][]{einstein_world_1931}. %
 In other worlds, nature conceals enough novelties in her womb to violate the intuitions of many future generations of scientists.

The theoretical reasons were accounted for by Michael Heller as he named three irremovable gaps in knowledge that cannot be covered by the progress science: the \textit{ontological}, the \textit{epistemological} and the \textit{axiological}
%\label{ref:RNDi3vF7dvejy}(Heller, 2003a, pp.142–143).
\parencite[][pp.142–143]{heller_chaos_2003}. %
 The ontological gap refers to the Leibnizian question of why there exists something rather than nothing while the epistemological gap prompts the famous Einsteinian puzzlement with the incomprehensibility of the comprehensibility of the Universe. Since the pertinent answers fall outside the competence of science, the problem of the ultimate origin of the structuring of the Universe will never be scientifically resolved. Grygiel argues that even if traces of the pre-scientific folk ontology were still operative, this only adds to how well human mind is actually protected from conquering all counterintuitivity: should the intuitive conceptual biases be ever overcome and should the folk ontology ever catch up with the actual state of the art in science, it is unlikely that nature itself will ever run out of surprises 
%\label{ref:RNDXwYn3taisJ}(Grygiel, 2017).
\parencite[][]{grygiel_science_2017}.%


Although it seems rational to accept that the folk ontology can be at least tamed or even transformed to some degree following the progress of science, this does not happen in course of simple inductive generalizations of frequently occurring empirical evidence. Rather, transformations of these ontologies take place as a~result of changes in the theoretical description of reality and their subsequent cultural assimilation to form a~scientifically informed world view. According to the well known holistic stance of Willard V.O. Quine, scientific theories themselves can never be abolished by a~single experiment because they constitute a~set of interconnected statements which ``face the tribunal of sense experience not individually but only as a~corporate body''
%\label{ref:RNDrzfHj0dzzj}(Quine, 1951, p.30).
\parencite[][p.30]{Quine1951-QUITDO-3}. %
 Moreover, theories function in conjunction with a~broader context of ``elaborate myths and fictions'' and need experimental agreement along their ``empirical edge'' only 
%\label{ref:RNDRK9jv0dtAY}(Quine, 1951, p.42).
\parencite[][p.42]{Quine1951-QUITDO-3}. %
 This means that the folk ontologies can easily coexist with the contradictory empirical evidence indicating that even frequent events violating the intuitive expectations will generate beliefs that they are caused by hidden intentional agents. Contrary to Hume’s empiricism, these events could still qualify as violating the laws of nature and interpreted as caused by God. Would they still be miracles, though?

The answer to this question must be sought in the classical qualification of miracles as rare events (with low probability) that lead to wonder and astonishment. Placed in the contemporary world of cell phones our medieval ancestors would most likely get used to them ringing without a~visible cause but they would still lack the theoretical basis to understand the physical nature of why these phones activate. Unlike the attribution of the intentional agency as a~cause of the counterintuitive events, wonder and astonishment are of psychological nature and they arise due to the rarity of the events experienced. Certainly, these events must be counterintuitive because intuitions would have no chance of being formed out of what is unusual. It turns out that the connection between the low probability of events and their surprising character comes to the fore in the information theory where the entropy of a~random variable is the average level of surprise or uncertainty in the possible outcomes of the variable. More importantly, however, the HADD mechanism has been also discovered to be sensitive to the traces of the activity of the intentional agencies in the form of highly organized patterns which by their nature exhibit low probability of appearance
%\label{ref:RNDQbM3jtYEds}(Barrett, 2004, pp.36–39; Grygiel, 2020).
\parencites[][pp.36–39]{barrett_why_2004}[][]{grygiel_doctrine_2020}.%


The cognitive considerations carried out in this section throw an interesting light on the coherence of the classical understanding of miracles. Firstly, it can be maintained that miracles are caused by a~hidden intentional agency which is attributed by the HADD mechanism responding to the violation of the intuitive folk ontologies. This in turn does not mean the bypassing or the violation of the constancy of the laws of nature at all and it squares with the ontological views of what constitutes the fundamental fabric of reality consistent with the contemporary science. Furthermore, miraculous events may also remain rare because the HADD mechanism will properly respond to their counterintuitivity and the psychological effect of wonder and astonishment will follow. Astonishingly enough, the application of the standard cognitive model of the formation of religious corroborates the consistency of the key components of the classical understanding of miracles in the context of the contemporary science with a~marked bent towards the Augustinian view where miracles are treated as contrary to nature not because they violate its laws but because these laws remain unknown. The demonstration of further consistency with Augustine calls for a~theological analysis which will be offered in the following section.

\section*{Challenging supernaturality}
It is commonly accepted that the conceptual foundation of the cognitive sciences is quite foreign to that of the classical philosophical discourse
%\label{ref:RNDcRhk8D4F1S}(e.g. Brożek, 2013; Grygiel, 2011).
\parencites[e.g.][]{brozek_philosophy_2013}[][]{grygiel_metodologiczne_2011}. %
 This pertains to the cognitive science of religion as well. It seems surprising, however, that researchers in the area of the cognitive science of religion do not properly address a~certain marked conceptual inconsistency that evidently plagues their inquiries. This concerns the direct match made between \textit{supernatural} and \textit{counterintuitive} 
%\label{ref:RNDodu4f4tug0}(e.g. Barrett, 2011, p.97)
\parencite[e.g.][p.97]{barrett_cognitive_2011} %
 which will lead to a~clear confusion in case when a~purely natural event has no visible cause and is explained by means of the action of an intentional agent. Consequently, the religious interpretation of such events will not discriminate between what is of the natural and what is of the Divine origin. This evident difficulty is a~good starting point to address the last issue regarding the coherence of the classical notion of a~miracle within the context of the contemporary science with the aid of the standard cognitive model of the formation of religious beliefs, namely, that of the Divine action.

The problem of the Divine action involved in the miraculous events within the world view consistent with the contemporary science has been the subject of many vivid discussions and controversies. The main point of contention is whether this world view admits of the so called \textit{special Divine action} which is understood as divine action which reaches beyond creation of the Universe and beyond the ordinary maintaining it by God in its existence. To put things in in short, is whether God can somehow intervene within the network laws that govern the Universe either by violating them or exploiting the loopholes in the causal closure of the Universe. This problem has become the focal point of a~major research project undertaken in the years 1988-2003 by a~large group of scientists, philosophers and theologians and termed by Wildman
%\label{ref:RNDdR6ymgg6uY}(2004)
\parencite*[][]{wildman_divine_2004} %
 as ``The Divine Action Project''. Most of the project’s participants conceded to the idea that the preferred mode of the Divine action in the world is \textit{non-interventionistic} on the grounds that it would be contradictory to hold that on one side God runs the Universe according to its laws and on the other disrupts this order by His interventions 
%\label{ref:RNDnWaW6PekcW}(e.g. Peacocke, 1995; Russell, 2001).
\parencites[e.g.][]{peacocke_gods_1995}[][]{russell_divine_2001}. %
 This stance has been objected to by Plantinga who argues that one can claim compatibility of special divine action by means of interventions in the context of the contemporary scientific theories such as quantum mechanics, for instance 
%\label{ref:RNDCfr6Ae0RRr}(Plantinga, 2008).
\parencite[][]{plantinga_what_2008}. %
 A~prominent representative of this group, an English theoretical physicist and theologian John Polkinghorne 
%\label{ref:RND0Qq2dOOCsa}(1998, p.92),
\parencite*[][p.92]{polkinghorne_science_1998}, %
 asserts the following:

\myquote{
It is theologically incredible that God acts as a~kind of celestial conjurer, doing occasional tricks to astonish people but most of the time not bothering. Such a~capricious notion of divine action is totally unacceptable. The main problem of miracle, from the theological point of view, is how such wholly exceptional events can be reconciled with divine consistency.

}
Moreover, Polkinghorne’s thinking reveals certain closeness to the terminology employed in the cognitive science of religion as suggest the use of the concept of \textit{regime}, that is the domain of experience, in which the human mind is accustomed to a~certain course of natural events. This is a~close match of the \textit{folk ontology}. As an example he introduces the phase changes which can lead to drastic changes of behavior governed by the same physical laws. In other words, these drastic changes are subject to unchangeable physical laws indicating that miracles do not have to imply the violation of these laws. Another suitable example would be the Einstein field equation which for small gravitational forces predicts the flatness of space-time consistent with the folk ontology while for strong gravity induced by large masses it predicts space-time curvature entirely foreign to our common sense perception.

Although Polkinghorne does not dwell on the issue of miracles at length, he makes an interesting observation that leaves a~valuable clue as to how to interpret miracles in the context of the scientific picture of the world. He states that: ``Miracles are not to be interpreted as divine acts against the laws of nature (for these law are themselves expressions of God’s will) but as more profound revelations of the character of the Divine relationship to creation''
%\label{ref:RNDWX77FDc5ZR}(Polkinghorne, 1998, p.93).
\parencite[][p.93]{polkinghorne_science_1998}. %
 In the clearly Augustinian tone Polkinghorne implies here that miracles should not infringe upon the deep harmony that exists between God and His creation and that they should be correlates of the natural course of events. It turns out that according to the methodological reflection on the growth of scientific knowledge that has been already addressed above, a~\textit{non-interventionist} model of the divine action is now preferred 
%\label{ref:RNDeeAImeoP8N}(e.g. Heller, 2002, pp.117–121).
\parencite[e.g.][pp.117–121]{heller_sens_2002}. %
 This model is best articulated in the context of \textit{panentheism}: ``all-in-God'', which is an ontological stance relating the natural and the supernatural orders as the natural realm being immersed in the supernatural 
%\label{ref:RNDHgbFGNrQBY}(e.g. Clayton and Peacocke, 2004).
\parencite[e.g.][]{clayton_whom_2004}.%


The non-intervetnionist model of the Divine action in the world neutralizes the difficulty of the match between the counterintuitive and the supernatural because in this model the immanent God achieves His goals solely through the workings of the laws of nature in which He is constantly present. The non-interventionist model squares well with a~broader set of ideas on the relation of the natural to the supernatural known as panenthesim (``all in God'')
%\label{ref:RNDz10LEmjPyo}(e.g. Peacocke, 2004).
\parencite[e.g.][]{peacocke_articulating_2004}. %
 Since in such circumstances God’s action occurs through the powers of nature only, counterintuitivity pertains exclusively to the events that follow laws unknown to science and not occur due to the divine interventions. This shows remarkable consistence with the Augustinian understanding of miracles as the works of nature itself because in the panentheistic setting all natural events qualify as supernatural since in each instance they reflect the divine causality mediated through the laws of nature. It turns out that the conceptual inadequacy of the division of all that exists into natural and supernatural domains has been pointed out by theologians. For instance, a~prominent German theologian Gisbert Greshake 
%\label{ref:RNDi30mMDj0wf}(1997, p.37)
\parencite*[][p.37]{greshake_dreieine_1997} %
 states:

\myquote{
In fact, there is no such thing as a~purely natural order. What creation is and is called is in fact always that world which was founded in the Son out of the most free love and was created for the Son and his ``\textit{Pleroma}'' (fulness). It is that world in which man was called to life with the triune God. It is the world of which the Prolog to the Gospel of St. John speaks that Logos has always been in it and of which the Old Testament clearly says that the Holy Spirit fills it and works~in~it.

}
Consequently, the acceptance of the stance of panentheism mandates the neutralization of the commonly accepted division of reality into natural and supernatural. The Divine immanence in the created (contingent) order makes everything that happens the work of God and there is not a~bit of reality that lies beyond His constant causal influence. Consequently, Greshake does not hesitate to claim that since according to the Christian doctrine God is triune in His essence, namely the unity of three Divine persons, there arises an urgent need for the Trinitarian ontology, cosmology, anthropology and sociology
%\label{ref:RNDRtgkaCnuBn}(Greshake, 1997, p.42).
\parencite[][p.42]{greshake_dreieine_1997}. %
 Furthermore, the immanence of God in creation can be viewed as manifesting itself as the rationality of the Universe. Inspired by a~Danish historian of science, Olaf Pedersen 
%\label{ref:RNDheNqT8xfAN}(2007, pp.63–65),
\parencite*[][pp.63–65]{pedersen_two_2007}, %
 Heller admits ``that Christ the Logos implies that God’s immanence in the world in its rationality 
%\label{ref:RNDB4VC7za3qG}(Heller, 2003b, p.57).
\parencite[][p.57]{heller_scientific_2003}. %
 As Heller maintains, this rationality can find varying expression in the human thought resulting in that this rationality can assume different ``incarnations''. While the first incarnation of this rationality was the Greek philosophy, its contemporary incarnation is the scientific rationality. Moreover, Heller 
%\label{ref:RND4Fd3MlFypB}(2016)
\parencite*[][]{heller_teologia_2016} %
 stipulates that the scientific rationality is superior to that of the Greeks for it unveils the Logos immanent in nature in a~greater degree.

\section*{Conclusions}
In conclusion of the study it can be asserted that the analysis of the nature of the miraculous events with the use of the standard cognitive model of the formation of religious beliefs reveals that this model comes to great assistance in demonstrating the marked coherence of the classical Augustinian understanding of miracles with the world view supported by the contemporary science. This happens at the expense of the distinction between the natural and supernatural because the immanence of God in nature makes everything that happens in nature directly His work and thus supernatural. Strictly speaking, there is no purely natural order. It is now a~point of controversy whether the cognitive mechanisms responsible for the attribution of an intentional agency to an event with a~hidden cause are error management or truth-tracking tools
%\label{ref:RNDhi3KYAxopq}(Barrett and Church, 2013; Van Eyghen, 2019).
\parencites[][]{barrett_should_2013}[][]{van_eyghen_is_2019}. %
 Their primary role is to prompt the recipient or the observer of a~counterintuitive event to read it as being caused by God and to interpret this event as miraculous. As it has been explained, this process is contextual for the attribution of an intentional agent is relativised to what the one who observes the miracle considers to be consistent with the expectations of the folk ontology he or she is equipped with. When this attribution is made, a~given event immediately acquires religious meaning because God conceptualized as an intentional agent easily combines into most of the religious narratives present in the contemporary culture 
%\label{ref:RNDz8GXAh37pX}(e.g. De Cruz and De Smedt, 2015, pp.172–178).
\parencite[e.g.][pp.172–178]{de_cruz_natural_2015}. %
 Contextuality of miracles thus exercised prevents the invalidation of miracles once the events that are associated with them acquire their natural scientific explanations reaching beyond the content of the folk ontology. In short, canonized saints remain canonized. The semantic reading of miracles remains in accord with the hermeneutical attitudes in the contemporary theology and has become the standard understanding in instances when miracles serve as arguments in favor of the Divine action 
%\label{ref:RNDFfJg1hCAzw}(e.g. Rusecki, 2006, pp.211–290).
\parencite[e.g.][pp.211–290]{rusecki_traktat_2006}. %
 Also, as De Cruz and De Smedt 
%\label{ref:RND5jacUx1XKS}(2015, p.160)
\parencite*[][p.160]{de_cruz_natural_2015} %
 point out, the cognitive support for miracles enhances their credibility because the unusualness of such an event is no longer the matter of an entirely subjective response in line with personal tastes and predilections but is inferred by the natural powers of human cognition.

The very last remark of this study is addressed to the critical mind: is it possible that the apparatus of the cognitive science of religion offers an ultimate proof that by the belief in miracles the human mind is fooled into thinking that some transcendent reality has dominion over the Universe? It turns out the scientific studies of religion using cognitive methods receive a~variety of philosophical interpretations among which those claiming that explaining religion means explaining religion away appear as quite dominant
%\label{ref:RNDivr7veuisS}(e.g. Boyer, 2001, p.76; Pyysiäinen, 2001, pp.78–79).
\parencites[e.g.][p.76]{boyer_religion_2001}[][pp.78–79]{pyysiainen_cognition_2001}. %
 These interpretations attempt at disproving the truthfulness of the religious claims based on the knowledge of the mechanism of their origin. As Murray and Goldberg 
%\label{ref:RND3u9aFJdkad}(2010, pp.193–199)
\parencite*[][pp.193–199]{murray_evolutionary_2010} %
 point out, however, this is but a~typical instance of a~logical error designated as the \textit{genetic fallacy} in which to know how a~given belief is formed is taken as the proof that the belief is true. By the same token one can claim that it would be a~genetic fallacy to discredit the religious meaning of miracles as caused by God just because the origins of this belief were scientifically explained by the tools of the cognitive science of religion. In addition to this Barrett suggests that genetic explanations of religion are in force when one does not believe that God exist only. If he or she does, the knowledge on how they arrived their beliefs serves to better understand their faith 
%\label{ref:RNDg9lWzR7uyD}(Barrett, 2011, pp.148–155; see also Wszołek, 2004, pp.51–54).
\parencites[][pp.148–155]{barrett_cognitive_2011}[see also][pp.51–54]{wszolek_wprowadzenie_2004}.%


\end{artengenv}
