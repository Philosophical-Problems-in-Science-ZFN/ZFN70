\begin{newrevplenv}{Robert Janusz}
	{Ważny krok ku zrozumieniu tożsamości filozofii Ośrodka Badań Interdyscyplinarnych}
	{Ważny krok ku zrozumieniu tożsamości\ldots}
	{Ważny krok ku zrozumieniu tożsamości filozofii Ośrodka Badań\\
	Interdyscyplinarnych}
	{Obserwatorium Watykańskie}
	{Kamil Trombik, \textit{Koncepcje filozofii przyrody w~Papieskiej Akademii Teologicznej w~Krakowie w~latach 1978--1993. Studium historyczno-filozoficzne}, Wydawnictwo «scriptum», Kraków 2021, ss.~263.}

\lettrine[loversize=0.13,lines=2,lraise=-0.03,nindent=0em,findent=0.2pt]%
{O}{}mawiana książka powstała na podstawie rozprawy doktorskiej Kamila Trombika obronionej na Wydziale Filozoficznym UPJPII w~Krakowie w~2020~r. (promotorem był Paweł Polak); pracę recenzowali: Jacek Rodzeń oraz Tadeusz Sierotowicz; publikacja została dofinansowana z~subwencji UPJPII w~Krakowie (2020).
Zobaczmy zatem, jak Autor podjął się realizacji nakreślonego w~tytule zadania.

We ,,Wstępie'' Autor omawia ,,filozofię przyrody'' jako dyscyplinę filozoficzną w~jej historycznym rozwoju, jej znaczenie, wielopostaciowość, zgodność z~naukami przyrodniczymi, polskie ośrodki, które ją uprawiały, w~szczególności Papieską Akademię Teologiczną w~Krakowie (PAT), z~jej różnymi kręgami badawczymi. Autor stawia pytanie-zadanie robocze: ,,jakie koncepcje filozofii przyrody były reprezentowane w~środowisku PAT w~Krakowie w~latach 1978--1993?''. Przedział czasowy jest wyznaczony działalnością seminariów interdyscyplinarnych organizowanych przez Michała Hellera i~Józefa Życińskiego w~rozwijającej się się uczelni. Mają temu sprostać: metoda rekonstrukcji, wywiady, analiza historyczno-filozoficzna oraz synteza, w~ujęciu chronologicznym.

Rozdział pierwszy pracy Kamila Trombika stanowi analizę źródłową i~kontekstową filozofii przyrody okresu powojennego, zawężając go do PAT oraz ustalonych granic czasowych, z~docenieniem przedwojennej, wolnej od (neo)pozytywistycznej ideologii, krakowskiej filozofii przyrody sięgającej XIX~w. i~filozofii przyrody rozwijanej też na Wydziale Teologicznym UJ, z~jego Katedrą Filozofii Chrześcijańskiej (1882). Po wojnie i~narzuceniu materialistycznej ideologii marksistowskiej wielu filozofów działało w~środowiskach katolickich: w~zakresie neoscholastyki (m.in. K.~Kłósak i~T.~Wojciechowski) oraz grupowało się wokół K.~Wojtyły (dialog interdyscyplinarny przyrodnicy-filozofowie), zwłaszcza po jego nominacji biskupiej (1963). Przyszły papież uważał te zagadnienia, dyskutowane na Franciszkańskiej 3, za problemy najwyższej wagi --~za podstawy dla filozofii i~światopoglądu
%\label{ref:RNDNwPvXK3rEG}(Trombik, 2021, s.~71–73, 83–84).
\parencite[][s.~71–73, 83–84]{trombik_koncepcje_2021}. %
 Udawało się wtedy wypracowywać wspólne stanowiska między filozofami i~przyrodnikami. Michał Heller został zatrudniony na Papieskim Wydziale Teologicznym (PWT) w~1972~r., zaś życzeniem Arcybiskupa było kontynuowanie tradycyjnych ,,spotkań i~współpracy filozofów, zwłaszcza filozofów przyrody z~fizykami''. Nieudane próby połączenia PWT i~Wydziału Filozoficznego Księży Jezuitów sprawiły, że K.~Wojtyła pragnął odrębnego instytutu filozoficznego, koniecznego m.in. jako ośrodka dyskusji światopoglądowych z~udziałem duchownych. Do ich grona dołączył m.in. Józef Życiński (doktorant K.~Kłósaka), który został zatrudniony na PWT jako asystent (1974). W~1976~r. powołano Papieski Wydział Filozoficzny w~Krakowie, co nie było mile widziane przez władze komunistyczne, stąd watykańskiej decyzji nie ujawniono, a~w 1978~r. powołano Instytut Filozofii przy PWT.

Rozdział drugi (lata 1978--1986) opisuje nowe, nieneoscholastyczne podejście do filozofii przyrody w~PWT/PAT. Graniczna data (1978) --~to wybór kard. Wojtyły na papieża, zaś 1986~r. jest wyznaczony programowym artykułem Hellera dotyczącym istoty ,,filozofii w~nauce''. Papieską Akademię Teologiczną z~Wydziałem Filozoficznym powołał Jan Paweł~II w~1981 r., tuż przed stanem wojennym, jednak jego \textit{motu proprio} ,,Beata Hedvigs'' zostało promulgowane 24~marca 1982~r., a~działalność rozpoczęła się w~r.a. 1982/83. Od jesieni 1978~r. Heller i~Życiński organizowali konwersatoria interdyscyplinarne nawiązujące do tradycji kard. Wojtyły, co stanowi pośrednio początek Ośrodka Badań Interdyscyplinarnych, kształtującego się stopniowo na bazie wcześniejszego Ośrodka Studiów Interdyscyplinarnych (łączącego filozofię przyrody, filozofię i~historię nauki oraz relacje nauka-wiara). Czasopismo \textit{Zagadnienia Filozoficzne w~Nauce}, wzrastało wraz z~nowymi strukturami akademickimi. Od 1983~r. Ośrodek zaczął współpracę z~Obserwatorium Watykańskim, wydając \textit{Philosophy in Science}, (Tucson, AZ, USA), z~programowym artykułem ,,filozofii w~nauce'': ,,The Evolving Interaction Between Philosophy and the Sciences: Towards a~Self-Critical Philosophy'' (W.G.~Stoeger~SJ), postulującym dynamiczność i~otwartość filozofii na naukę. Z~Obserwatorium zorganizowano Sympozjum ,,The Galileo Affair: a~Meeting on Faith and Science'' (1984). Neotomistyczna filozofia przyrody na PAT zaczęła słabnąć, zwłaszcza po śmierci K.~Kłósaka (1982), znanego krytyka podważającego naukowość marksizmu tuż po II wojnie światowej. Jednocześnie stroniący od kręgu Hellera T.~Wojciechowski nie znalazł kontynuatorów. Heller i~Życiński kierowali się ideą (parafrazując Kanta): ,,Filozofia bez nauki jest kulawa, nauka bez filozofii jest ślepa''. Skutkowało to odejściem od klasycznej metafizyki, jej pojęć i~metod, gdyż nauka zmieniła obraz świata, co nie przekreślało samej filozofii, ale dogmatyzm, przez który nie wchodzi ona w~dialog z~nauką; otwarte pozostawały problemy, np. istnienia i~natury Boga --~analizowane w~kontekście danych naukowych, zwłaszcza przez Hellera w~jego ,,filozofii w~nauce (fizyce)''
%\label{ref:RNDYQk5GyXWk9}(zob. Heller, 2019; zob. także Polak, 2019).
\parencites[zob.][]{heller_how_2019}[zob. także][]{polak_philosophy_2019}. %
 Życiński zaś zwrócił się ku filozofii nauki (o przyrodzie) w~kontekście filozofii analitycznej. Podejście interdyscyplinarne stawało się bliskie antyfundacjonizmowi; sądzono, że falsyfikowalność w~nauce \textit{analogicznie} (sic!) powinna rugować dogmatyzm filozoficzny, sprzyjając wolnym koncepcjom. W~kontekście politycznym PRL stawiało to filozofię chrześcijańską na pozycji dość oryginalnej.

Trzeci rozdział opisuje rozkwit tej oryginalnej filozofii przyrody na PAT (1987--1993), gdyż kierunek arystotelesowsko-tomistyczny nie stał się alternatywą wobec środowiska Hellera i~Życińskiego; ,,filozofia w~nauce'' stopniowo stawała się zatem dominująca. W~1993~r. zakończyły się konwersatoria interdyscyplinarne, a~Wydział Filozoficzny PAT przechodził gruntowne przemiany. Poszerzały się także kontakty zagraniczne (m.in. z: The Catholic University of America, Waszyngton; L'Université Catholique de Louvain (Belgia); Specola Vaticana, Castel Gandolfo; Institut für die Wissenschaften vom Menschen, Wiedeń), a~także z~lokalnymi: Wydziałem Filozoficznym Towarzystwa Jezusowego i~z Kolegium Filozoficzno-Teologicznym \mbox{oo.~dominikanów}. W~1989~r. nastąpiło zrównanie tytułów akademickich z~nadawanymi przez państwo. Co ciekawe, w~1991~r. zatwierdzono nowy statut z~poprawką dotyczącą ,,filozofii chrześcijańskiej'', od którego to terminu dystansowali się pracownicy Wydziału Filozoficznego
%\label{ref:RNDQYLzda0Guq}(Trombik, 2021, s.~168–169).
\parencite[][s.~168–169]{trombik_koncepcje_2021}. %
 Kiedy filozofia przyrody stawała się coraz bardziej autonomiczna, próbowano przebudowywać filozofię chrześcijańską, tracąc Kłósakowski uniwersalizm i~maksymalizm; podobnie postępowało rozdwojenie w~samej filozofii przyrody: między Helerem (z Życińskim) oraz Wojciechowskim (z Lenartowiczem). Pozycja OBI umacniała się wszechstronnie, a~same konwersatoria przekształciły się w~konferencje metodologiczne organizowane raz do roku. Warto podkreślić inicjatywę ,,Wykładów Coyne'a'' (1990--1993), związaną z~listem Jana Pawła~II do Dyrektora Obserwatorium Watykańskiego z~okazji 300-lecia Newtonowskich \textit{Principiów}. Była to papieska zachęta do współpracy teologów, filozofów i~uczonych nad relacjami nauka-wiara. \textit{Zagadnienia Filozoficzne w~Nauce} stawały się coraz wyrazistsze z~ugruntowaną już nazwą OBI (Ośrodek Badań Interdyscyplinarnych, 1990). W~kontekście niedostrzegania owoców nauki przez filozofię, Heller zwracał uwagę na to, że filozofia przyrody ma bogatą tradycję (np. matematyczność/racjonalność świata), a~pojęcia tradycyjne zanikają w~społeczeństwie (np. pojęcie substancji, a~pojęcie materii jest nawet zbyteczne). Nastawiony naukowo Heller podchodził ostrożnie do ,,metafizyki przyrody'' --~filozofując poza obszarem nauk, uważał transcendencję za horyzont poznawczy dla jakiejkolwiek refleksji nad nauką, co zaowocowało jego ,,teologią nauki'' (rozumiał racjonalność jako wartość); odrzucał więc ontologiczny naturalizm i~scjentyzm. Zostało to chłodno przyjęte przez filozofów chrześcijańskich, zwłaszcza neotomistów; ,,filozofia w~nauce'' przechylała się w~ich rozumieniu w~stronę ,,filozofii nauki'', przy czym ta ,,nauka'' była przez nich traktowana z~pozycji izolacji od filozofii. Heller zaś widział filozofię ,,w'' nauce (fizyce), próbując tłumaczyć to na wiele sposobów, np. matematyczność świata, korespondująca z~badaniem świata, oznacza, że sama przyroda jest matematyczna. Nieco odmiennie traktował filozofię ,,w'' nauce Życiński. Podchodził on do problemów bardziej od strony tradycyjnych pytań metafizycznych (istnienie i~natura Boga itd.), co zbliżało go do A.N.~Whiteheada, który wpisywał się w~metafizyki starożytne, cenione przez neoscholastykę, ale za fundament przyjmował właśnie filozofię przyrody, nie ontologię, i~cenił nowoczesną logikę. Tak więc Życiński jakby kontynuował dawniejszą tradycję krakowskiej filozofii przyrody; jednak trzymając się z~dala od samej ,,filozofii przyrody'' był bliżej jej metodologii, choć z~dala od formalizacji samej filozofii. Uważał, że różnorodność metod, języków itd. bierze się z~bogactwa uwarunkowań bytowych, pozwalających racjonalnie tłumaczyć obserwacje świata. Sądził, że jest możliwa wielka unifikacja metafizyczna (pogodzenie nauk przyrodniczych, humanistyki i~chrześcijaństwa). Tak więc Heller i~Życiński zaimplementowali w~PAT (z różnych stron) silny nurt ,,filozofii w~nauce'', która zaczęła się rozwijać (pierwsi jej uczniowie: W.~Skoczny, W.~Wójcik, Z.~Wolak, Z.~Liana i~in.) i~oddziaływać szerzej poprzez współpracowników i~sympatyków.

W~,,Zakończeniu'' Trombik podsumowuje stawiane pracy cele i~dostrzega (za Hellerem) wpływ kard. Wojtyły na formację tak krakowskiego środowiska, jak i~rozumienie roli nauki, filozofii i~teologii w~budowaniu kultury narodu, czego już szerszym świadectwem były seminaria interdyscyplinarne z~papieżem Janem Pawłem~II w~Castel Gandolfo. Obszerna ,,Bibliografia'' kończy 263-stronicową książkę niezawierającą indeksów.

Po przedstawieniu treści --~pora na drobne uwagi. Po pierwsze, benedyktyńska praca Kamila Trombika stanowi bardzo cenną pozycję godną polecenia tak historykom nauki, jak i~filozofom przyrody/nauki. Oceniam ją jako wzorową analizę historyczno-filozoficzną środowiska Wydziału Filozoficznego Papieskiej Akademii Teologicznej w~okresie poszukiwań swojej tożsamości, w~szczególnym czasie, gdy papieżem został krakowski kardynał, późniejszy św. Jan Paweł~II. Książka jest bardzo bogata w~istotne szczegóły historyczne, które ukazują zakorzenienie filozofii przyrody w~szerokiej kulturze niezmiernie trudnej historii Polski zniewolonej, stawiającej swe niepewne kroki wolności ,,formalnej''. Miarą gruntownego zrozumienia przez Autora samej istoty rzeczy oraz opanowania metodycznego warsztatu jest to, że niełatwe problemy historyczno-filozoficzne Trombik podejmuje swobodnie i~towarzyszy temu prostota przekazu, co czyni z~książki pasjonują lekturę. Akcent położony na historyczną analizę stwarza oczywisty niedosyt analizy samej filozofii (przyrody/nauki) na PAT, czego Autor sam jest świadom, i~nie jest oczywiście jego winą, że w~pojedynczej publikacji zaspokojenie tego niedosytu nie jest możliwe. Być może Trombik zamierza rozwijać swe badania w~przyszłych pracach na ten cel ukierunkowanych? W~końcu można życzyć UPJPII podobnych świetnych naukowców (nie tylko świeckich, ale i~duchownych), aby --~co tak cenił św. kard. Karol Wojtyła/Jan Paweł~II --~interdyscyplinarna głębia badań Prawdy i~życia Nią --~relacja nauka-wiara --~nie zostało zamienione na fajerwerki filozofiopodobne, niezdolne uwolnić myślenie z~jego technokratyczno-socjologicznych manipulacji ideologicznych.



%-------------------------------


\selectlanguage{english}
\vspace{5mm}%
\begin{flushright}
{\chaptitleeng\color{black!50}{An important step towards an understanding of the type of the philosophy at the Center of Interdisciplinary Studies}}
\end{flushright}

%\vspace{10mm}%
{\subsubsectit{\hfill Abstract}}\\
{The Center of Interdisciplinary Studies in Cracow has a very rich tradition that has been studied by many and recently by Kamil Trombik. The very difficult period for the Church and for philosophy during the materialistic Marxist ideology was an opportunity for card. K.~Wojtyła to outline a new mode of dialog between science and religion. The future Center, organized by Michał Heller and Józef Życiński not only captured this idea but transformed it into an academic institution centered on the  idea of philosophy in science, which was also developed at the Vatican Observatory by W.G.~Stoeger et al. Trombik's book is an excellent study of the historical context and initial years of activity of the Center and its development.}\par%
\vspace{2mm}%
{\subsubsectit{\hfill Keywords}}\\%
{Center of Interdisciplinary Studies in Cracow, philosophy in science, Karol Wojtyła, John Paul II, Michał Heller, Józef Życiński.}%

\selectlanguage{polish}

\end{newrevplenv}