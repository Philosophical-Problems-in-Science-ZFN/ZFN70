\begin{artplenv}{Dominika Oramus}
	{Nowe światy literackie: literaturoznawstwo współczesne a~nauki ścisłe}
	{Nowe światy literackie: literaturoznawstwo współczesne a~nauki\ldots}
	{Nowe światy literackie: literaturoznawstwo współczesne a~nauki\\ścisłe}
	{University of Warsaw}
	{New literary worlds. Contemporary literature studies and science}
	{Since 1959, when C.P. Snow delivered his seminal lecture \textit{The Two Cultures} on the lack of understanding between scholars working in the humanities and their colleagues from science departments, the gap between the two groups has been one of the most notorious clichés of contemporary Western culture. The aim of this article is to show that this seemingly insurmountable abyss between sciences and the humanities that was brought to the forefront during the mid-20\textsuperscript{th} century is slowly receding into history. Literature studies today is heavily indebted to modern science. Biology (especially evolutionary biology), physics (especially quantum physics), and ecology (especially the Anthropocene studies) are among the most important subjects scholars of literature have to take into account. In order to prove this point I~shortly describe literary genres which introduce modern science to the readers: science fiction, cyberpunk, solarpunk, lablit, quantum fiction, and cli-fi. I~also refer to the newly-emerged schools of criticism-science fiction studies, ecocriticism and evocriticism-to show how scholars discuss these texts within the framework of the humanities. Additionally, I~give a~sample discussion of one of the cli-fi's classics, J.G. Ballard’s \textit{The Drowned World} and also shortly discuss two science fiction novels concerned with the civilisational conflict between science and humanities: Stanislaw Lem's \textit{His Master's Voice} and Margaret Atwood's \textit{Oryx and Crake}.}
	{science fiction, climate fiction, quantum fiction, cyberpunk, solarpunk, science versus humanities, anthropocene, James Graham Ballard, Stanisław Lem, Margaret Atwood.}




\lettrine[loversize=0.13,lines=2,lraise=-0.01,nindent=0em,findent=0.2pt]%
{L}{}iteraturoznawstwo pojmowane jako studia nad rozmaitego typu dyskursami pozwala śledzić fascynacje i~obawy związane z~doświadczeniem życia we współczesności oraz nie wyklucza z~badań tekstów kultury odbiegających stylistycznie czy formalnie od wielkich dzieł z~kanonu literatury narodowej ani światowej. Takie poszerzenie literaturoznawczego pola badawczego skutkuje ważnym spostrzeżeniem: współczesna humanistyka jest pod ogromnym wpływem nauk ścisłych. Biologia (zwłaszcza teoria ewolucji) i~ekologia (zwłaszcza teoria antropocenu), fizyka (zwłaszcza kwantowa) to najważniejsze obok rewolucji informatycznej naukowe konteksty życia współczesnych ludzi. Znajdują one odbicie w~powstających od drugiej połowy XX wieku tekstach literackich oraz szkołach krytycznych analizujących te teksty. Poniższy krótki przegląd kilku wybranych podgatunków literackich inspirowanych szeroko pojętym przyrodoznawstwem pokazuje w~miniaturze, w~jaki sposób nauki ścisłe inspirują twórców literatury popularnej i~jak podział na literaturę popularną i~wysoką odchodzi do przeszłości.

\section*{\textit{Science fiction} (sci-fi), cyberpunk, solarpunk}
W~ostatniej dekadzie XX wieku część anglosaskich środowisk akademickich uznało \textit{science fiction} za przedmiot badań godny uwagi. Pod koniec ubiegłego wieku powstała słynna \textit{The Encyklopedia of Science Fiction} Johna Clute’a i~Petera Nichollsa, w~której znajdziemy m.in. hasło ,,Definitions of SF''
%\label{ref:RNDMBMddO8e91}(Clute, Nicholls, 1993, s.~311–314).
\parencite[][s.~311–314]{clute_encyclopedia_1993}. %
 Ukazuje ono, jak w~kolejnych dekadach XX wieku zmieniało się pojmowanie terminu fantastyka naukowa -- co pośrednio pokazuje zmiany recepcji odkryć naukowych w~kulturze popularnej. Kingsley Amis, autor \textit{New Maps of Hell} 
%\label{ref:RNDWDxvVuS5Gt}(1960),
\parencite*[][]{amis_new_1960}, %
 jednej z~pierwszych brytyjskich książek o~tym gatunku twierdzi, że \textit{science fiction} nie może operować ekstrapolacją, a~jej zadaniem nie jest przewidywanie na podstawie już istniejących przesłanek dalszych losów naszej cywilizacji\footnote{“Science Fiction is that class of prose narrative treating of a~situation that could not arise in the world we know, but which is hypothesized on the basis of some innovation in science or technology, or pseudo-technology, whether human or extra-terrestrial in origin'' 
%\label{ref:RNDS0xm7Rj1nn}(Amis, 1960, s.~14).
\parencite[][s.~14]{amis_new_1960}.%
}. Rozdział poświęcony definicji fantastyki naukowej został przetłumaczony na język polski i~ukazał się w~antologii \textit{Spór o~SF}:

\myquote{
\textit{Science fiction} jest klasą narracji prozatorskiej przedstawiającą sytuacje, które nie mogłyby się zdarzyć w~świecie, jaki znamy, ale są hipotetycznie postulowane na podstawie jakiegoś odkrycia naukowego lub technicznego, albo pseudonaukowego lub pseudotechnicznego, pochodzenia ziemskiego lub pozaziemskiego
%\label{ref:RNDq8f8JPiOiw}(Handke i~in., 1989, s.~13).
\parencite[][s.~13]{handke_spor_1989}.%
}

James Gunn -- amerykański nestor \textit{science fiction studies} i~twórca najważniejszych antologii gatunku oraz podwalin teorii go opisujących uważa że:

\myquote{
\textit{Science fiction} jest gałęzią literatury, która zajmuje się opisem wpływu zmian na ludzi mieszkających w~świecie realnym, obserwowanym na tle przeszłości, przyszłości lub też z~odległości. Często opowiada o~zmianach zachodzących w~świecie nauki lub techniki, a~zazwyczaj dotyczy spraw, których znaczenie względne jest większe niż znaczenie jednostki czy społeczeństwa; niebezpieczeństwo często tu zagraża cywilizacji lub całej rasie
%\label{ref:RNDv7QICYxF5c}(Gunn, 1985, s.~9).
\parencite[][s.~9]{gunn_droga_1985}.%
}

Jeden z~najważniejszych brytyjskich pisarzy \textit{science fiction} tworzących w~wieku XX, Brian Aldiss, pojmował \textit{science fiction} niezwykle szeroko:

\myquote{
\textit{Science fiction} jest poszukiwaniem definicji człowieka i~jego miejsca we wszechświecie; definicji, która byłaby zgodna z~zaawansowanym, lecz nieuporządkowanym stanem naszej wiedzy (\textit{science}), zwykle prezentowanym w~formie gotyckiej lub postgotyckiej
%\label{ref:RNDdVL9SKw9xH}(Handke i~in., 1989, s.~19).
\parencite[][s.~19]{handke_spor_1989}.%
}

Definicja ta podkreśla, że to dzięki fantastyce możemy opisać doświadczenie psychiczne, jakim jest życie w~ramach cywilizacji technicznej, która istnieje dzięki nauce tak zaawansowanej, że niezrozumiałej dla pojedynczych ludzi. Myśl Aldissa rozwinęła Judith Merril, propagatorka terminu \textit{speculative fiction}:

\myquote{
SF (\textit{science fiction}, ale także \textit{speculative fiction}) to literatura, której celem jest eksplorowanie i~odkrywanie natury rzeczywistości, człowieka i~kosmosu za pomocą technik takich jak: ekstrapolacja, projekcja, tworzenie analogii oraz pisarskie testowanie hipotez... używam terminu \textit{speculative fiction} aby opisać literaturę wykorzystującą tradycję ,,metody naukowej'' (obserwacje, hipotezy i~eksperymenty), aby stworzyć i~przebadać przybliżenia rzeczywistości, które powstają w~oparciu o~znane wszystkim fakty zmodyfikowane przez wprowadzenie zmian. Zmiany te mogą być tworem wyobraźni lub wynikać z~wynalazków, a~dzięki nim powstaje nowe środowisko, w~którym funkcjonują bohaterowie. To obserwacje ich reakcji i~postrzegania mają odkryć nowe prawdy -- o~wynalazkach, o~bohaterach lub o~jednym i~drugim
%\label{ref:RNDQqwiZSWj7k}(Merril, 1966, s.~35–36, tłum D.O.).
\parencite[][tłum. D.O.]{merril_what_1966}.%
}

Definicja Merril pozwala widzieć fantastykę jako narzędzie poznania współczesnego świata oraz bliskiej przyszłości -- dzięki koncentracji na stechnicyzowanym otoczeniu stwarza swoiste laboratorium wyobraźni, w~którym z~kolei umieszcza bohaterów. Ich w~dużej mierze podświadome reakcje na technologię są treścią utworów. Zarówno Merril, jak i~Aldiss, Gunn oraz Amis to pisarze i~krytycy dwudziestowieczni, sprzed doby Internetu -- rozumieli oni \textit{science fiction} jako platformę dyskusji o~technice w~przyszłości. Rewolucja informatyczna, jaka dokonała się w~chwili wprowadzenia komputerów do życia społecznego sprawiła, że \textit{science fiction} pozwoliło opisywać nową teraźniejszość.

Powstanie Internetu zrewolucjonizowało kulturę, wyobraźnię, rozrywkę, rynek książkowy, styl życia ludzkości -- to w~tym momencie stechnicyzowana przyszłość rodem z~\textit{science fiction} stała się ludzką teraźniejszością. Rozwój technik komputerowych spowodował zdominowanie kultury wizualnej przez narastającą falę obrazów-kopii, których oryginał nie istnieje, gdyż są one wygenerowaną technicznie iluzją. Zajęły one ważne miejsce w~świadomości społecznej i~stały się łatwo rozpoznawalnymi ikonami. Choć nigdy nie istniały fizycznie, jak np. bohaterowie filmów fantastycznych, stanowią element uniwersalnego kodu zrozumiały w~całym cywilizowanym świecie.

Jean Baudrillard określił takie ,,kopie bez oryginału'' jako symulakry
%\label{ref:RNDalZAolCx06}(Baudrillard, 2005),
\parencite[][]{baudrillard_symulakry_2005}, %
 które nie są ograniczone do jednego konkretnego tekstu, filmu czy symulacji komputerowej, krążą w~sieci pomnażając liczbę kopii, odbić i~iluzji\footnote{Rozwój technik filmowych powoduje narastającą falę symulakr: gnomy, smoki i~olbrzymy z~filmów o~Harrym Potterze, cała menażeria fantastycznych istot z~\textit{Władcy Pierścieni} i~wielu innych kasowych przebojów filmowych, zajmują ważne miejsce w~świadomości społecznej. Są łatwo rozpoznawalnymi ikonami i~choć nigdy nie istniały fizycznie, stanowią element uniwersalnego kodu zrozumiały w~całym cywilizowanym świecie. Powstają o~nich gry, produkowane są gadżety, a~na odwiedzanych przez rzesze internautów stronach internetowych ,,zawieszane'' są kolejne fikcje -- obrazkowe lub pisane teksty ,,fan-fic'', czyli fikcji tworzonych przez fanów na motywach ulubionych dzieł filmowych czy literackich. Dla wielu mieszkańców postmodernistycznego krajobrazu symulakry te są prawdziwe -- za ich pośrednictwem komunikują się z~innymi internautami, których fizycznie nigdy nie spotkali i~nie spotkają. W~eseju \textit{Simulacra and Science Fiction} Baudrillard opisuje między innymi fikcję generowaną na wielką skalę (na przykład przez współczesne media) niszcząc materialną ,,prawdę'', którą niby przedstawia. Baudrillard sięga do przykładu opowiadania Borgesa, w~którym wykonana w~skali 1:1 mapa pokryła i~w rezultacie zniszczyła przedstawianą krainę. Żyjemy w~czasach implozji znaczeń; wedle Baudrillarda po czasach ekspansji i~towarzyszących jej romantycznych wizji heroicznego podboju kosmosu (w rodzaju \textit{Kronik marsjańskich} Raya Bradbury’ego) nadszedł czas cywilizacyjnego zapadania się w~siebie, zamykania oczu i~tworzenia nibyprzestrzeni zaludnionej przez iluzje -- jak w~niekończących się cyklach halucynacji, które przeżywają bohaterowie Philipa K. Dicka lub w~cyberprzestrzeni z~powieści cyberpunkowych.}. Dla wielu mieszkańców postmodernistycznego krajobrazu symulakry te są prawdziwe -- za ich pośrednictwem komunikują się z~innymi internautami, których fizycznie nigdy nie spotkali. Wirtualny świat wymagał stworzenia całego nowego języka do opisania doświadczeń pracy, komunikacji i~rozrywki w~sieci -- i~to właśnie odłam ,,twardej'' (czyli inspirowanej wprost nowymi odkryciami naukowymi) \textit{science fiction} nazwany \textit{cyberpunk} dał współczesnej kulturze aparat pojęciowy, by mówić o~tym całkiem nowym środowisku. Teksty i~filmy cyberpunkowe pokazują bliską przyszłość cywilizacji Zachodu -- społeczeństwa informatycznego, globalnego, gdzie dzięki rozwiniętej technice symulakry zastępują rzeczywistość. Jednocześnie jednak świat ten boryka się z~realnymi skażeniami, epidemiami i~chorobami cywilizacyjnymi.

Znakiem rozpoznawczym cyberpunku są komputery, sieć, Internet, fascynacja współczesną elektroniką, a~zwłaszcza jej gadżetami. Wiele powszechnie używanych dziś słów (rzeczywistość wirtualna -- VR, haker, cyberprzestrzeń, sieć) ma swoje źródło w~tekstach cyberpunkowych. Literatura ta odzwierciedla, ale i~kształtuje rzeczywistość, oddaje wrażenie życia we współczesnym świecie zdominowanym przez Internet i~technologiczne gadżety, symulakry, symulacje i~animacje komputerowe. Cyberpunk jest refleksją nad kulturowymi konsekwencjami powszechnego stosowania bardzo zaawansowanej techniki -- między innymi wiele uwagi poświęca cyborgizacji, zatarciu się różnic między człowiekiem a~maszyną (np. synapsy ludzkiego układu nerwowego kompatybilne z~końcówkami komputera i~chipami). Cyberpunk jest odpowiedzialny za rozpropagowanie terminologii hakerów oraz zainteresowanie ,,stykiem'' człowieka i~maszyny: implantami, VR, domenami internetowymi, AI (sztuczną inteligencją). Wykreował także stereotyp bohatera czasów wielkomiejskiej przyszłości, żyjącego wedle zasad obowiązujących w~skażonej technologicznej dżungli, lecz mimo to posługujących się własnym kodem honorowym.

Cyberpunk był literaturą \textit{science fiction} przełomu mileniów, która dość szybko przekształciła się w~całą gamę podgatunków potomnych -- najsłynniejszą jest steampunk (przeniesienie estetyki cyberpunkowej w~czasy wiktoriańskie), a~najważniejszą z~punktu widzenia wpływu nauk ścisłych na rozwój literatury -- rodzący się dopiero solarpunk.

Steampunk można więc traktować jako nostalgiczną impresję -- przeniesienie cyberpunkowej estetyki w~fabuły rodem z~historii alternatywnej. Jego związek z~naukami ścisłymi jest bardzo nikły -- choć można mówić o~inspiracjach historią techniki. W~takim ujęciu staje się on pokrewny książkom z~podgatunku powieści neowiktoriańskiej. Pisarstwo takie stało się modne w~okresie, gdy Anglia obchodziła stulecie Złotego Jubileuszu królowej Wiktorii i~przedstawia jej czasy z~pewną dozą nostalgii, stosując zazwyczaj postmodernistyczne techniki narracji, z~których najważniejsza to narracja równoległa. W~wielu powieściach neowiktoriańskich dwa plany czasowe, dziewiętnastowieczny i~współczesny przeplatają się -- jest tak np. w~najsłynniejszej z~nich, \textit{Opętaniu} Antonii Susan Byatt, a~zabieg ten występuje również w~powieściach steampunkowych\footnote{O~\textit{steampunku} i~jego związkach z~literaturą wiktoriańską z~jednej, a \textit{cyberpunkiem} z~drugiej strony Piechota pisze w~\textit{Między utopią a~melancholią. W~kręgu nowoczesnej i~ponowoczesnej literatury fantastycznej}
%\label{ref:RNDW7O3HhYJx6}(2015, s.~87–90):
\parencite*[][s.~87–90]{piechota_miedzy_2015}: %
 ,,To specyficzna mutacja cyberpunku, który w~dynamiczny sposób poszerzył swoje oddziaływania na inne pola kultury, jak film, gry komputerowe, a~nawet design. Dwa człony nazwy (\textit{steam} oraz \textit{punk}) stają się wyznacznikami zainteresowań pisarzy tej literatury. \textit{Steam} (z ang. para, energia parowa) odsyła nas do epoki wiktoriańskiej, wykorzystującej w~przemyśle parę mechaniczną. Autorzy spoglądają na XIX-wieczną rzeczywistość z~pewną nostalgią oraz tęsknotą. Drugi człon -- punk -- odwołuje się do tradycji buntu przeciwko establishmentowi oraz wielkim korporacjom. Twórcy \textit{steampunkowi} tęsknią za czasami, w~których kupno określonego produktu jednoznacznie wiązało się z~jego trwałością oraz gwarantowaną jakością''.}.

Solarpunk jest dla niniejszych rozważań znacznie istotniejszy: pod koniec drugiej dekady XXI wieku stał się -- w~swojej odmianie dystopijnej -- główną płaszczyzną dywagacji na temat roli nauk ścisłych w~zagładzie ekologicznej naszej planety. Jednocześnie jednak część utworów solarpunkowych poświęconych jest nie katastrofalnej teraźniejszości, ale lepszej przyszłości. Wpisując się w~tradycję utopijną, ich twórcy starają się pokazać, że mądrze używana zaawansowana nauka (tak inżynieria genetyczna, jak i~fizyka) może pomóc przezwyciężyć kryzys ekologiczny i~dać asumpt do stworzenia cywilizacji przyszłości, której bliskie będą ideały zrównoważenia cywilizacji i~natury. Solarpunk jest nie tylko podgatunkiem literackim czy artystycznym, ale również ruchem społecznym, zrzesza za pośrednictwem domen internetowych młodych ludzi szczerze zaniepokojonych zanieczyszczeniem środowiska i~modelem konsumpcji, który zdominował współczesny świat. Ludzie ci niekiedy piszą blogi, a~niekiedy utwory literackie, upatrując w~tej estetyce i~ideologii szans na lepszą przyszłość. Na jednej z~takich stron czytamy:

\myquote{
Solarpunk to odmiana fikcji spekulatywnej, prąd artystyczny, moda i~ruch społeczny poświęcony odpowiedziom na pytanie ,,jak wyglądać będzie cywilizacja nie niszcząca środowiska i~jak do niej doprowadzić?'' Estetyka solarpunkowa łączy to, co praktyczne, z~tym, co piękne, i~to, co dobrze zaprojektowane, z~tym, co dzikie i~ekologiczne: jasne i~kolorowe z~porządnym i~stonowanym. Solarpunk jest czasem optymistyczny i~utopijny, a~czasem skoncentrowany na trudach walki, by utopię osiągnąć -- nie jest jednak dystopijny. Ponieważ nasz świat gnębią nieszczęścia, potrzeba nam rad, nie ostrzeżeń. Rad jak żyć wygodnie bez korzystania z~paliw kopalnych, jak sprawiedliwie radzić sobie z~niedoborami i~równo dzielić nadwyżki, jak być \textit{fair} dla siebie nawzajem i~planety, na której żyjemy. Solarpunk to wizja przyszłości, przemyślana prowokacja intelektualna i~możliwy do wdrożenia styl życia
%\label{ref:RNDhdbBMjPmlm}(Solarpunk (definicja),
\parencite[][tłum. D.O.]{noauthor_solarpunk_2021}.
}

To dzięki naukom ścisłym społeczności opisywane w~utworach solarpunkowych są w~stanie osiągnąć równowagę, korzystać z~natury tak, by jej nie niszczyć. Zrozumienie zasad funkcjonowania ekosystemów, zaawansowana genetyka, mistrzowskie wykorzystanie sztucznej inteligencji, a~nade wszystko rozwinięta energetyka eksperymentalna zakładająca uzyskiwanie taniej energii z~odnawialnych źródeł to filary utopijnej przyszłości.

\section*{\textit{Climate fiction} (cli-fi), ekokrytyka, evokrytyka}
Solarpunk jest niekiedy postrzegany jako awangardowy i~utopijny odłam cli-fi, czyli powieści o~zmianach klimatu, nowego podgatunku literackiego, który zdefiniował pisarz i~krytyk Dan Bloom. Nazwa ta, choć powstała w~latach 90. ubiegłego stulecia, obejmuje utwory powstające na przestrzeni drugiej połowy XX wieku (część z~nich powstała zanim narodziła się współczesna świadomość ekologiczna, terminy takie jak ,,globalne ocieplenie'' czy antropocen). Bloom twierdzi, że cli-fi (termin utworzony z~pierwszych sylab angielskich słów ,,klimat'' i~,,fikcja'') to gatunek siostrzany powieści katastroficznej i~\textit{science fiction} (nazwa cli-fi utworzona jest na wzór sci-fi). Można mówić o~jego odrębności, gdyż koncentruje się wyłącznie na zmianach klimatu: ich przyczynach, skutkach, naukowych wytłumaczeniach.

Za powieść założycielską tego podgatunku wielu krytyków uważa napisany jeszcze w~latach sześćdziesiątych XX w. \textit{Zatopiony świat} J.G. Ballarda. Jim Clarke w~pierwszym zdaniu eseju ,,Reading Climate Change in J.G. Ballard'' pisze ,,cli-fi powstało zanim jeszcze klimat zaczął się zmieniać''
%\label{ref:RND11lKuQbk33}(Clarke, 2013, s.~7, tłum. D.O.).
\parencite[][tłum. D.O.]{clarke_reading_2013}. %
 Rachele Dini podejmuje temat zmian klimatycznych we wczesnych utworach Ballarda w~artykule ,,‘Resurrected from its Own Sewers’: Waste, Landscape, and the Environment in J.G. Ballard's 1960s Climate Fiction'' 
%\label{ref:RNDyjU37CjeOx}(Dini, 2021).
\parencite[][]{dini_resurrected_2021}. %
 Adrian Tait zaś napisał esej ,,Nature Reclaims Her Own: J.G. Ballard's \textit{The Drowned World}'' 
%\label{ref:RNDbg28YGH4xv}(Tait, 2016)
\parencite[][]{mcgrath_nature_2016}, %
 w~którym przyrównuje fragmenty \textit{Zatopionego świata} do cytatów ze słynnej, napisanej w~tym samym okresie \textit{Cichej wiosny} Rachel Carson oraz całkiem współczesnych biuletynów publikowanych przez Intergovernmental Panel for Climate Change 
%\label{ref:RNDcnU9zHVRle}(Tait, 2016, s.~158).
\parencite[][s.~158]{mcgrath_nature_2016}. %
 Co ciekawe, w~zachowaniu bohaterów tej powieści można zaobserwować psychiczne symptomy charakteryzujące dzisiejszych ludzi cierpiących na depresję klimatyczną, czyli tzw. ,,\textit{eco-anxiety}'' 
%\label{ref:RNDjkGQyqPREd}(Gifford, Gifford, 2016).
\parencite[][]{gifford_largely_2016}.%


\textit{Zatopiony świat} to pozornie konwencjonalna i~prosta historyjka o~globalnej zagładzie i~smutnym losie niedobitków rasy ludzkiej, jednak katastrofa i~jej skutki to tylko pretekst. Seria burz na Słońcu doprowadziła do katastrofalnego ocieplenia Ziemi, roztopione lody zalały wielkie połacie znanych nam dzisiaj lądów, a~wzmożona radiacja doprowadziła do mutacji roślin i~zwierząt. Zmutowane organizmy na kartach książki Ballarda przypominają przedstawicieli pierwotnych gatunków stworzeń z~najdawniejszych geologicznych epok. Stopniowo, wraz ze zmianami klimatu i~linii brzegowych, świat upodabnia się do tego sprzed milionów lat i~cofa się ewolucja świata ożywionego:

\myquote{
Wskutek wzrostu poziomu temperatury, wilgotności i~promieniowania flora i~fauna naszej planety zaczyna przyjmować formy, które występowały na Ziemi, kiedy panowały tu ostatnio takie warunki, czyli mniej więcej w~epoce triasu. [...] Wszędzie występowała ta sama lawina przemian, ciągnąca życie wstecz, w~przeszłość, i~to tak daleko, że tych kilka organizmów złożonych, którym udało się nie stracić równowagi na zboczu metamorfoz, stanowi teraz anomalię. Mam tu na myśli grupkę płazów, ptaki i~człowieka [...] zlekceważyliśmy najważniejszą istotę na naszej planecie
%\label{ref:RNDUbflWU2T0U}(Ballard, 1998, s.~63).
\parencite[][s.~63]{ballard_zatopiony_1998}.%
}

Słowa te wypowiada jeden z~bohaterów, który ma własne teorie na temat regresu rasy ludzkiej. Nie sugeruje on powolnego przekształcania się ludzi w~troglodytów, lecz udowadnia, że ludzkość stała się anachronizmem, gdyż zmuszona jest żyć w~epoce znacznie wcześniejszej niż jej właściwy biologiczny czas. Wierzy także, że mimo ludzkiej psychiki i~anatomii naczelnych gdzieś głęboko na poziomie komórek zachowaliśmy wspomnienia z~najwcześniejszych etapów ewolucji. ,,Pamiętamy'' naszych prymitywnych przodków. Ta tendencja do ruchu wstecz jest wrodzona, głęboko w~duszy pozostały ślady przejścia od jednokomórkowców do \textit{Homo sapiens sapiens}. W~teorii tej pobrzmiewają freudowskie echa:

\myquote{
Pomyśl jak często większość z~nas miewa ostatnio poczucie \textit{déjà vu}, poczucie, że wszystko to już kiedyś widzieliśmy, a~nawet że aż za dobrze pamiętamy te bagna i~laguny. […] Wszędzie w~przyrodzie widać dowody na istnienie wrodzonych mechanizmów, wyzwalających siły, które spoczywały w~uśpieniu przez tysiące pokoleń, lecz których moc pozostała do dziś nie naruszona
%\label{ref:RNDVD84PP0Ypw}(Ballard, 1998, s.~65).
\parencite[][s.~65]{ballard_zatopiony_1998}.%
}

Bohaterowie Ballarda są wyobcowani i~zamknięci w~sobie -- nie są w~stanie myśleć o~przeszłości i~dojrzeć teraźniejszość. Najlepszą definicję dziwnego szaleństwa, jakie ich ogarnia, daje wymyślona przez Ballarda nowa nauka, neuronika:

\myquote{
Jeśli chcesz, możesz to sobie nazwać Psychologią Totalnych Ekwiwalentów albo krótko: neuroniką, i~odrzucić ją jako metabiologiczne urojenie. Ja jednak jestem przekonany, że tak jak posuwamy się w~przeszłość w~czasie geograficznym, tak wkraczamy również w~korytarz owodni, cofając się w~rdzeniowym i~archeopsychicznym czasie, przypominając sobie podświadomie krajobrazy minionych epok, ich rzeźbę geologiczną i~im tylko właściwą faunę i~florę, tak nam znajomą, jak podróżnikowi korzystającemu z~Wellsowskiego wehikułu czasu
%\label{ref:RND0MxYkuhiHK}(Ballard, 1998, s.~67).
\parencite[][s.~67]{ballard_zatopiony_1998}.%
}

Chociaż bohaterowie Ballarda żartują z~psychoanalizy, wyśmiewają myśl, że rzeczywistość zaczyna przypominać test Rorschacha, \textit{Zatopiony świat} nosi wyraźne wpływy freudowskiej teorii dominacji popędu śmierci nad zasadą przyjemności, Thanatosa nad Erosem. W~późnym eseju \textit{Poza zasadą przyjemności}
%\label{ref:RNDhPXIaCiL7C}(Freud, 1994)
 %tłum. pol. 1994
Freud 
\parencite*[][]{freud_poza_1994} %
odwołuje swoje wcześniejsze twierdzenie, że seksualność i~instynkt samozachowawczy to podstawowe i~biologicznie najstarsze siły. Przyznaje, że teoria zasady przyjemności i~zasady rzeczywistości, które wzajemnie się modyfikują by zapewnić nam maksimum satysfakcji, a~jednocześnie ustrzec nas przed zgubnymi skutkami folgowania wszelkim popędom, nie tłumaczy w~pełni zachowań ludzkich. Instynktowne potrzeby kochania, rozmnażania się i~samoobrony są powierzchowne i~ewolucyjnie najmłodsze, pod nimi ukryty jest prastary instynkt śmierci pchający rasę ludzką ku samozagładzie.

Powyższe odczytanie \textit{Zatopionego świata} pozwala zrozumieć, czemu powieść ta, pół wieku po pierwszym wydaniu, została uznana za klasyczny tekst cli-fi. Ballard opisuje stany psychiczne bliskie odkryciom współczesnych badaczy ,,\textit{eco-anxiety}''. Paul Huebener, którego książki opisują kryzys ekologiczny i~jego echa we współczesnej kulturze, zwraca uwagę na fakt, że depresja klimatyczna pojawia się wtedy, gdy ludzie pojmują, iż zmiany klimatu są nieodwracalne i~muszą doprowadzić do zagłady Ziemi
%\label{ref:RNDhWovHARIC7}(Huebener, 2020).
\parencite[][]{huebener_natures_2020}. %
 Bezsilność jaka ich ogarnia sprawia, że są psychicznie nastawieni na najgorsze. W~artykule \textit{A~Visit to the Climate Anxiety Doctor} Stephanne Taylor -- dziennikarka specjalizująca się w~zagadnieniach klimatycznych -- tłumaczy to zjawisko:

\myquote{
Zmiany klimatu wpływają na ludzką psychikę zarówno kiedy przybierają postać gwałtownych, katastrofalnych zjawisk, jak i~długofalowych, stopniowych procesów. Psychologowie odkryli, że kataklizmy skutkują innymi problemami psychicznymi niż zmiany stopniowe. Gwałtowne katastrofy powodują u~ofiar traumę, zmiany stopniowe -- poczucie beznadziejności, fatalizm i~wreszcie depresję. Wszelkie takie zaburzenia psychiczne mają podobne podłoże: strach o~coraz bardziej niepewną przyszłość i~świadomość, że we współczesnym świecie nie podejmuje się odpowiednich kroków, by zmianom zaradzić
%\label{ref:RND5fbW0fL0wi}(Taylor, 2016, tłum. D.O.).
\parencite[][tłum. D.O.]{taylor_visit_2016}.%
}

Taylor zwraca uwagę na podświadomy lęk o~przyszłość i~nagłe ataki paniki i~smutku, jakie odczuwają pacjenci cierpiący na depresję klimatyczną. Mówią oni o~,,nagłym poczuciu dojmującej beznadziejności, jaka opanowuje ich gdy słyszą o~zagładzie raf koralowych''
i~o~,,dziwnym, przemożnym lęku''
%\label{ref:RNDmDxkISCo4C}(Taylor, 2016, tłum. D.O.),
\parencite[][tłum. D.O.]{taylor_visit_2016}, %
 przypominającym stany psychiczne opisywane przez Ballarda. U~ofiar depresji klimatycznej ,,brak jakiejkolwiek nadziei''
% (Hamilton)
\parencite{hamilton_climate_2016}
 sprawia, że zaburzona zostaje percepcja czasu. Michelle Bastian, badaczka tego zjawiska, stwierdza, że: ,,standardowy czas mierzony zegarami nie jest adekwatny w~kontekście zmian klimatu'' 
%\label{ref:RNDEFbUAtIDnL}(Bastian, 2012, s.~39).
\parencite[][s.~39]{bastian_fatally_2012}. %
 W~naszej epoce nie możemy już wyobrażać sobie czasu jako drogi prowadzącej od teraz ku nieskończonej przyszłości, ale raczej stosujemy model odliczania -- ile jeszcze czasu na tej planecie nam zostało.

Cli-fi pokazuje bohaterów zmagających się z~psychicznymi skutkami depresji klimatycznej, przekonanych, że są świadkami ostatnich dni ludzkości na planecie\footnote{,,Widzę analogię do strachu przed zagładą, jaki panował w~latach 50. w~związku z~zagrożeniem wojną atomową. Zwykli ludzie wtedy również odczuwali silny niepokój i~wiedzieli, że jesteśmy zagrożeni anihilacją''
%\label{ref:RNDdtSt3Ufdqe}(Hamilton, 2016),
\parencite[][]{hamilton_climate_2016}, %
 twierdzi Robert Gifford, profesor psychologii z~University of Victoria, a~także członek American Psychological Association. Organizacja ta uznała oficjalnie zagrożenie płynące z~,,już występującego i~prognozowanego na przyszłość wpływu zmian klimatu na psychikę''
% (Hamilton)
\parencite[][]{hamilton_climate_2016}.
 }. Michelle Bastian przypomina w~kontekście zmian klimatycznych ,,zegar zagłady'', stworzony przez naukowców w~roku 1947. Wtedy to użyto ,,powszechnie zrozumiałej konwencji -- północ oznacza koniec czasu'' 
%\label{ref:RND6Nzfef3oPG}(Bastian, 2012, s.~39),
\parencite[][s.~39]{bastian_fatally_2012}, %
 by graficznie unaocznić zagrożenie wojną atomową. Bastian zwraca uwagę na fakt, że wskazówka minutowa nie pokazuje w~tym wypadku liczby minut brakujących do pełnej godziny, ale prawdopodobieństwo końca świata. Na przestrzeni ostatniego półwiecza zagrożenie wojną atomową malało, ale jednocześnie wzrastało niebezpieczeństwo nieodwracalnego zniszczenia biosfery, spowodowanego przez cywilizację ludzką. W~styczniu 2012 r.~naukowcy związani z~\textit{Bulletin of the Atomic Scientists} przestawili zegar zagłady na za pięć dwunasta, ,,częściowo z~powodu braku działań zapobiegających katastrofie klimatycznej'' 
%\label{ref:RNDHiMs7n9a5k}(Bastian, 2012, s.~39).
\parencite[][s.~39]{bastian_fatally_2012}. %
 Powieść Ballarda antycypuje te strachy, a~raczej przedstawia w~sposób przekonujący reakcje ludzkiej psychiki na świadomość, że czas, który nam pozostał, jest ograniczony.

Zarówno cli-fi, jak solarpunk są więc zarówno prądami literackimi, jak i~wyrazem ambiwalentnego pojmowania roli nauk ścisłych we współczesnym świecie. To rozwinięta dzięki nauce technika doprowadza nas na granice zagłady i~być może ona nas ocali. Pogłębiona refleksja nad cywilizacją i~kulturą jako czynnikami kształtującymi środowisko naturalne znalazła wyraz w~nowym prądzie krytycznym: ekokrytyce. Anna Barcz definiuje ją w~tekście \textit{Przyroda -- bliska czy daleka? Ekokrytyka i~nowe sposoby poetyki odpowiedzialności za przyrodę w~literaturz}e w~sposób następujący:

\myquote{
W~ciągu ostatnich lat, nie tylko pod wpływem zainteresowania nauk humanistycznych problemami ochrony środowiska, ale i~z powodu refleksji nad przemianami postaw człowieka wobec otaczającej go przyrody i~związanymi z~tym zjawiskami kulturowymi, pojawiły się nowe perspektywy do badań literackich, które można by zbiorczo określić mianem krytyki ekologicznej lub coraz bardziej rozpowszechniającym się skrótem -- ekokrytyki. Nowo powstałe dyscypliny, których atrakcyjność zatacza coraz większe kręgi, to nie tylko interesująca z~punktu widzenia literaturoznawstwa ekokrytyka, ale szereg innych powiązanych z~nią nurtów, takich jak studia nad zwierzętami (\textit{animal studies}), \textit{nature writing}, krytyka środowiskowa (\textit{environmental criticism}), \textit{ecocinecriticism}, ekopoezja, ekopedagogika, ekofeminizm, czy ekologia polityczna. Są to nowe, dyskursywne propozycje czerpiące swoje źródła w~większości z~krytycznego namysłu nad antropocentryczną tradycją humanistyki, zmierzające w~stronę poetyki różnorodności opartej na świadomości istnienia innych gatunków i~perspektyw myślenia o~świecie, koncentrujące się na relacji ludzi ze zwierzętami i~na obecności lub braku wartości proekologicznych w~tekstach, ale przede wszystkim na reprezentacji środowiska naturalnego w~różnych formach twórczej ekspresji -- na nowych, często eksperymentalnych z~założenia, próbach odniesienia się do przyrody
%\label{ref:RND9FFUavPerd}(Barcz, 2012, s.~59).
\parencite[][s.~59]{barcz_przyroda_2012}. %
}

Powodzenie ekokrytyki jako szkoły krytycznej sprawiło, że wielu literaturoznawców zainteresowało się przyrodoznawstwem -- nie tylko ekologią, ale także teorią ewolucji. Grupa zafascynowanych teorią Darwina badaczy określana czasem mianem evokrytyków
%\label{ref:RNDdNFR23UIRp}(Boyd, 2010, s.~1–2)
\parencite[][s.~1–2]{boyd_origin_2010} %
 wychodzi z~założenia, że ewolucja i~jej wszelkie popularne przeróbki pełnią w~cywilizacji zachodniej rolę ,,teorii wszystkiego'', a~darwinowski klucz (byty rozwijają się na drodze ewolucji, w~procesach doboru naturalnego, przetrwania najlepiej przystosowanych oraz rozprzestrzenienia się najkorzystniejszych cech) może zostać zastosowany do wyjaśniania całego wachlarza zjawisk\footnote{Od doboru i~selekcji naturalnej teorii naukowych (Karl R. Popper), przez naukę o~moralności (Matt Ridley), socjobiologię (Edward O. Wilson), językoznawstwo, wiedzę o~matematyce, badania rozprzestrzeniania się mód i~wynalazków (teoria memów), po socjologię, antropologię i~antropogenezę -- popularyzatorzy nauki używają darwinowskich paradygmatów (w znaczeniu nadanemu słowu paradygmat przez Thomasa Kuhna), by wyjaśniać niespecjalistom postępy rozmaitych dziedzin wiedzy. Dla części popularyzatorów (Edward O. Wilson, Bill Bryson) darwinizm jest kwintesencją naukowego podejścia do przyrody, źródłem ,,kosmogonicznego'' i~ateistycznego mitu o~Stworzeniu, alternatywnego do wersji biblijnej, z~czym łączy się żywa zwłaszcza w~Stanach Zjednoczonych debata pomiędzy darwinistami i~kreacjonistami.}. Brian Boyd, najważniejszy obecnie evokrytyk, we wstępie do zredagowanej przez siebie antologii \textit{On the Origin of Stories} pokazuje, że darwinizm funkcjonuje w~świadomości humanistów jako Teoria przez duże ,,T'', tak jak jeszcze niedawno teoria względności i~mechanika kwantowa (Darwin zastąpił ostatnio Einsteina jako ikona genialnego naukowca). Życie Darwina jest obecnie tematem nie tylko biografii i~fabularyzowanych powieści oraz filmów, ale także całkiem fikcyjnych opowieści (pisanych z~punktu widzenia jego służących czy znajomych) -- samo stało się ,,mityczne'' i~jak klasyczne mity-opowieści funkcjonuje w~rozmaitych wersjach (Irving Stone, Roger MacDonald, Thorvald Steen i~inni). Niektóre momenty i~wydarzenia z~tego życia stały się ogólnie znane, niemal anegdotyczne (historia Alfreda Wallace’a, rejs na okręcie HMS Beagle, pobyt na wyspach Galapagos).

Na gruncie literaturoznawstwa angielskiego można mówić o~całym podgatunku powieści neowiktoriańskiej osnutej wokół fikcyjnych postaci darwinowskich podróżników-naturalistów, przyrodników kolekcjonujących minerały, skamienieliny, próbki fauny i~flory i~wysnuwających na tej podstawie teorie naukowe. Powieści te często wpisują się w~poetykę rozmaitych ważnych nurtów literatury współczesnej (powieść postkolonialna, \textit{historiographic metafiction}, debata o~historii i~pamięci).

Od kilkudziesięciu już lat filozofowie nauki (Karl R. Popper, Thomas Kuhn i~in.) zajmują się logiką i~psychologią odkrycia naukowego, odtwarzając procesy myślowe prowadzące do wielkich przełomów w~naszym pojmowaniu mechanizmów funkcjonowania Wszechświata. Filozofia nauki tworzy swoistą metanarrację, by użyć terminu Jean-Françoisa Lyotarda zdefiniowanego w~\textit{Kondycji ponowoczesnej}, wielką opowieść o~mistrzach, uczniach i~rewolucjonistach wyznaczającą ramy postępu w~nauce. Metanarracja ta pozwala nie tylko pojąć mechanizm odkrycia naukowego, ale także zrozumieć, jak stan wiedzy uczonych danej epoki przekładał się na jej kulturę -- wizję świata funkcjonującą w~danym społeczeństwie.

Inna grupa literaturoznawców uważających przyrodoznawstwo za nieodzowny kontekst literatury współczesnej czerpie inspiracje z~geologii, a~zwłaszcza tzw. teorii antropocenu. Pojęcie to powstało w~początkach obecnego stulecia. Paul Crutzen w~artykule ,,Geology of Mankind''
%\label{ref:RNDDBxf4YyyAT}(2002)
\parencite*[][]{crutzen_geology_2002} %
opublikowanym na łamach \textit{Nature} pisał: ,,wydaje się uzasadnione by używać terminu ‘antropocen’ do opisu współczesnej nam epoki geologicznej, która została zdominowana przez działalność człowieka, a~nastąpiła po ciepłym holocenie trwającym od 10–12 millenniów''. Ten słynny artykuł odbił się szerokim echem również w~humanistyce -- a~zwłaszcza literaturoznawstwie. Melina Pereira Savi w~artykule ,,The Anthropocene (and) (in) the Humanities: Possibilities for Literary Studies'' 
%\label{ref:RNDfiWoo2GmA5}(2017)
\parencite*[][]{savi_anthropocene_2017} %
 przedstawia, jak w~ostatnich kilkunastu latach literaturoznawcy zmieniają swoje podejście do natury i~roli rasy ludzkiej w~kształtowaniu oblicza naszej planety. Jej zdaniem piśmiennictwo to wyróżnia się zastosowaniem wielkiej skali -- historia ludzkości pokazywana jest z~perspektywy odległych tysiącleci, a~stratygrafia staje się jedną z~najczęściej stosowanych metafor -- to pozostawione przez nas zanieczyszczenia i~śmieci będą wyznacznikiem czasów naszego panowania na planecie. Zmiana paradygmatu dotyczy także kultury popularnej: świadomość zagrożenia nieodwracalną zmianą klimatu i~zagładą ziemskich biocenoz widoczna jest już także w~powieściach dla młodzieży i~filmach katastroficznych. Adam Trexler usystematyzował wiadomości o~powieściach doby antropocenu w~ważnej monografii \textit{Anthropocene Fictions} 
%\label{ref:RNDSvqw858maH}(2015).
\parencite*[][]{trexler_anthropocene_2015}. %
 Broni w~niej tezy, że w~dzisiejszych czasach cli-fi poświęcona antropocentrycznej zmianie klimatu jest osobnym i~ważnym podgatunkiem powieści.

\section*{Humy \textit{versus} Fizy i~\textit{quantum fiction}}
W~latach pięćdziesiątych XX wieku słynny powieściopisarz i~fizyk angielski C.P. Snow wygłosił słynny wykład Dwie kultury, poświęcony fizyce i~literaturze, a~szerzej naukom ścisłym i~humanistyce, jako dwóm odrębnym dziedzinom współczesnej cywilizacji. Snow upatrywał przyczyn kryzysu intelektualnego w~kulturze Zachodu w~braku porozumienia właśnie między sferą nauk ścisłych a~humanistyką
%\label{ref:RND3pHJpdGBVs}(Snow, 1999).
\parencite[][]{snow_dwie_1999}. %
 Liczne głosy krytyków wzorujących się na Snowie wskazywały w~kolejnych dziesięcioleciach na pogłębienie się przepaści między dwoma sposobami opisu rzeczywistości, stworzonymi w~odpowiedzi na te zjawiska: analizą zjawisk świata zewnętrznego metodą naukową a~systemową analizą wytworów kultury ludzkiej 
%\label{ref:RND1ty8FXfoXe}(zob. np. Weiner, 2018)
\parencite[zob. np.][]{weiner_postnaturalizm_2018}%
\footnote{Pod koniec ubiegłego tysiąclecia Brockman wydał słynny zbiór esejów popularnonaukowych pod tytułem \textit{Trzecia kultura}. Broni w~niej tezy, że przepowiednie Snowa nie sprawdziły się w~żadnej mierze: ,,Humaniści nadal nie potrafią porozumieć się z~fizykami czy matematykami. Szczęśliwie przedstawiciele nauk ścisłych zaczęli porozumiewać się bezpośrednio z~szeroką publicznością. Tradycyjnie ukształtowane przez humanistów media funkcjonowały dotychczas jakby pionowo -- profesorowie z~wyżyn kierowali swoje słowa ku nizinom, a~dziennikarze nieśmiało zadawali im pytania'' 
%\label{ref:RNDAOlwSAllyg}(Brockman, 1996, s.~17).
\parencite[][s.~17]{brockman_trzecia_1996}. %
 Konsekwencją narodzin tak pojmowanej trzeciej kultury będzie, zdaniem Brockmana, degradacja i~zanik tradycyjnie pojmowanej kultury humanistycznej. Wykształceni znawcy dawnych epok, tradycji i~języków nie są już potrzebni, by tłumaczyć jak działa świat i~cywilizacja i~jakie miejsce zajmuje w~niej człowiek.}. Mówi się też o~braku porozumienia między ich przedstawicielami, a~konflikt ten stał się tematem literackim. Na przestrzeni ostatniego półwiecza podejmowali go m.in. Stanisław Lem i~Margaret Atwood.

\textit{Głos Pana}
%\label{ref:RNDFbDgyFuwRU}(Lem, 2002)
\parencite[][]{lem_glos_2002} %
 to wydana po raz pierwszy w~roku 1968 powieść poświęcona próbom odczytania przez ziemskich naukowców wiadomości zarejestrowanej przez systemy badające przestrzeń kosmiczną. Rozkodowanie komunikatu jest wielkim wyzwaniem: aby mu podołać, powstaje zespół badawczy o~kryptonimem MAVO (\textit{Master’s Voice}), w~którym współpracować mają naukowcy z~bardzo różnych dyscyplin -- zarówno humanistycznych, jak i~ścisłych. Projekt MAVO pod piórem Lema jest alegorycznym obrazem współczesnej wieży z~kości słoniowej -- naukowcy uniwersyteccy żyją w~oderwaniu od świata zewnętrznego, odizolowani na kampusach, jak pracownicy projektu MAVO na pustyni. Projekt jest utajniony, jego wpływ na rzeczywistość społeczną -- żaden, a~jedyne, czemu oddają się profesorowie, to kłótnie między sobą lub uleganie depresji związanej z~własną bezużytecznością. Piotr Hogarth, główny bohater powieści, twierdzi, że ,,pojawił się projekt, by psychoanalitycy i~psychologowie zostali służbowo przeniesieni ze stanowisk badawczych «listu gwiazdowego» na stanowiska lekarzy tych, co listu odczytać nie umieją i~przez to cierpią od stresów'' 
%\label{ref:RNDK7HGDPt1PV}(Lem, 2002, s.~85).
\parencite[][s.~85]{lem_glos_2002}. %
 W~ten więc sposób MAVO odwraca się od świata zewnętrznego i~patrzy do środka, sam na siebie -- siebie bada, sobą jest zainteresowany, tworząc jeden gigantyczny obraz wyczerpania, na jakie cierpi nauka współczesna.

Projekt MAVO jawi się w~rezultacie takich tarć jako dwie odrębne kultury -- niczym z~eseju Snowa:

\myquote{
W~ogóle tarcia między humanistami a~przyrodnikami projektu były na porządku dziennym. Pierwszych zwano u~nas ,,Humami'', drugich zaś -- ,,Fizami'', przy czym słownik wewnętrznego żargonu projektu był wcale bogaty, zarówno nim samym, jak i~formom współżycia obu stronnictw mógłby się z~pożytkiem zainteresować kiedyś socjolog
%\label{ref:RNDLA3hwWPmss}(Lem, 2002, s.~83).
\parencite[][s.~83]{lem_glos_2002}.%
}

Hogarth uważa, że zaproszenie humanistów do pracy przy projekcie było wielkim błędem. Produkowali zbyt wiele ezoterycznych i~niefalsyfikowalnych teorii, pisali tysiące stron opracowań, z~których niewiele wynikało. Generalnie nadmiar zapisanych stron -- relacji, artykułów i~raportów sprawił, że kilka wartościowych spostrzeżeń, które nasunęły się ludziom pracującym w~laboratoriach, zniknęło w~zalewie redundantnych tekstów. Hogarth tłumaczy zachowanie humanistów reakcjami obronnymi, jego zdaniem praca nad rozszyfrowaniem Głosu Pana była dla nich zbyt wyczerpująca emocjonalnie: ,,Biednym naszym Humom robiły się frustracje i~kompleksy, ponieważ byli skazani, w~istocie rzeczy, na kompletne, jakkolwiek przystrojone rozmaitymi pozorami, bezrobocie''
%\label{ref:RNDFGFyDEZBe3}(Lem, 2002, s.~85).
\parencite[][s.~85]{lem_glos_2002}. %
 Humaniści zdają się zbędni -- tak w~projekcie MAVO, jak i~we współczesnej Akademii. Charyzmatyczni profesorowie literatury i~neofilologii zdają sobie sprawę z~własnej nieprzydatności i~zazdroszczą kolegom z~wydziałów ścisłych: tamci zarabiają znacznie więcej i~studiują rzeczy prawdziwe: przyrodę, materię, prawa fizyki. ,,Humy'', przynajmniej z~pozoru, są ,,Fizom'' zbędne.

Kilkadziesiąt lat po Lemie Margaret Atwood w~powieści \textit{Oryks i~Derkacz}
%\label{ref:RNDBSr8vsYaR0}(2004)
\parencite*[][]{atwood_oryks_2004} %
 przedstawiła ten sam konflikt w~realiach doby antropocenu, kiedy to fizyka, chemia i~biologia rządzić będą niepodzielnie, co pośrednio doprowadzi do zagłady rasy ludzkiej. Akcja książki toczy się na ekologicznie zatrutej pustyni, jaką staną się Stany Zjednoczone w~ostatnich latach trwania cywilizacji Zachodu. Prognozowana przez Atwood na pierwszą połowę XXI wieku apokalipsa ekologiczna składa się z~szeregu czynników. Między innymi zauważyć należy skutki pogłębiającego się na przełomie XX i~XXI wieku rozdźwięku między kulturą nauk ścisłych i~kulturą humanistyczną, który w~bliskiej nam przyszłości (a niedawnej przeszłości bohaterów Atwood) doprowadził do całkowitej degeneracji humanistyki i~pozornego zwycięstwa nauki. Jednakże w~wyniku odhumanizowania zachodniego świata, nauki ścisłe błyskawicznie przekształciły się z~teoretycznych na stosowane, a~następnie zdegradowały się do roli technologii wytwarzającej zaawansowane technicznie i~biotechnologicznie towary na sprzedaż. Technologia i~biotechnologia kształtują oblicze świata, a~w końcu prokurują koniec rasy ludzkiej, którą zabijają wyhodowane w~laboratorium wirusy.

Fascynacja literatury fizyką w~wieku XXI nie ogranicza się jednak do opisywania konfliktu humanistyka--nauki ścisłe: powstają również podgatunki powieści postmodernistycznej inspirowane bezpośrednio teoriami naukowymi. Przykładem takiej fascynacji jest fikcja kwantowa, typ powieści, która opisuje ludzkie postrzeganie rzeczywistości w~kategoriach zaczerpniętych z~mechaniki kwantowej. Narracja naśladuje efekt obserwatora: główny bohater to świadomy obserwator świata przedstawionego, a~widząc i~opisując rzeczywistość wpływa na to, jaki JEST świat przedstawiony. Powieści takie mają fabułę rozgrywającą się w~wielu równoległych rzeczywistościach multiświata jednocześnie, a~akcja dzieje się w~tej rzeczywistości, którą akurat obserwuje bohater. To właśnie on przez akt obserwacji ,,ziszcza'' ten wariant świata, który obserwuje -- w~chwili gdy go obserwuje.

Uznawana za podgatunek literatury postmodernistycznej fikcja kwantowa powstała w~roku 1996 dzięki napisanej przez Vannę Bontę książce pod tytułem \textit{Flight. A~Quantum Fiction Novel} (\textit{Ucieczka. Powieść kwantowa})
%\label{ref:RND6mOwizFfps}(1996).
\parencite*[][]{brockman_trzecia_1996}. %
 Przedstawiciele tego podgatunku nie są liczni (Scarlett Thomas, Andrew Crumley, Blake Crouch), często powieści zostają dość arbitralnie (a nawet anachronicznie) zaliczane do fikcji kwantowej jedynie na podstawie tego, że ich narracja odzwierciedla jakieś założenie fizyki posteinsteinowskiej, na przykład względność postrzegania świata. Zazwyczaj w~powieściach takich elementy paranormalne lub fantastyczne pozwalają na skonstruowanie świata rządzonego nie przez prawa fizyki klasycznej, ale kwantowej -- to świadomość, czy też ,,istota duchowa'' z~innego uniwersum okazuje się obserwatorem kształtującym rzeczywistość. Bohaterowie doświadczają zjawisk takich jak synchroniczność (mają świadomość jednoczesnego przebywania w~różnych wariantach rzeczywistości), światy równoległe -- z~alternatywnymi wersjami ziemskiej historii -- i~podróże po multiwersum. Umysł ludzki (lub innych istot inteligentnych) współtworzy świat przedstawiony, a~fabuła polega na ,,załamywaniu funkcji'', tj. urzeczywistnianiu tej możliwości rozwoju akcji, którą postrzega bohater-obserwator. Obiekty makro zachowują się niczym cząstki świata subatomowego.

Bohaterowie Crumey’a czy Croucha zmuszeni są do funkcjonowania w~antyintuicyjnym kwantowym świecie. Narratorka humorystycznego opowiadania Connie Willis ,,W «Rialto»'' zawartego w~książce \textit{Niebieski księżyc}
%\label{ref:RNDd16usYkm8N}(2010)
\parencite*[][]{willis_niebieski_2010} %
 jest fizyczką kwantową, co potęguje efekt komiczny, gdyż na samym początku przyznaje ona, że teoria kwantowa to dla niej czarna magia i~ma ,,czasami niejasne podejrzenia, że dotyczy to wszystkich [...] ale nikt nie chce się do tego przyznać'' 
%\label{ref:RNDY4YpeWOjUy}(Willis, 2010, s.~9).
\parencite[][s.~9]{willis_niebieski_2010}. %
 Tytułowy hotel w~Hollywood jest miejscem corocznego zjazdu fizyków zajmujących się mechaniką kwantową. Na bardzo różnych poziomach organizacji tekstu -- od poszczególnych scenek i~dialogów, aż po całość fabuły -- opowiadanie odzwierciedla paradoksy świata kwantów. W~ogólnym ujęciu konferencja to model świata subatomowego, fizycy i~obsługa to cząstki, a~panujący chaos -- antyintuicyjne zjawiska obserwowane w~świecie subatomowym. Skojarzenia modelu chaosu z~bałaganem i~paradoksami konferencji naukowej (opowiadanie nawiązuje też do konwencji powieści kampusowej) wzmaga efekt komiczny. Recepcjonistka Tiffany, a~właściwie absurdalnie niekompetentna aktorka i~modelka dorabiająca sobie w~hotelu, dodatkowo potęguje chaos. Goście kierowani są do nie swoich pokoi, meldowani w~innych hotelach, klucze są gubione, sesje przesuwane, wszyscy nieustannie czegoś lub kogoś poszukują, a~największym błędem okazuje się postępowanie zgodne z~logiką i~zdrowym rozsądkiem. By czegokolwiek dokonać, należy naśladować paradoksalne modele rodem ze świata subatomowego:

\myquote{
Kim właściwie jest modelka aktorka? Jest modelką \textit{albo} aktorką, czy modelką \textit{i} aktorką. Z~pewnością nie była recepcjonistką. Może elektrony były takimi Tiffany mikroświata; to mogłaby wyjaśnić ich falowo-cząstkową dwoistość. Może w~ogóle nie były elektronami, tylko dorabiały sobie w~charakterze elektronów, żeby zapłacić za kurs stanu singletonowego?
%\label{ref:RNDGqQl7ELreh}(Willis, 2010, s.~30)
\parencite[][s.~30]{willis_niebieski_2010}%
}

Odniesienia do teorii kwantowej zamieniają życie zabłąkanej w~hotelu, poszukującej na konferencji kolegów, sesji, referatów, pokoju czy bufetu fizyczki w~cykl modeli. Próba dotarcia na seminarium z~ominięciem recepcji jest opisana jako model przenikania elektronów przez barierę potencjału, ,,nawet gdy nie miały dostatecznej energii''
%\label{ref:RNDFaDILmOyfE}(Willis, 2010, s.~11).
\parencite[][s.~11]{willis_niebieski_2010}. %
 Ubrania fizyczki ściśnięte w~walizce po długiej tułaczce bagażu ,,wyglądają niemal tak samo jak na poły zdechły kot Schrödingera'' 
%\label{ref:RNDlacoFZtVNB}(Willis, 2010, s.~12).
\parencite[][s.~12]{willis_niebieski_2010}. %
 Ona sama pada ,,ofiarą kolapsu funkcji falowej'' 
%\label{ref:RNDqS4h0p66RV}(Willis, 2010, s.~15),
\parencite[][s.~15]{willis_niebieski_2010}, %
 a~jej znajoma, fizyczka z~Japonii, wpada na pomysł, jak naprawić rzutnik poprzez dostrojenie ,,granic warunków brzegowych niecki fraktalnej'', czyli walnięcie ,,projektora w~bok'' 
%\label{ref:RNDV27AV1H2l9}(Willis, 2010, s.~21).
\parencite[][s.~21]{willis_niebieski_2010}. %
 Opowiadanie przytacza i~nazywa paradoksy będące podstawą modeli -- oszczędza więc humanistom wysiłku wytropienia ich w~podręcznikach. Jest przykładem zabawy intertekstualnej odwracającej stereotyp mądrego fizyka, gdyż właśnie fizykom/naukowcom przydarzają się przygody, które w~powieści kampusowej spotykają raczej profesorów literatury.

Fikcja kwantowa jest najciekawszym, ale nie jedynym podgatunkiem powieści współczesnej pisanej pod wpływem nauk ścisłych. W~2001 roku Jennifer Rohn zaproponowała termin ‘\textit{lablit}’, by zbiorczo określić grupę powieści poświęconą portretowi współczesnego świata laboratoriów -- naukowców, badaczy i~świata zatrudniających ich instytucji. \textit{Lablit} jej zdaniem pozwala zrozumieć, jak nauki ścisłe postrzegane są w~kulturze postmodernizmu -- termin ten jest jednak na tyle niekonkretny, że można go zastosować do części utworów, które należą do cli-fi, sci-fi czy \textit{quantum fiction}.

\section*{Konkluzja}
Zadziwiająca proliferacja nazw podgatunków używanych do opisu tych tekstów oraz szkół krytycznych pokazuje, jak w~ostatnich latach rozmywają się tradycyjne podziały gatunkowe. Obok \textit{science fiction} -- pierwszego historycznie gatunku literackiego zakładającego włączenie nauk ścisłych do sfery zainteresowania humanistów -- zaczęły tworzyć się podgatunki potomne i~hybrydowe: wśród nich cyberpunk, steampunk i~solarpunk. Jednocześnie od lat osiemdziesiątych XX w. rozwijają się w~Stanach Zjednoczonych \textit{science} \textit{fiction studies} -- poświęcona im gałąź literaturoznawstwa i~filmoznawstwa. W~wieku XXI pogarszająca się sytuacja ekologiczna planety sprawiła, że kwestie ochrony środowiska stały się bliskie wielu ludziom, nie tylko specjalistom. Powstawać zaczęła literatura spod znaku tzw. cli-fi (\textit{climate fiction}), jeden z~przedmiotów zainteresowania ekokrytyki -- prądu przeciwstawiającego się podziałowi natura \textit{versus} kultura.

Olbrzymi wpływ fizyki na kształt dzisiejszego świata ma również przełożenie na literaturę. Powodzenie książek popularnonaukowych pokazuje, że zainteresowanie fizyką i~poczucie, że to ona kształtuje obraz współczesnej cywilizacji, jest powszechne także wśród niespecjalistów. Trend ten jest charakterystyczny dla całej współczesnej kultury Zachodu. Powieści o~fizykach, fizyka jako metafora, pisanie powieści w~sposób naśladujący paradoksy nowej fizyki -- to częste motywy we współczesnych utworach. Prócz powstającej od ponad stu lat \textit{science fiction}, wprost inspirowanej odkryciami z~zakresu fizyki, chemii i~astrofizyki, zaczęły powstawać utwory tzw. \textit{quantum fiction} (literatura kwantowa) i~\textit{lablit} (literatura laboratoriów).

Rozważania o~roli nauk ścisłych we współczesnej refleksji literaturoznawczej, czy raczej szerzej -- humanistycznej, warto podsumować spostrzeżeniem, że na przestrzeni ostatnich kilku dekad obserwujemy jej stałe zwiększanie się. W~połowie wieku XX uświadomienie faktu, że wysoko rozwinięta nauka może stworzyć broń zdolną zniszczyć naszą planetę, stało się prawdziwym szokiem. Od ponad wieku literatura starała się zasymilować odkrycia nowej fizyki i~włączyć do własnego dyskursu, próbując postrzegać fizykę jako przedmiot refleksji humanistycznej. Nauki humanistyczne widzą w~fizyku człowieka opisującego zarówno Wszechświat, jak i~siebie, obserwatora tego Wszechświata. Jednocześnie nasze życie codzienne skomplikowało się w~sposób, który możemy pojąć dzięki odniesieniom do nauk ścisłych. \textit{Science fiction} i~jej pokrewne czy potomne gatunki: cyberpunk, solarpunk i~cli-fi pozwalają oswoić ten nowy i~fantastyczny świat. Dzięki utworom cyberpunkowym zaawansowana informatyka stała się tematem rozważań humanistycznych, zaś narastające poczucie zagrożenia związane z~zatruciem ziemskich biocenoz znalazło swój wyraz w~literaturze doby antropocenu. To dzięki utworom cli-fi i~solarpunk niespecjaliści mogą dzielić się obawami o~przyszłość Ziemi i~snuć -- niekiedy niestety utopijne -- scenariusze jej ocalenia.

Nauki humanistyczne, które badają dokonania ludzkości i~opisują stworzoną przez nas kulturę, w~ostatnich dziesięcioleciach musiały włączyć w~sferę swoich zainteresowań dyskurs nauk ścisłych: fizyki, biologii, chemii i~geologii, a~obecnie także ekologii. Dzieje się tak dlatego, że wzrastająca świadomość olbrzymiego zagrożenia, jakie niesie ze sobą zatrucie świata i~naruszenie równowagi w~naturze, staje się nieodłącznym elementem życia współczesnych ludzi. Cywilizacja techniczna gloryfikowała nauki ścisłe, ale bezmyślne wykorzystywanie nowych technologii zaprowadziło ją na skraj katastrofy. Jednak to właśnie w~naukach ścisłych ludzie upatrują nadziei na ratunek, na stworzenie cywilizacyjnie zaawansowanego, a~jednocześnie ekologicznie czystego świata. Tak jak kilkadziesiąt lat temu oczekiwano wojny atomowej jednocześnie mając nadzieję, że nie nastąpi ona nigdy, tak dziś ludzkość wie, że tyka nowy, ekologiczny ,,zegar zagłady''. Literatura staje się sposobem wyrażania tych strachów i~tej nadziei, a~nowe podgatunki literackie narodziły się w~odpowiedzi na trapiące ludzkość traumy.

\paragraph{Nota bibliograficzna}
Pisząc artykuł wykorzystywałam tezy z~trzech monografii mojego autorstwa \textit{O~pomieszaniu gatunków. Science fiction a~postmodernizm}
%\label{ref:RNDQPUmQePq2Z}(2010),
\parencite*[][]{oramus_o_2010}, %
 \textit{Darwinowskie} \textit{paradygmaty. Mit teorii ewolucji w~kulturze współczesnej} 
%\label{ref:RNDBhWVJeJPjw}(2015)
\parencite*[][]{oramus_darwinowskie_2015}, %
 \textit{Stany splątane. Fizyka i~literatura współczesna} 
%\label{ref:RNDF73Cmp6ONa}(2020).
\parencite*[][]{oramus_stany_2020}.%



\end{artplenv}