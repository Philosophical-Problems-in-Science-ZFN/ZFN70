\begin{newrevplenv}{Jerzy Pogonowski}
	{Poznanie geometryczne z~kognitywnego punktu widzenia}
	{Poznanie geometryczne z~kognitywnego punktu widzenia}
	{Poznanie geometryczne z~kognitywnego punktu widzenia}
	{Uniwersytet im. Adama Mickiewicza w~Poznaniu, Zakład Logiki Stosowanej}
	{Mateusz Hohol, {\em Foundations of Geometric Cognition},
	Routledge,\\
	London and New York, 2020, ss.~188.}








\section{Uwagi wstępne}

\lettrine[loversize=0.13,lines=2,lraise=-0.03,nindent=0em,findent=0.2pt]%
{K}{}ilka lat temu Mateusz Hohol wspólnie z~Bartoszem Brożkiem
opublikował książkę {\em Umysł matematyczny} \parencite{brozek_umysl_2014},
która dotąd miała trzy wydania (2014, 2016, 2017) i~doczekała się bardzo pozytywnych recenzji. Książka {\em
Foundations of Geometric Cognition} również dotyczy specyficznego
rodzaju ludzkiego poznania, jakim jest poznanie matematyczne, przy
czym uwaga autora skierowana jest przede wszystkim na poznanie
geometryczne. Te zagadnienia wstępnie omawiał autor stosunkowo
niedawno w~kilku artykułach, np.: \textit{Od przestrzeni do
abstrakcyjnych pojęć. W~stronę poznania geometrycznego} \parencite[, 
tekst odczytu wygłoszonego na V~Konferencji Filozofii Matematyki i~Informatyki, 
Poznań, 2016]{hohol_od_2018}, \textit{Cognitive artifacts for
geometric reasoning} \parencite{hohol_cognitive_2019}. Mateusz Hohol
prowadzi także kursy dotyczące poznania matematycznego (wspólnie z~Krzysztofem Ciporą). Zajmują go również problemy metodologiczne
nauk kognitywnych, czego wyrazem jest m.in. książka {\em Wyjaśnić
umysł: struktura teorii neurokognitywnych} \parencite{hohol_wyjasnic_2013}. Ma
wreszcie doświadczenie, jeśli chodzi o~eksperymentalne aspekty
nauk kognitywnych, co poświadczają jego liczne artykuły
przedstawiające wyniki konkretnych badań.

Refleksja nad matematyką jako sferą ludzkiego poznania ma
wielowiekową tradycję, datującą się od czasów Platona. Rozważane w~niej były problemy ontologiczne (jak istnieją obiekty
matematyczne), epistemologiczne (jaki mamy dostęp poznawczy do
obiektów matematycznych), zastanawiano się nad związkami łączącymi
matematykę ze światem fizycznym (dlaczego matematyka jest
skuteczna w~opisie tego świata i~dostarcza wiarygodnych
predykcji), próbowano też objaśniać mechanizmy rządzące rozwojem
matematyki. Tego typu refleksje należą do filozofii, metodologii
nauk oraz historii matematyki. Niektóre nowe typy refleksji
proponują natomiast nauki kognitywne, których głównym zadaniem
jest badanie możliwości poznawczych, przede wszystkim człowieka,
ale także innych organizmów. Ponieważ nauki kognitywne odwołują
się do ustaleń i~metod wypracowanych w~wielu innych dyscyplinach
(m.in.: biologii, psychologii, logice, lingwistyce, informatyce),
więc można oczekiwać, że proponowane przez nie wizje i~teorie
struktur poznawczych będą wieloaspektowe i~że owa
interdyscyplinarność dostarczy głębszych wyjaśnień badanych
zjawisk.

Prace proponujące kognitywne ujęcia matematyki dotyczą m.in.:
mechanizmów tworzenia pojęć abstrakcyjnych, przyswajania pojęcia
liczby, wykształcania się umiejętności arytmetycznych, orientacji
przestrzennej, reprezentacji pojęć matematycznych w~umyśle. Stosunkowo
liczne w~literaturze przedmiotu są prace poświęcone aspektom
czysto numerycznym. Mniej liczne są natomiast opracowania
poświęcone podstawom poznania geometrycznego, co autor podkreśla w~swojej książce.

\section{Kognitywne podejście do geometrii}

Mateusz Hohol stara się wykazać w~{\em Foundations of Geometric
Cognition}, że ludzkie umiejętności geometryczne są ściśle
powiązane z~możliwościami poznawczymi, które oferuje nam mózg,
zwłaszcza w~dziedzinie reprezentacji wzrokowego postrzegania
przestrzennego, przede wszystkim odnoszącego się do rozpoznawania
obiektów oraz przemieszczania się w~przestrzeni. Możliwości te
realizują się już na wczesnym etapie rozwoju. Za rozwój poznania
geometrycznego są jednak odpowiedzialne również inne czynniki
wskazane przez autora, m.in.: umiejętność tworzenia pojęć
abstrakcyjnych, kształtowanie się myślenia geometrycznego w~oparciu o~struktury językowe oraz umiejętność tworzenia diagramów.
W~centrum zainteresowania autora znajduje się geometria
euklidesowa -- autor przedstawia argumentacje, które mają
uzasadnić, że stworzenie takiego systemu geometrii było możliwe
właśnie ze względu na wspomniane wyżej czynniki.

Autor podzielił tekst na cztery rozdziały, zachowując wewnątrz
każdego z~nich jednorodność poruszanej problematyki. Za trafne pod
względem kompozycyjnym uważać należy rozpoczęcie każdego rozdziału
zapowiedzią zawartych w~nim wyników, a~zakończenie podsumowaniem
tego, co w~danym rozdziale udało się wykazać.

\subsection{2.1. Myślenie geometryczne}

Mateusz Hohol rozpoczyna rozdział pierwszy ({\em Geometric
thinking, the paradise of abstraction}) od prezentacji wybranych
faktów dotyczących genezy systemu geometrii wyłożonego w~{\em
Elementach} Euklidesa. Jak wiadomo, traktat Euklidesa obejmował
nie tylko geometrię płaską i~przestrzenną (Księgi I--IV,
XI--XIII), ale zawierał także rozdziały poświęcone arytmetyce
(Księgi VII--IX) i~teorii proporcji wielkości (Księgi V--VI).
Przez wiele wieków {\em Elementy} były podstawowym tekstem
matematycznym, a~geometria traktowana była jako główna dyscyplina
matematyczna. Także refleksja filozoficzna nad matematyką skupiała
się na problemach związanych z~geometrią, co autor pokazuje,
przytaczając poglądy Platona, Kartezjusza i~Kanta. Mateusz Hohol
odnotowuje także uwagę Henri Poincar\'{e}go, że geometria
euklidesowa jest i~pozostanie najbardziej dogodnym spośród wielu
możliwych systemów geometrii. Wskazuje jednak również na pogląd
przeciwny, głoszony przez Hermanna von Helmholtza, który uważał,
że geometria euklidesowa nie jest jakoś szczególnie
uprzywilejowana. Jej użyteczność w~opisie otaczającej nas
rzeczywistości powinna podlegać, wedle Helmholtza, empirycznej
weryfikacji.

Rozdział pierwszy porusza jeszcze trzy inne typy zagadnień:
psychologiczne aspekty rozwoju umiejętności geometrycznych,
nauczanie geometrii euklidesowej w~szkole oraz stosunek
dotychczasowych refleksji kognitywnych do geometrii. Autor
przypomina ustalenia B\"{a}rbel Inhelder i~Jeana Piageta dotyczące
rozwoju intelektualnego dzieci, przede wszystkim jeśli chodzi o~przyswajanie sobie pojęć geometrycznych, ale przywołuje także
wyniki bardziej współczesnych badań, ukazujące pewne ograniczenia
stwierdzeń formułowanych przez wspomnianych psychologów. Pisząc o~nauczaniu geometrii w~szkole, autor przedstawia propozycje Diny
van Hiele-Geldolf oraz Pierre'a van Hiele, którzy postulowali, że
przyswajanie sobie pojęć i~umiejętności geometrycznych odbywa się
na kolejnych etapach, tworzących swoistą hierarchię. Etap pierwszy
jest wizualny, drugi opisowy, trzeci relacyjny, czwarty odnosi się
do formalnych dedukcji, a~piąty do rozważań na metapoziomie.
Mateusz Hohol przywołuje jednak również ustalenia innych badaczy,
którzy uzupełniali omawiany model, opierając się na wynikach
eksperymentów dydaktycznych.

Pod koniec rozdziału pierwszego autor krótko przypomina znaczące
fakty związane z~genezą nauk kognitywnych. Zauważa, że w~dotychczasowych refleksjach dotyczących matematyki prowadzonych w~tych naukach mieliśmy do czynienia z~ciekawą różnicą: w~badaniach
zdolności numerycznych odnoszono się zwykle do najprostszych
czynności związanych z~liczeniem, natomiast rozważania dotyczące
geometrii (prowadzone np. w~ramach badań nad sztuczną
inteligencją) odwoływały się raczej do rozumowań, przeprowadzanych
na bardziej abstrakcyjnym poziomie.

\subsection{2.2. Badania eksperymentalne}

Rozdział drugi ({\em The hardwired foundations of geometric
cognition}) informuje przede wszystkim o~wynikach badań
eksperymentalnych nad poznaniem geometrycznym, ale także o~próbach
konstrukcji teorii, wyjaśniających wyniki eksperymentów.
Występujący w~tytule rozdziału termin ,,hardwired'' ma następujące
objaśnienia w~{\em Cambridge Dictionary}:

%\begin{small}
\begin{enumerate}

\item automatically thinking or behaving in a~particular way;

\item built to work in a~particular way that cannot be changed
with new software;

\item physically connected by wires for carrying a~signal.

\end{enumerate}
%\end{small}

Słowniki angielsko-polskie podają następujące tłumaczenia terminu
,,hard\-wired'': zaprogramowany, zakorzeniony, wbudowany,
zakodowany, podłączony. Tytuł rozdziału zapowiada więc, że będzie
w~nim mowa o~tym, jak poznanie geometryczne opiera się na
strukturach i~funkcjonowaniu mózgu i~układu nerwowego, a~dokładniej: co na razie wiemy na ten temat na podstawie
przeprowadzanych eksperymentów.

Ponieważ dysponujemy sporą różnorodnością wyników badań
eksperymentalnych dotyczących tego, jak ludzie i~zwierzęta poznają
świat, więc potrzebna jest jakaś rozsądna ich systematyzacja.
Mateusz Hohol wykorzystuje w~tym celu cztery rodzaje pytań
eksplanacyjnych zaproponowanych przez Nikolaasa Tinbergena w~\textit{On
aims and methods of ethology} \parencite{tinbergen_aims_1963}:
\begin{center}
\begin{small}
\begin{tabular}{|l|l|l|}

\hline

Ujęcie & Aspekty bezpośrednie & Aspekty końcowe (ewolucyjne)\\

\hline

synchroniczne & Jak to działa? & Jak to wspomaga przystosowanie?\\

 & (pytanie o~przyczynę) & (wartość adaptacyjna)\\

\hline

diachroniczne & Jak to się rozwija? & Jak to powstało?\\

 & (ontogeneza) & (filogeneza)\\

\hline

\end{tabular}
\end{small}
\end{center}
\medskip

W~filozoficznej refleksji nad poznaniem geometrycznym wybierano
(za Kantem) tezę o~wrodzoności tego typu poznania i~w konsekwencji
przekonanie o~istnieniu jakiegoś modułu w~mózgu, który byłby za
geometryczne poznanie odpowiedzialny lub też wybierano pogląd
(reprezentowany przez Helmholtza), że poznanie to kształtowane
jest w~trakcie uczenia się struktury przestrzennej świata
rozpoznawanej w~aktach percepcji. We współczesnych badaniach w~psychologii oraz w~neuronauce często zakłada się, że bardziej
zaawansowane zdolności poznawcze nadbudowywane są nad prostszymi,
filogenetycznie ukształtowanymi zdolnościami, poprzez interakcje
podmiotów poznających z~ich otoczeniem fizycznym i~społecznym.
Mateusz Hohol przyjmuje takie właśnie założenia w~swoich analizach
genezy i~funkcjonowania poznania geometrycznego.

Dla ustalenia tych filogenetycznych uwarunkowań przeprowa\-dzano
wiele eksperymentów (na dzieciach i~dorosłych, a~także na
gryzoniach, ptakach i~innych stworzeniach) mających wykazać, że w~próbach osiągnięcia celu jakiejś akcji (np. poszukiwaniu
pożywienia) podmioty eksperymentu wykorzystują informacje natury
geometrycznej. Takie informacje dotyczyć mogą kierunków,
odległości, kątów, kształtów, ale również pewnych operacji
geometrycznych, np. rotacji kształtów płaskich oraz brył.
Prowadzono także obserwacje zachowań innych zwierząt (np. owadów),
wnioskując na ich podstawie o~wykorzystywaniu przez badane istoty
reprezentacji przestrzennych.

Wyniki badań eksperymentalnych powinny być oczywiście
interpretowane w~ramach jakiejś teorii. Autor przywołuje kilka
znanych z~literatury przedmiotu propozycji, np. postulat istnienia
szczególnego systemu poznawczego u~kręgowców, nazywanego modułem
geometrycznym, który miałby być odpowiedzialny za wykorzystanie
wskazówek geometrycznych z~otoczenia dla lokalizacji określonych
miejsc. Tzw. rama metryczna będąca składnikiem tego modułu miałaby
odpowiadać temu, co wcześniej w~psychologii nazywano mapą
poznawczą, a~co odnosiło się do niejęzykowych reprezentacji
umysłowych środowiska. Za przejawy działania takiego modułu uważa
się aktywację tzw. komórek miejsca, znajdujących się w~hipokampie.
Także w~korze śródwęchowej wykryto komórki siatki, kierunku ruchu,
prędkości oraz komórki graniczne. Autor zwraca szczególną uwagę na
propozycje przedłożone przez Elizabeth Spelke i~jej
współpracowników, które zamiast pojęcia modułu wykorzystują
pojęcie rdzennego systemu poznawczego. System ten spełnia wiele
funkcji, jest np. odpowiedzialny za przyswojenie sobie przez
dziecko zasad zachowania się obiektów nieożywionych, wykształcenie
intuicji protonumerycznych, rozumienie nakierowanych na cel
zachowań obiektów ożywionych. Umożliwia on nam także geometryczną
reprezentację otoczenia, w~którym się poruszamy oraz rozpoznawanie
kształtów płaskich i~przestrzennych. Jeśli chodzi zatem o~poznanie
geometryczne, to (zdaniem Spelke) bazuje ono na dwóch systemach
poznawczych odbierających i~przetwarzających informacje z~otoczenia, które charakteryzują się tym, że są stare ewolucyjnie,
wczesne rozwojowo, a~w~dodatku także kulturowo uniwersalne:

\begin{enumerate}

\item {\em Core system of layout geometry}. Związany jest z~hipokampem i~korą śródwęchową. Przetwarza reprezentacje
trójwymiarowych układów przestrzennych. Jest wykorzystywany do
orientacji w~otaczającym organizm środowisku. System ten jest
czuły na informacje dotyczące odległości oraz kierunku (ale nie
kątów).

\item {\em Core system of object geometry}. Związany jest z~bocznymi strukturami płata potylicznego. Przetwarza obrazy
dwuwymiarowe i~ruchome wzorce wizualne, umożliwia m.in.
dokonywanie w~umyśle obrotu obiektów. Służy przede wszystkim do
kategoryzacji obiektów. System ten jest czuły na informacje
dotyczące długości i~kątów (ale nie kierunku).

\end{enumerate}

W~dalszych częściach tego rozdziału autor odpowiada na cztery
pytania wspomniane wcześniej, czyli pytania o: przyczynę, wartość
adaptacyjną, ontogenezę i~filogenezę rdzennych systemów geometrii.
Szczególnie interesujące są fragmenty dotyczące filogenezy, w~których
autor omawia wyniki obserwacji i~eksperymentów (oraz związanych z~nimi kontrowersji) poświęconych reprezentacji przestrzennej i~nawigacji u~różnych gatunków zwierząt. Interpretacja tych wyników
może skłaniać do przypuszczenia, że jakieś formy poznania
geometrycznego były obecne już na bardzo wczesnych etapach
ewolucji. Z~kolei ustalenia ontogenetyczne pozwalają przypuszczać,
że pewne ograniczenia każdego z~dwóch rdzennych systemów geometrii
zostają przezwyciężone poprzez nowy, specyficzny dla człowieka
system, w~którym adekwatnie ujmowane są długość, kierunek, kąty, a~którego utworzenie umożliwione jest poprzez wykorzystanie języka
zawierającego wyrażenia przestrzenne oraz wykorzystanie
symbolicznych reprezentacji graficznych.

\subsection{2.3. Abstrakcje i~poznanie ucieleśnione}

Rozdział trzeci ({\em Embodiment and abstraction}) dotyczy
tworzenia i~przetwarzania pojęć abstrakcyjnych. Status pojęć
abstrakcyjnych był zawsze w~centrum zainteresowania filozofii. W~książce Mateusza Hohola nie chodzi jednak oczywiście o~odwoływanie
się do refleksji filozoficznych na temat tych pojęć, ale o~to, jak
ujmowane są one w~perspektywie nauk kognitywnych. W~początkowej
fazie rozwoju tych nauk pojęcia rozumiane były jako amodalne
reprezentacje umysłowe, przetwarzane przez systemy poznawcze niezwiązane bezpośrednio np. z~percepcją ruchu, a~odwołujące się
raczej do manipulacji na symbolach. Mateusz Hohol wskazuje jednak
na pewne istotne trudności takiego amodalnego podejścia na
przykładzie omawianego przez Harnada tzw. problemu ugruntowania
pojęć: w~jaki sposób możemy nadać znaczenie symbolom, odwołując
się jedynie do innych symboli i~łączących je reguł? 

Bardziej
współczesne podejścia w~naukach kognitywnych próbują rozwiązać
tego typu problemy poprzez odwołanie się do idei poznania
ucieleśnionego. Wedle takich ujęć pojęcia ugruntowane są w~ostatecznym rozrachunku na aktywnościach sensoro-motorycznych.
Skrajna wersja ucieleśnionego poznania w~odniesieniu do pojęć
matematycznych została zaprezentowana w~monografii Lakoffa i~N\'{u}\~{n}eza {\em Where Mathematics Comes from: How the Embodied Mind Brings Mathematics into Being} \parencite{lakoff_where_2000},
której autorzy twierdzą, że całość genezy i~funkcjonowania
matematyki można wyjaśnić przez odwołanie się do tworzenia
abstrakcyjnych pojęć metodą metafor poznawczych, które wywodzą te
pojęcia z~pojęć prostszych, a~ostatecznie odwołują się do
doświadczenia sensoro-motorycznego. Propozycjom Lakoffa i~N\'{u}\~{n}eza brakuje jednak solidnych podstaw empirycznych, a~ich prezentacja zawiera stwierdzenia, na których nietrafność
zwracali uwagę matematycy w~recenzjach tej pracy. Mateusz Hohol
przywołuje również zastrzeżenia wobec tej koncepcji formułowane na
gruncie psychologii rozwojowej (zdolność rozumienia metafor
poznawczych jest późniejsza niż zdolność rozumienia pojęć
abstrakcyjnych), neuronauki (struktury sensoro-motoryczne nie
zawsze są aktywowane podczas operowania na pojęciach
abstrakcyjnych) czy wreszcie historii matematyki (podstawy
greckiej geometrii nie były ustalane metodą tworzenia metafor
pojęciowych). 

Mateusz Hohol odnosi się natomiast z~sympatią do
pewnych umiarkowanych koncepcji ucieleśnionego poznania, jak np.
koncepcja ucieleśnionego i~odcieleśnionego poznania Guya Dove'a,
która miałaby uzgodnić reprezentacyjną teorię umysłu ze stosownie
rozumianym poznaniem ucieleśnionym, odwołując się również do
wypracowanej przez Lawrence'a Barsalou teorii symboli
percepcyjno-\mbox{-motorycznych} i~proponowanej przez Allana Paivio
koncepcji podwójnego kodowania. Potwierdzono empirycznie istnienie
w~strukturach sensoro-motorycznych tzw. symulatorów, będących
swoistymi wzorcami aktywności, przywoływanymi podczas operowania
pojęciami. Teoria podwójnego kodowania postuluje istnienie dwóch
rodzajów reprezentacji umysłowych: jeden z~nich funkcjonuje
analogowo, na zasadzie podobieństwa percepcyjnego, drugi natomiast
odpowiedzialny jest za kodowanie symboli językowych. Pojęcia
konkretne kodowane i~przetwarzane są w~obu wymienionych rodzajach
reprezentacji, natomiast pojęcia abstrakcyjne jedynie w~drugim z~nich. I~to właśnie pojęcia abstrakcyjne są od-cieleśnione w~tym
sensie, gdyż ich znaczenie nie zależy tylko od czynników
sensoro-motorycznych. Oba systemy reprezentacji jednak
współdziałają, przez co problem ugruntowania pojęć wydaje się być
rozwiązany. Mateusz Hohol przytacza liczne przykłady wyników badań
eksperymentalnych (neuroobrazowanie, przezczaszkowa stymulacja
magnetyczna, potencjały wywołane, eksperymenty behawioralne),
które potwierdzają tezę o~werbalnym kodowaniu pojęć
abstrakcyjnych. Wedle Guya Dove'a kodowanie językowe pojęć wyłania
się w~procesie ontogenezy, zgodnie z~koncepcją rozwoju
psychicznego proponowaną przez Lwa Wygotskiego.

\subsection{2.4. Historia kognitywna}

W~rozdziale czwartym ({\em Cognitive artifacts and Euclid:
diagrams and formulae}) autor wraca do omawianego na początku
książki systemu geometrii Euklidesa. Nie są to jednak rozważania
na temat historii matematyki, ale raczej to, co Reviel Netz
w~swoim znanym dziele {\em The Shaping of
Deduction in Greek Mathematics} \parencite{netz_shaping_1999} nazywa
historią kognitywną. W~interesującym nas
tutaj przypadku chodzi o~historię kognitywną poznania
geometrycznego, którego rezultatem jest w~pełni ukształtowany
system geometrii euklidesowej. Istotną rolę w~tym systemie pełnią
rozumowania dedukcyjne. Do podstawowych środków używanych w~tych
rozumowaniach Netz zalicza diagramy wyposażone w~oznaczenia
literowe oraz wyspecjalizowany język, odnoszący się do pojęć
geometrycznych i~skłądający się ze stabilnych co do formy i~znaczenia wyrażeń. Wedle Netza użycie tych środków przyczynia się
do uzyskania przez rozważane rozumowania dwóch fundamentalnych
cech: pełnej ogólności oraz koniecznego charakteru uzyskiwanych
wniosków.

Jak wiadomo, w~{\em Elementach} podaje się definicje, postulaty i~pojęcia wspólne. Niektóre z~definicji są objaśnieniami
intuicyjnymi. Pojęcia wspólne związane są z~tym co dzisiaj
nazywamy aksjomatami identyczności. Postulaty sformułowane są
następująco \parencite[s.~275]{euklides_elementy_2013}:

\begin{enumerate}

\item Niech będzie postulowane, aby z~każdego punktu do każdego
punktu poprowadzić linię prostą.

\item I~przedłużyć ograniczoną prostą w~sposób ciągły na prostej.

\item Z~danego centrum i~danym promieniem zakreślić koło.

\item Kąty proste są równe jeden drugiemu.

\item Gdy prosta, padając na dwie inne proste, tworzy kąty
wewnętrzne na tej samej części mniejsze dwóm kątom prostym, to te
dwie proste przedłużane nieskończenie dotkną się na tej części, na
której są mniejsze dwóm kątom prostym.

\end{enumerate}

We wszystkich twierdzeniach i~zadaniach konstrukcyjnych
wykorzystuje się jedynie podane definicje, pojęcia wspólne i~postulaty, bez żadnych odwołań do jakichkolwiek ustaleń
zewnętrznych. Całkiem osobnym problemem jest to, że w~rozumowaniach Euklidesa są pewne ukryte założenia, dotyczące np.
przecinania się linii, na co zwrócono uwagę w~XIX wieku,
proponując aksjomatyczne ujęcia geometrii euklidesowej (Moritz
Pasch, David Hilbert).

Zasadniczo wszystkie twierdzenia w~{\em Elementach} Euklidesa mają
wyraźnie zaznaczoną schematyczną budowę \parencite[s.~122--123]{euklides_elementy_2013}:

\begin{enumerate}

\item {\em Protasis}. Teza.

\item {\em Ekhtesis}. Ustalenie oznaczeń odnoszących się do
diagramu.

\item {\em Diorismos}. Powtórzona teza w~wersji z~oznaczeniami.

\item {\em Katauskene}. Konstrukcja, zasadniczy pomysł dowodu,
trik.

\item {\em Apodeixis}. Uzasadnienie.

\item {\em Symperasma}. Powtórzona teza.

\end{enumerate}

W~przypadku dowodów przeprowadzanych metodą nie wprost (np.
twierdzenie 18 w~Księdze V~i twierdzenia 7 oraz 26 w~Księdze VI)
część 4 rozpoczyna zwrot ,,w przeciwnym razie'' (lub zaprzeczenie
tezy), a~część 5 kończy zwrot ,,co jest niemożliwe''
(,,absurdalne''). Wyróżnia się ponadto:

\begin{enumerate}

\item {\em twierdzenia} -- część 6 jest powtórzeniem części 1, w~której użyty jest zwrot ,,twierdzę, że'';

\item {\em problemy (zadania konstrukcyjne)} -- część 6 jest
powtórzeniem części 3, a~w części 1 używa się zwrotu ,,należy
więc''.

\end{enumerate}

Poszczególne twierdzenia {\em Elementów} prezentowane są w~postaci
diagramu (z literowymi oznaczeniami) oraz tekstu, zorganizowanego
wedle podanego wyżej schematu. Mateusz Hohol analizuje szczegółowo
twierdzenie 1 z~Księgi I~(problem konstrukcji trójkąta
równobocznego). Dyskutuje rolę diagramów w~różnych ujęciach
geometrii (ilustracyjną np. w~{\em Grundlagen der Geometrie}
Davida Hilberta \parencite{hilbert_grundlagen_1899}, a~informacyjną w~{\em Elementach}
Euklidesa) oraz ich status ontologiczny (czy są reprezentacją
czegoś innego, jakichś bytów platońskich, czy też są po prostu
konkretnymi obiektami). Przywołuje także ustalenia Reviela Netza
dotyczące struktury i~funkcji języka geometrii greckiej. Wedle
tych ustaleń określić można zespół wykorzystywanych wyrażeń
językowych, dzieląc je na kilka typów: wyrażenia dotyczące
obiektów, konstrukcji, zależności (np. proporcji), argumentacji, a~także formuły drugiego rzędu (np. {\em quod erat demonstrandum}).

Jeśli chodzi o~ogólność twierdzeń geometrycznych {\em Elementów},
to -- zdaniem Netza -- wynika ona z~możliwości powtarzania
rozumowań, na co miałyby wskazywać np. liczne w~tekście użycia
wyrażeń w~rodzaju: ,,w podobny sposób dowodzimy, że''. Takie
wyrażenia mają wskazywać na posiadaną przez matematyka dyspozycję
do przeprowadzania dowodów podobnych do już uzyskanych.
Twierdzenia geometryczne uzyskują jednak walor ogólności także
przez to, że odwołujemy się w~nich wyłącznie do znaczenia
używanych terminów (określonego przez definicje i~postulaty).

Mateusz Hohol cytuje następujący fragment z~pracy Netza, odnoszący
się do waloru konieczności dowodów w~matematyce greckiej \parencite[s.~215]{netz_shaping_1999}:

\myquote{

The necessity preserving properties of Greek mathematical proofs
are all reflected by their proofs, and no meta-mathematical
considerations are required. As a~rule, the necessity of
assertions is either self-evident (as in starting points) or
dependent on nothing beyond the immediate background. [\ldots] The
structure of derivation is fully explicit. Immediate inspection is
possible; this, and no meta-mathematical consideration is the key
to necessity. [\ldots] Greek mathematical proofs offer nowhere to
hide. Everything is expectable.

}

Formalne pojęcie dowodu zostało zaproponowane dopiero wtedy, gdy
logika matematyczna osiągnęła wystarczający ku temu poziom
rozwoju. Jednak zanim to nastąpiło matematycy przeprowadzali rzecz
jasna rozumowania dedukcyjne, które były akceptowane w~ich
społeczności. Historia matematyki notuje stosunkowo nieliczne
przypadki nieuprawnionych (z późniejszego punktu widzenia)
rozumowań matematycznych. Refleksja teoretyczna nad rozwojem
pojęcia dowodu matematycznego jest żywo obecna we współczesnej
filozofii matematyki, nie było jednak zamiarem autora recenzowanej
książki włączenie się do takiej refleksji. Również Reviel Netz we
wstępie do swojej książki deklaruje, że nie zajmuje się tak
ogólnymi problemami, a~skupia swoją uwagę na konkretnej praktyce
dowodowej w~matematyce greckiej.

\section{Podsumowanie}

Dokonajmy na koniec krótkiego podsumowania: co udało się ustalić
autorowi, o~jakie elementy można byłoby ewentualnie uzupełnić jego
argumentację, jakie dalsze badania wydają się potrzebne w~interesującej nas materii.

Mateusz Hohol uwzględnia w~swoich rozważaniach imponującą liczbę
przeprowadzonych w~naukach kognitywnych eksperymentów i~obserwacji. Sama bibliografia liczy kilkaset pozycji, a~na prawie
każdej stronie książki przywoływane są wyniki badań (na marginesie
warto dodać, że autor także przeprowadzał badania empiryczne). Ten
ogrom materiału jest jednak dobrze uporządkowany, m.in. poprzez
przemyślane pogrupowanie tego materiału, tak, aby uzyskać
odpowiedzi na wspomniane wyżej pytania eksplanacyjne (o przyczynę,
wartości adaptacyjne, ontogenezę i~filogenezę).

Autor argumentuje na rzecz swojej wizji poznania geometrycznego
odwołując się właśnie do tak uporządkowanego materiału. Wedle
Mateusza Hohola nasze kompetencje geometryczne nie są wynikiem
uzyskanego jedynie poprzez akty percepcji przestrzennego obrazu
otaczającego nas świata, nie są też wytworzone wyłącznie poprzez
eksplorację otoczenia. Autor twierdzi mianowicie, że źródłem
kompetencji geometrycznych są dwa rdzenne systemy poznawcze,
dotyczące rozpoznawania obiektów oraz układu otoczenia. Za
filogenetycznym zakorzenieniem tych układów ma przemawiać to, że
można zaobserwować ich działanie we wczesnych etapach rozwoju
dziecka, a~także u~innych zwierząt. Jednak dopiero współdziałanie
tych systemów pozwala uzyskać wrażliwość zarówno na kąty i~kierunki, jak i~na odległości. Z~kolei to współdziałanie
wzmacniane jest w~trakcie indywidualnego rozwoju, wspomaganego
używaniem języka oraz innych inwencji kulturowych.

Mateusz Hohol unika też skrajności w~podejściu do umysłowego
przetwarzania pojęć abstrakcyjnych, czyli z~jednej strony
stanowiska w~pełni amodalnego, a~z drugiej stanowiska
propagującego całkowite ucieleśnienie poznania. Wybiera stanowisko
pośrednie, umiarkowanego ucieleśnienia, dopuszczające wpływ
systemu sensoro-motorycznego na tworzenie pojęć abstrakcyjnych,
ale uwzględniające też istotną rolę języka w~ustalaniu znaczeń
pojęć abstrakcyjnych.

Autor odnosi się wreszcie także do cech specyficznych systemu
geometrii przedstawionego w~{\em Elementach} Euklidesa, a~konkretnie do ogólności oraz koniecznego charakteru twierdzeń
systemu. Wychodząc od koncepcji historii kognitywnych Reviela
Netza, Mateusz Hohol argumentuje na rzecz tezy, że poznanie
geometryczne specyficzne dla rodzaju ludzkiego przedstawiane w~tym
systemie wyłoniło się w~szczególnej ,,niszy kognitywnej'', z~charakterystycznym dla niej użyciem skodyfikowanego języka oraz
diagramów, wzmacniającym filogenetycznie zakorzenione możliwości
poznawcze.

Mateusz Hohol nie twierdzi bynajmniej, że uzyskaliśmy oto
ostateczną i~spójną wizję poznania geometrycznego. W~krótkim
rozdziale końcowym ({\em Conclusions and future directions for
research}) zwraca uwagę na potrzebę dalszych badań
eksperymentalnych dotyczących funkcjonowania i~współdziałania
rdzennych geometrycznych systemów poznawczych, a~także na potrzebę
wypracowania ogólnej perspektywy teoretycznej, w~której poznanie
geometryczne uwzględniałoby nie tylko rozwój indywidualny, ale
również czynniki natury społecznej.

Może warto podkreślić, że propozycje teoretyczne autora są rzecz
jasna {\em zewnętrzne} wobec samej geometrii, rozumianej jako
część matematyki. Oceniać je należy zatem jako określone
stanowisko epistemologiczne, biorące pod uwagę wyniki szczegółowe
różnych nauk (biologii, neuronauki, etologii itd.).

Dodam jeszcze garść uwag, które nasunęły mi się przy lekturze {\em
Foundations of Geometric Cognition}. Nie mają one charakteru
krytycznego wobec omawianego tekstu, wiążą się raczej z~oczekiwaniami (być może naiwnymi) piszącego te słowa wobec
rozważań na temat poznania geometrycznego. Jest oczywiste, że inne
oczekiwania w~tej materii może mieć matematyk, inne specjalista
nauk kognitywnych, a~jeszcze inne filozof zajmujący się
epistemologią. Książka z~pewnością doczeka się takich
specjalistycznych omówień, natomiast niniejsze uwagi mają
charakter dość ogólny.

\subsection{3.1. Uwagi historyczne}

Co to znaczy, że dane pojęcie jest geometryczne? W~systemie
geometrii Euklidesa pojęcia wyjściowe mają wysoce abstrakcyjny
charakter: punkt jest ,,tym, co nie ma części'', linia jest
,,długością bez szerokości'', linia prosta jest ,,tym, co leży
równo względem punktów na niej'' (jest przy tym obiektem
skończonym, który można jednak ,,dowolnie przedłużać w~sposób
ciągły''), powierzchnia jest ,,tym, co ma tylko długość i~szerokość'', powierzchnia płaska to ,,ta, która leży równo
względem prostych na niej'' itd. Pewne dalsze pojęcia definiowane
są poprzez zależności z~innymi lub poprzez wynik operacji
wykonywanych na innych (np. sfera jako wynik obrotu półokręgu).
Nie są to więc pojęcia, które bezpośrednio odpowiadają czemuś w~środowisku otaczającym człowieka. Pewnymi przybliżeniami są np.
płaska powierzchnia morza, widoczny promień światła, miejsce, na
które pada ten promień.

W~książce Mateusza Hohola mówi się głównie o~takich pojęciach jak:
odległość, kąt oraz kierunek (często: odróżnienie lewej od
prawej). Jak rozumiem, to właśnie te pojęcia mogą być uwzględniane
w~eksperymentach przeprowadzanych w~naukach kognitywnych.
Zastanawiam się, jak ustalenia na temat filogenetycznego
zakorzenienia tych pojęć przekładają się na tezę, że nasze
poznanie geometryczne bazuje na geometrii {\em euklidesowej} lub
prowadzi do zachowań zgodnych z~faktami (twierdzeniami) na gruncie
tej geometrii. Czy wyniki przeprowadzanych w~naukach kognitywnych
eksperymentów i~obserwacji jednoznacznie przesądzają o~tym, że
chodzi właśnie o~geometrię euklidesową, a~nie jakiś system --
nazwijmy to tak -- {\em protogeometrii}, która dopiero potem, na
skutek działania czynników kulturowych, przybiera postać systemu
geometrii euklidesowej? System geometrii przedstawiony w~{\em
Elementach} poprzedzały liczne obserwacje i~ustalenia dotyczące
praktycznych aspektów stosunków przestrzennych, o~czym obszernie
informują dzieła poświęcone początkom matematyki. Hipotetyczna
protogeometria uwzględniać mogłaby kierunki, odległości i~kąty
(może również kształty i~ruchy?), ale także lokalne (w skali
ludzkiego doświadczenia) własności przestrzeni fizycznej.

W~badaniach kognitywnych nad poznaniem numerycznym mówi się m.in.
o~tzw. {\em approximate number system} oraz o~{\em object tracking
system}: pierwszy z~nich miałby być odpowiedzialny za przybliżone
ustalanie wielkości kolekcji bez odwoływania się do języka, drugi
za umiejętność operowania na bardzo małych liczebnościach (do
czterech elementów). Czy w~badaniach nad poznaniem geometrycznym,
w~szczególności tych dotyczących omawianych w~książce rdzennych
geometrycznych systemów poznawczych również zakładamy
(odkrywamy?), że systemy te mają charakter przybliżony?

Książka Mateusza Hohola zawiera stosunkowo niewiele uwag
historycznych na temat rozwoju systemów geometrycznych. Nie można
jej tego braku zarzucać, bo cele autora nie były nakierowane na
taką analizę. Sądzę jednak, że rozwijając (tworząc?) teorię
poznania geometrycznego i~zakładając przy tym koncepcję historii
kognitywnej Reviela Netza należałoby sprawie tej poświęcić o~wiele
więcej uwagi. Jest bowiem w~tej historii wiele znaczących momentów
i~przełomów, które powinny być fascynujące dla nauk kognitywnych.
Materiału źródłowego do refleksji dostarczają liczne opracowania
historyczne (pisane głównie przez samych matematyków), a~otwartym
problemem pozostaje, czy nauki kognitywne mogą ten materiał
efektywnie wykorzystać dla swoich celów. Nie pretendując do
kompletności wskazałbym na kilka następujących zagadnień
(omawianych zresztą w~opracowaniach historycznych):

\begin{enumerate}

\item Geometryczne ujmowanie wielkości w~matematyce greckiej, co
zdaniem niektórych było odpowiedzialne za mniejszy nacisk
kładziony na rozważania arytmetyczne i~algebraiczne. Mawia się, że
połowa teorii liczb rzeczywistych była dostępna już Eudoksosowi,
jednak całość tej teorii uzyskaliśmy dopiero w~wieku XIX (Richard
Dedekind, Georg Cantor, Karl Weierstrass i~in.).

\item Dwa typy ujęć geometrii: syntetyczne, odwołujące się jedynie
do aksjomatów i~konstrukcji postulowanych w~systemie (Euklides, a~później np. Karl von Staudt) oraz analityczne (Kartezjusz, Fermat,
a~później wielu innych), wykorzystujące układy współrzędnych i~algebraiczne reprezentacje tworów geometrycznych. To drugie
podejście umożliwione zostało poprzez nowe rozumienie obiektów
geometrycznych i~wykonywanych na nich operacji \parencite{descartes_geometria_2015}.

\item Różnica między porządkiem chronologicznym a~logicznym w~dziedzinie systemów geometrycznych. Geometria rzutowa (Desargues,
a~później m.in. Poncelet, Monge, Chasles i~in.) została
wyodrębniona o~wiele później niż geometria euklidesowa, choć w~porządku logicznym jest o~wiele bardziej ogólna od tej ostatniej
(podobne ustalenia dotyczą też innych systemów geometrii, np.
afinicznej). Jak wiadomo, z~różnicą między porządkiem
chronologicznym a~logicznym mamy do czynienia także w~przypadku
rozumienia pojęcia liczby.

\item Zmiany w~poglądach na to, czym właściwie są systemy
geometrii. Zwykle podkreśla się, że stworzenie (odkrycie)
geometrii nieeuklidesowych przyczyniło się do porzucenia tezy, że
geometria Euklidesa z~konieczności jest {\em prawdziwą} geometrią,
dobrze opisującą świat fizyczny. Przełomowe było też powiązanie
systemów geometrii z~grupami przekształceń oraz niezmiennikami
takich przekształceń, proponowane w~programie z~Erlangen \parencite{klein_vergleichende_1872}. Być może ciekawe dla nauk kognitywnych byłoby zwrócenie
uwagi na {\em przekształcenia} obiektów geometrycznych oraz ich
niezmienniki.

\item Wykorzystanie w~rekonstrukcji systemu geometrii Euklidesa
liczb rzeczywistych (wraz z~charakterystycznym dla nich aksjomatem
ciągłości), co stało się głównie za sprawą {\em Grundlagen der
Geometrie} \parencite*{hilbert_grundlagen_1899} Dawida Hilberta. Miało to konsekwencje
metateoretyczne (m.in. możliwość udowodnienia kategoryczności
systemu). Autorzy, piszący w~XIX wieku o~aksjomacie ciągłości (np.
Georg Cantor, Richard Dedekind, Heinrich Weber) w~sposób wyraźny
podkreślali, że sformułowanie aksjomatu ciągłości jest twórczym
aktem matematyki, niezależnym od możliwości rozstrzygnięcia, czy
własność ciągłości przysługuje przestrzeni fizycznej.

\item Uogólnienia, które odwołują się do potęgi myślenia
matematycznego, nieskrępowanego doświadczeniem potocznym. Tu
dobrym (wczesnym historycznie) przykładem są systemy geometrii
wielowymiarowej (Hermann Grassmann, Bernhard Riemann). Całkiem
współczesne dzieje geometrii (i, ogólnie, matematyki) dostarczają
licznych dalszych przykładów.

\end{enumerate}

Może warte porównania z~punktu widzenia nauk kognitywnych są też
zestawy pojęć pierwotnych proponowanych przez matematyków. Moritz
Pash: punkt, odcinek; Giuseppe Peano: punkt, relacja leżenia
między; Mario Pieri: punkt, ruch; David Hilbert: punkt, prosta,
płaszczyzna, relacja leżenia między, relacja leżenia na,
przystawanie; Oswald Veblen: punkt, porządek; Edward Huntington:
sfera, zawieranie; Alfred Tarski: punkt, relacja leżenia między,
przystawanie. Doborem pojęć pierwotnych mogą kierować względy
metateoretyczne (czy było tak w~przypadku Euklidesa?), ale
motywację stanowić mogą też przekonania na temat intuicyjności
tych pojęć.

Przyznaję, że oczekiwałem, iż Mateusz Hohol wspomni w~swojej
książce o~podejściach do geometrii, które za punkt wyjścia biorą
nie wysoce abstrakcyjne pojęcia punktu i~prostej, ale raczej
pojęcia bryły i~relacji bycia częścią, tak jak w~słynnej
propozycji Alfreda Tarskiego \parencite{tarski_les_1929}. W~takich ujęciach (a
zaproponowano ich kilka) punkty są otrzymywane jako wynik
infinitarnych operacji na obszarach. Matematyczne obszary i~bryły
wydają się być bliżej związane z~obiektami i~fragmentami naszego
otoczenia. Relacja bycia częścią stanowi podstawę mereologicznego
rozumienia pojęcia zbioru. Nie wiem, czy w~naukach kognitywnych
przeprowadzano eksperymenty mające orzekać, które z~rozumień
pojęcia zbioru -- mereologiczne czy dystrybutywne -- jest jakoś
,,naturalne poznawczo''. Rozpoznawanie obiektów, ich konturów, ich
ruchów dotyczy chyba raczej obszarów, brył oraz ich części.
Umiejętność tworzenia pojęć abstrakcyjnych (punkt, prosta itp.)
jest rzecz jasna o~wiele późniejsza.

Filozoficzna refleksja nad przestrzenią i~geometrią dostarczała
różnych propozycji i~argumentacji, czasem spekulatywnych, a~czasem
odwołujących się do wyników badań empirycznych. Zwolennikiem
istnienia absolutnej przestrzeni był, jak wiadomo, Newton, podając
np. słynny argument odnoszący się do zachowania się wody w~wiadrze
wprawionym w~ruch wirowy. Inne było stanowisko Leibniza, dla
którego przestrzeń jako taka nie istniała samoistnie, a~tworzona
była jedynie poprzez wzajemne relacje między przedmiotami. Czy
przyjmując perspektywę historii kognitywnej Reviela Netza
uwzględniamy (oprócz zwrócenia uwagi na warstwę językową i~diagramy) także założenia natury filozoficznej, w~kontekście
których powstawał system geometrii Euklidesa? Być może jednym z~takich założeń jest akceptowanie tylko nieskończoności
potencjalnej (pamiętamy, że proste u~Euklidesa są obiektami
skończonymi, które mogą być jedynie dowolnie przedłużane). Już w~czasach przed Euklidesem istniały też różne poglądy na temat
struktury kontinuum geometrycznego (np. stanowiska Demokryta i~Arystotelesa). Jednak pojęcie ciągłości w~systemie Euklidesa nie
jest dogłębnie analizowane.

Historia kognitywna poznania geometrycznego powinna też chyba
uwzględniać rozwój optyki jako teoretycznej refleksji nad
widzeniem. Traktaty na ten temat pisał zarówno Euklides, jak i~Kartezjusz i~Newton. Interesujące jest zagadnienie, jak ustalenia
na terenie optyki wpływały na wyniki przewidywane i~uzyskiwane w~geometrii.

\subsection{3.2. Reprezentacje przestrzenne}

Nigdy nie prowadziłem badań eksperymentalnych dotyczących poznania
matematycznego, więc moje uwagi na ten temat nie są profesjonalne.
Jestem jednak ciekaw, jakie są ogólne zasady przeprowadzania
eksperymetów odnoszących się do reprezentacji przestrzennych u~różnych stworzeń. Reprezentacje przestrzenne mogą być oparte na
wzroku, węchu, słuchu, dotyku, grawitacji, echolokacji, zmianach
ciśnienia, polach magnetycznych, odczuciu temperatury lub
wilgotności, poczuciu równowagi i~zapewne jeszcze innych
czynnikach. Jeśli -- jak sugerowane jest to w~książce -- te
reprezentacje są stare filogenetycznie, to czy możliwe jest
uzyskanie na podstawie eksperymentów jednoznacznych wniosków na
temat poznania geometrycznego? Jak możliwości poznawcze
pojedynczych organizmów przekładają się na zachowania całych ich
stad lub kolonii (np. stada ptasie, kolonie mrówek i~termitów)?

Nie jestem pewien, czy eksperymenty myślowe mogą zostać z~pożytkiem wykorzystane w~badaniach poznania geometrycznego (choć
oczywiście mogą dostarczać rozrywki intelektualnej). Czy myślący
ocean ({\em Solaris}) dysponuje reprezentacjami przestrzennymi?
Czy inteligentne chmury, w~których otoczeniu nie byłoby żadnych
ciał stałych tworzyłyby reprezentacje przestrzenne oparte na
geometrii euklidesowej? Czy w~takim przypadku ciała stałe byłyby
obiektami ich mitologii? W~książce Mateusza Hohola nie
przeprowadza się tak niepoważnych rozważań, ale może miejsce dla
nich (w stosownie poważnej wersji) znalazłoby się w~refleksjach
nad sztuczną inteligencją?

W~przypadku człowieka mamy do czynienia z~widzeniem dwuocznym (co
umożliwia widzenie stereoskopowe), a~obraz na siatkówce jest
odwrócony (dopiero mózg jest odpowiedzialny za jego ponowne
odwrócenie). Może umknęło to mojej uwadze w~lekturze książki, ale
czy dysponujemy już jakimiś wnioskami na temat percepcji wzrokowej
stworzeń o~innej budowie oczu? Jakie reprezentacje geometryczne
mają takie stworzenia? Ponadto, w~ludzkich reprezentacjach
przestrzennych istotną rolę odgrywa korelacja ręka--oko. Jakie
korelacje są ważne dla stworzeń, które mają całkiem inną budowę
anatomiczną niż człowiek?

O~ile pamiętam, w~książce nie ma odniesień do iluzji (wzrokowych,
dotykowych, słuchowych itd.). Czy można je pomijać w~rekonstruowaniu podstaw poznania geometrycznego? W~interpretacji
obrazów bierzemy pod uwagę kierunek padania światła, rolę cienia,
tło itp. Zarówno przy manipulacji obiektami jak i~przy ocenie
układów przestrzennych bierzemy pod uwagę (odczuwamy) grawitację.
Oceny długości różnią się co do trafności w~zależności od tego,
czy porównujemy długości horyzontalne, czy też wertykalne.

Nie mamy oczywiście możliwości porównania intuicji geometrycznych,
które mieli nasi bardzo dawni przodkowie z~intuicjami żywionymi
obecnie na wczesnych etapach rozwoju. Jak rozumiem, pewne wnioski
na ten temat uzyskiwane są na podstawie badań przeprowadzanych
wśród plemion żyjących do dzisiaj we względnym odosobnieniu (np. w~Amazonii). Musimy zapewne pogodzić się z~faktem, że propozycje
dotyczące formowania się najwcześniejszych reprezentacji
przestrzennych mają w~sporej mierze charakter spekulacyjny.

W~tzw. językoznawstwie kognitywnym (którego nie jestem ani znawcą,
ani zagorzałym fanem, choć doceniam niektóre formułowane w~nim
propozycje) wiele uwagi poświęca się różnorodności językowych
reprezentacji zależności przestrzennych w~językach świata.
Poszczególne języki używają różnych środków morfo-syntaktycznych
dla wyrażania takich zależności. Dla przykładu, po polsku powiemy, że talerz z~podobizną św. Jana Pawła II może leżeć {\em na} stole lub wisieć
{\em na} ścianie, po niemiecku mamy różnicę: liegt {\em auf} dem
Tisch, h\"{a}ngt {\em an} der Wand. Akceptowalne (gramatycznie!)
jest powiedzenie, że prezydent pocałował papieża w~rękę (w
pierścień), a~także powiedzenie, że prezydent pocałował rękę
(pierścień) papieża, akceptowalne jest wreszcie powiedzenie, że
papież pocałował płytę lotniska, ale nie jest akceptowalne
powiedzenie, że papież pocałował lotnisko w~płytę. W~pewnych
językach przyszłość jest tym, co znajduje się w~przodzie, a~przeszłość tym, co znajduje się w~tyle, ale w~niektórych innych
językach jest akurat na odwrót. Gdy na drodze naszego marszu
znajduje się góra, to w~pewnych językach widzimy jej przód, a~w
innych widzimy właśnie jej tył. Byłbym bardzo ostrożny w~kierowaniu się wskazówkami z~języków etnicznych przy formułowaniu
tez na temat wpływu struktury gramatycznej i~leksykalnej języka na
formowanie się reprezentacji przestrzennych. Jestem minimalistą,
jeśli chodzi o~interpretacje tez o~relatywizmie i~determinizmie
językowym: poszczególne języki gramatykalizują różne rodzaje
informacji, uważam jednak (dogmatycznie), że każdy rodzaj
informacji wyrażony w~jednym języku może zostać przekazany (być
może z~użyciem innych środków) w~każdym innym języku.

\subsection{3.3. Uwagi dydaktyczne}

Poglądy dotyczące tego, jak powinna wyglądać dydaktyka matematyki
zmieniały się nie tylko pod wpływem rozwoju samej matematyki.
Istotną rolę odgrywały w~procesie tych zmian także inne czynniki:
ustalenia na temat rozwoju osobniczego, zmiany technologiczne,
naciski polityczne, uprzedzenia itd. Mateusz Hohol pisze krótko o~nauczaniu geometrii w~rozdziale pierwszym książki, kończąc swoje
uwagi apelem, by dydaktycy matematyki pilnie śledzili to, co nauki
kognitywne mają do powiedzenia o~poznaniu matematycznym.
Oczywiście apel ten jest rozsądny i~godny zauważenia. Może jednak
nie wszystko, co mówi się na temat matematyki i~jej dydaktyki w~naukach kognitywnych przekłada się na skuteczne rady edukacyjne
(mam na myśli np. niektóre zalecenia tzw. ucieleśnionej matematyki
w~wydaniu Lakoffa i~N\'{u}\~{n}eza).

Ponadto nawet szlachetne w~swych intencjach zalecenia mogą być
bądź niemożliwe do realizacji, bądź wręcz szkodliwe. Znanym
przykładem jest spektakularne fiasko programu {\em New Math}
usilnie propagowanego w~latach sześćdziesiątych i~siedemdziesiątych XX wieku, ale mamy też wcześniejsze przykłady,
jak choćby propozycja George'a Halsteda z~1904 roku nauczania
geometrii w~oparciu o~system aksjomatyczny Davida Hilberta z~{\em
Grundlagen der Geometrie}, która również nie odniosła sukcesu.

Nie miejsce tu, aby przedstawiać możliwości nauczania geometrii i~oceniać różne sposoby nauczania. Geometria Euklidesa była i~jest
nauczana w~szkole na różne sposoby i~z~różnych perspektyw. Na
marginesie dodam, że samemu Euklidesowi przypisuje się autorstwo
(zaginionej, wspomnianej jedynie przez Proklosa) księgi {\em
Pseudaria}, która zawierała przykłady niepoprawnych dowodów,
zebrane ku przestrodze uczących się matematyki \parencite{acerbi_euclids_2008}.
Powiem jednak parę słów o~czym innym, a~mianowicie o~programie
kursów matematycznych proponowanych na studiach kognitywistycznych
w~Polsce. W~kilku ośrodkach prowadzi się jednosemestralny kurs
{\em Matematyczne podstawy kognitywistyki}, w~Warszawie kurs trwa
(słusznie!) dłużej. Podczas jednego semestru udaje się (mówię
teraz o~Poznaniu) omówić w~elementarnym zakresie kawałeczek
matematyki dyskretnej (zbiory, relacje, funkcje, zliczanie
obiektów, dowody przez indukcję, proste pojęcia algebraiczne) oraz
wprowadzić w~podstawy analizy (zbieżność ciągów, granica, ciągłość
i~pochodna funkcji jednej zmiennej rzeczywistej, badanie funkcji,
czasem też całka Riemanna, o~ile starczy czasu, np. nie przepadną
wykłady mające się odbyć akurat w~dni będące świętami
katolicko-państwowymi). Trzeba też dodać, że omawiany materiał
jest z~konieczności częściową powtórką tego, co mówiono w~szkole,
ponieważ wielu studentów wybierających kognitywistykę nie ma
dobrych wspomnień z~lekcji matematyki w~szkole. Prowadząc w~Poznaniu te zajęcia wedle przedłożonego mi syllabusa wielce
żałuję, że nie mogę w~danych ramach czasowych opowiedzieć o~nieco
bardziej zaawansowanych sprawach (np. o~równaniach różniczkowych),
których znajomość byłaby niezbędna dla rozumienia, co robi się we
współczesnych badaniach kognitywnych. Przede wszystkim jednak
uważam, że obecny syllabus jest ułomny z~powodu braku w~nim treści
geometrycznych i~topologicznych. W~przygotowywanym podręczniku
chciałbym rozszerzyć program, dodając właśnie te treści. Między
innymi dlatego z~ciekawością przeczytałem książkę Mateusza Hohola
{\em Foundations of Geometric Cognition}. Mogę z~przekonaniem
polecić jej lekturę osobom zainteresowanym poznaniem
matematycznym.




%-------------------------




\selectlanguage{english}
\vspace{5mm}%
\begin{flushright}
{\chaptitleeng\color{black!50}{Geometric cognition\\from a~cognitive point of view}}
\end{flushright}

%\vspace{10mm}%
{\subsubsectit{\hfill Abstract}}\\
{Tekst poświęcony jest omówieniu treści nowej książki Mateusza
Hohola {\em Foundations of Geometric Cognition}. Dotychczas o~poznaniu matematycznym pisano w~naukach kognitywnych odwołując się
głównie do zdolności numerycznych, recenzowana książka stanowi
zatem istotne uzupełnienie wcześniejszych rozważań kognitywistów.
Jej zaletą jest uwzględnienie dużej liczby wyników badań
eksperymentalnych, a~także formułowanie propozycji teoretycznych
wystrzegających się radykalnych, jednostronnych rozwiązań. Autor
twierdzi, że źródłem kompetencji geometrycznych są dwa
filogenetycznie zakorzenione rdzenne systemy poznawcze, dotyczące
rozpoznawania obiektów oraz układu otoczenia. Ponadto
współdziałanie tych systemów wzmacniane jest w~trakcie
indywidualnego rozwoju, wspomaganego używaniem języka oraz innych
inwencji kulturowych. Oprócz omówienia treści książki niniejszy
tekst zawiera uwagi inspirowane jej lekturą, dotyczące historii
geometrii, reprezentacji przestrzennych oraz nauczania matematyki.}\par%
\vspace{2mm}%
{\subsubsectit{\hfill Keywords}}\\%
{lorem, ipsum, dolor, sit, amet.}%

\selectlanguage{polish}




\end{newrevplenv}
