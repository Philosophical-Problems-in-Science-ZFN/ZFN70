\begin{newrevplenv}{Jacek Rodzeń}
	{Teologia nauki --~skazana na sukces?}
	{Teologia nauki --~skazana na sukces?}
	{Teologia nauki --~skazana na sukces?}
	{Uniwersytet Jana Kochanowskiego w Kielcach, Instytut Historii}
	{Michał Heller, \textit{Nauka i~Teologia --~niekoniecznie tylko na jednej planecie}, Copernicus Center Press, Kraków 2019, ss.~128.\label{rodzen_anfang}}





\lettrine[loversize=0.13,lines=2,lraise=-0.03,nindent=0em,findent=0.2pt]%
{Z}{}~wprowadzenia do omawianej książki wynika, że temat teologii nauki ,,drążył'' umysł Michała Hellera od początków lat 90. XX wieku
%\label{ref:RNDXUBPt0mgdk}(Heller, 2019, s.~8).
\parencite[][s.~8]{heller_nauka_2019}. %
 W~istocie, tekst, inicjujący zainteresowanie nowym obszarem refleksji teologicznej, nie pochodzi z~jego książki \textit{Nowa fizyka i~nowa teologia}, wydanej w~1992 roku, tylko wydanego dziesięć lat wcześniej artykułu, zamieszczonego w~czasopiśmie teologicznym \textit{Communio}
%\label{ref:RNDpdOrYXhnfs}(Heller, 1982).
\parencite[][]{heller_stworzenie_1982}. %
 W~artykule tym Heller po raz pierwszy zamieścił paragraf zatytułowany ,,Program teologii nauki''. Do poruszonych tam zwięźle zagadnień nie powracał jednak w~obszerniejszych opracowaniach, poza ,,nieco rozszerzoną wersją'' 
%\label{ref:RNDjSLVZIE0mB}(Heller, 2019, s.~8)
\parencite[][s.~8]{heller_nauka_2019} %
 fragmentu artykułu z~\textit{Communio} opublikowaną dwukrotnie w~języku angielskim 
%\label{ref:RNDOtu93pRgu9}(Heller, 1996, 2003).
\parencite[][]{heller_program_2003}.%


Warto też dodać, że w~czasie II Seminarium Interdyscyplinarnego w~Castel Gandolfo w~1982 roku, z~udziałem Ojca Świętego Jana Pawła II, swoją refleksją na temat możliwych perspektyw dla ,,nowej dyscypliny badawczej'' podzielił się również Józef Życiński
%\label{ref:RNDsCfjKR9G2U}(1984).
\parencite*[][]{zycinski_w_1984}. %
 Zaznaczył on jednocześnie, iż ową ,,nową dyscyplinę'' Michał Heller nazwał ,,teologią nauki''. Wynika stąd, że obaj filozofowie, znani skądinąd ze współpracy przy wielu wspólnych przedsięwzięciach pisarskich, musieli już na początku lat 80. XX wieku dyskutować ze sobą o~nowym obszarze refleksji teologicznej 
%\label{ref:RNDuyAbug4DjO}(więcej na ten temat zob. Polak, 2015).
\parencite[więcej na ten temat zob.][]{polak_teologia_2015}.%


W~najnowszej ekspozycji projektu teologii nauki, zatytułowanej \textit{Nauka i~Teologia --~niekoniecznie tylko na jednej planecie}, Heller stara się dać odpowiedź na szereg nurtujących w~tym przypadku pytań, na które w~dotychczasowych, krótkich opracowaniach, nie było po prostu zbyt wiele miejsca. Jego książka składa się niejako z~dwóch segmentów. Pierwszy dotyczy kwestii będących wprost przedmiotem zainteresowania teologii nauki (wprowadzenie, paragraf~2.2., rozdział~6). Drugi, kwestii towarzyszących nowemu polu badawczemu, bez których projekt teologii nauki mógłby zostać niewłaściwie odczytany (rozdział~1, paragraf~2.1. rozdziały~4 i~7).

We ,,Wprowadzeniu'' Heller słusznie zauważa, że w~każdej nowej dyscyplinie naukowej nie można wszystkiego zaplanować z~góry, a~jej rozwój odbywa się głównie przez tworzenie nowych sytuacji problemowych
%\label{ref:RND1nfcTD6jKb}(Heller, 2019, s.~16).
\parencite[][s.~16]{heller_nauka_2019}. %
 Wydaje się, że oprócz zagadnień omawianych w~pierwszym segmencie prezentowanej książki, takie sytuacje problemowe można odnaleźć w~rozdziałach 3 i~5. Pierwsza dotyczy odniesienia teologii do zmieniających się obrazów świata przyrody. Druga, konieczności zrewidowania ze strony teologii jej własnego --~jak to Heller nazywa --~chrysto-geocentrycznego zorientowania. Chodzi w~tym przypadku o~kwestię włączenia w~dyskurs teologiczny możliwości istnienia rozumnych istot pozaziemskich oraz, łączącego się z~tą ostatnią perspektywą, problemu jednorazowości ziemskiego wcielenia Jezusa Chrystusa
% (na ten temat por. Rodzeń, 2016).
\parencite[na ten temat por.][]{rodzen_2016}.

W~pierwszym, wspomnianym powyżej segmencie książki Hellera, można odnaleźć zarówno wymienione we wcześniejszych jego pracach, jak i~nowsze tematy, które powinny zostać objęte zainteresowaniem teologii nauki. Są wśród nich: kwestia poznawalności (racjonalności) świata przez nauki matematyczno-empiryczne, problem interpretowania doktryny chrześcijańskiej z~uwzględnieniem szeroko pojętej wiedzy naukowej, kwestia analogiczności tajemnic naukowych w~stosunku do tajemnic teologicznych oraz zagadnienie aksjologicznego wymiaru nauki i~pracy naukowej. Można jednak odnieść wrażenie, że wymienione tematy zostały jedynie zasygnalizowane i~krótko omówione.

Natomiast szerzej został rozwinięty drugi segment \textit{Nauki i~Teologii}…, w~którym Heller omówił szereg kwestii o~charakterze bardziej metodologicznym, które zbiorczo można określić mianem meta-teologii nauki. Są to kwestie bardzo istotne, gdyż gwarantują nie tylko właściwe rozumienie sensu rozwijania samej teologii nauki, lecz także dają pewne wskazówki, jak takie przedsięwzięcie prowadzić. Można je nazwać warunkami brzegowymi dla sensownego uprawiania teologii nauki. Warto je w~sposób syntetyczny wymienić: (1) otwartość teologii chrześcijańskiej (dostrzeżenie w~nauce nowego \textit{locus theologicus}), (2) umiejętność korzystania z~dorobku filozofii i~historii nauki (teologia nie może rozciągać swojej refleksji bezpośrednio nad nauką), (3) niekonfliktowe oddziaływanie na siebie teologii i~nauki (teologia nie może pozostawać w~separacji wobec nauki), (4) czerpanie inspiracji z~teologicznej prawdy o~stworzeniu świata (teologiczna wymowa badalności świata przez naukę oraz charakteru informacji, jaką można o~nim pozyskiwać), (5) umiejętność korzystania z~dorobku różnych nurtów teologii filozoficznej (wzbogacenie problematyki teologii nauki i~wyostrzenie jej metod), (6) gotowość do przyjęcia perspektywy poznawczej wykraczającej poza ziemski horyzont życia i~aktywności intelektualnej (objęcie refleksją teologiczną także dzieła innych, aniżeli istoty ludzkie, form życia rozumnego w~kosmosie).

Do powyższych punktów należy jednak dorzucić jedno, jak sądzę istotne, zastrzeżenie. Nie będą one miały większego znaczenia w~sytuacji, gdy podważy się racjonalność i~wartość poznawczą tak teologii, jak i~nauk przyrodniczych. Niestety, dominujące tendencje i~klimat we współczesnej dyskusji wokół tych dwóch obszarów aktywności intelektualnej i~badawczej nie napawają do optymizmu. Wystarczy wspomnieć o~dużym zainteresowaniu, z~jednej strony nurtem tzw. nowego ateizmu (który deprecjonując wszelki dyskurs teistyczny jako nieracjonalny, podważa zarazem fundamenty teologii)
%\label{ref:RNDHSYO2Jh2Qs}(zob. Haught, 2008),
\parencite[zob.][]{haught_god_2008}, %
 z~drugiej konstruktywizmu społecznego w~historii i~filozofii nauki (traktującego metody i~wyniki nauki jako jedynie efekt gry czynników społeczno-politycznych). Dlatego --~jak się wydaje --~sprawa polemiki z~wiodącymi przedstawicielami obydwu powyższych nurtów myślowych powinna być ważna dla każdego, kto uprawia teologię, nie wyłączając oczywiście zwolenników rozwijania projektu teologii nauki jako dyscypliny teologicznej. Należy przy tym pamiętać, iż zwłaszcza prominentni reprezentanci wspomnianych dwóch nurtów, niezależnie od wartości kreowanych przez siebie narracji, dysponują dziś zazwyczaj znacznymi środkami finansowymi, zdobywają światowe rynki wydawnicze i~– chciałoby się powiedzieć ,,niestety'' --~coraz liczniejsze kampusy uniwersyteckie.

W~kontekście omówienia książki Michała Hellera poświęconej nowej dyscyplinie teologicznej, warto przy tej okazji postawić pytanie: czy w~dotychczasowej światowej literaturze przedmiotu dostrzeżono potrzebę teologii nauki, a~może nawet jest ona gdzieś owocnie rozwijana. Jak zwykle w~takich przypadkach, mimo braku wyraźnych odwołań do wyrażenia ,,teologia nauki'', teologiczna refleksja nad tak czy inaczej pojętą nauką była i~jest przedmiotem zainteresowania wielu badaczy, zarówno teologów, jak i~nie-teologów. Są to jednak zwykle próby fragmentaryczne, niepretendujące do miana systematycznie rozwiniętych i~dogłębnych.

Mimo to, jak się wydaje, jedynym --~oprócz Hellerowskiego i~jego uczniów --~rozwiniętym projektem teologii nauki, a~właściwie ,,teologii przedsięwzięcia naukowego'' (\textit{theology of scientific endeavour}), jest praca amerykańskiego fizyka i~teologa w~jednym Christophera B. Kaisera
%\label{ref:RND0UizPS3hlm}(2007).
\parencite*[][]{kaiser_toward_2007}. %
 Z~racji interdyscyplinarnego przygotowania tego autora (który ma na swoim koncie doktoraty z~astrofizyki i~teologii dogmatycznej), istnieje zasadne przekonanie o~rzetelności prowadzonych przez niego analiz. Zresztą Kaiser już dwadzieścia lat wcześniej dał się poznać jako wytrawny znawca problematyki relacji nauki przyrodnicze–religia, pisząc znakomitą monografią historyczną nt. wpływu chrześcijańskiej teologii stworzenia na koncepcje przyrody 
%\label{ref:RNDup7W6wps2l}(Kaiser, 1997).
\parencite[][]{kaiser_creational_1997}.%


Według Kaisera teologia nauki rodzi się nie przez konfrontowanie wyników nauki z~prawdami tradycji biblijnej (sam Kaiser jest członkiem amerykańskiego Kościoła reformowanego), a~przez refleksję nad uwarunkowaniami (\textit{preconditions}) umożliwiającymi badania naukowe. W~swojej książce wymienia cztery takie uwarunkowania: prawidłowości fizyczne w~przyrodzie, zdolność umysłu ludzkiego do ich odczytywania, wreszcie uwarunkowania kulturowe i~społeczne pracy umysłowej uczonych. Zdaniem Kaisera sama nauka nie jest w~stanie wyjaśnić głębokiego znaczenia tych uwarunkowań. I~w tym momencie pojawia się odniesienie do tradycji biblijnej, dając impuls do rozwinięcia teologii nauki. Kaiser pisze: ,,[N]asze próby zmuszenia nauki do wyjaśnienia jej własnych podstaw angażują nas w~dyskurs teologiczny i~prowadzą do kolejnych pytań teologicznych. Wysiłek naukowy wtapia się w~wysiłek teologiczny bez żadnej wyrwy między nimi. To, co budujemy, przypomina bardziej tunele, aniżeli mosty''
%\label{ref:RNDOrXapZ5O3M}(Kaiser, 2007, s.~124).
\parencite[][s.~124]{kaiser_toward_2007}. %
 Łatwo zauważyć pewną bliskość niektórych ujęć Kaisera z~wątkami pojawiającymi się w~\textit{Nauce i~Teologii}… Hellera. Wystarczy wspomnieć rozrzucone w~różnych miejscach tej ostatniej pracy uwagi dotyczące poznawalności świata przyrody oraz uwarunkowań poznawczych człowieka.

Na zakończenie jeszcze tylko jedna uwaga. Teologia nie jest tylko ,,naukowo zorganizowaną refleksją nad wiarą religijną i~jej treścią''
%\label{ref:RNDAMhbGkzouj}(Heller, 2019, s.~23).
\parencite[][s.~23]{heller_nauka_2019}. %
 Oprócz wymiaru intelektualno-poznawczego nosi w~sobie także wymiar praktyczno-zbawczy. Chodzi o~to, że we wspólnocie wierzących teologia pełni również rolę służebną, stanowiąc dla jej członków drogowskaz na drodze uświęcenia i~zbawienia (wymiar soteriologiczny). W~związku z~tym uwaga dotyczy po części tego, o~czym Michał Heller mimochodem wspomniał w~swoim wczesnym wprowadzeniu do zagadnień teologii nauki, zawartym w~książce \textit{Nowa fizyka i~nowa teologia} 
%\label{ref:RNDiFv8LccA97}(Heller, 1992).
\parencite[][]{heller_nowa_1992}.%
 Wspomniał wtedy o~pokrewieństwie własnego projektu do refleksji z~zakresu tzw. teologii ,,wartości ziemskich''. Z~kolei w~ujęciu teologii nauki Józefa Życińskiego z~roku 1982 pojawiło się m.in. odniesienie do prac Pierre’a Teilharda de Chardin. Wśród teologów panuje opinia, iż obok Teilharda de Chardin, tacy wybitni frankofońscy teologowie XX wieku, jak Gustave Thils i~Marie-Dominique Chenu są współtwórcami obszaru refleksji nazywanego teologią rzeczywistości ziemskich 
%\label{ref:RNDFGZa0O3Vst}(zob. np. Thils, 1946, 1949; Graczyk, 1992).
\parencites[zob. np.][]{thils_theologie_1946}[][]{thils_theologie_1949}[][]{graczyk_francuska_1992}. %
 Wspomniani autorzy w~swoich pracach traktują różne aspekty aktywności chrześcijan jako formę uczestnictwa w~dziele tworzenia, uświęcania i~zbawiania świata (\textit{constructio et consecratio mundi}). Można sądzić, iż niektóre przemyślenia i~rozwiązania z~zakresu teologii rzeczywistości ziemskich byłyby przydatne także na gruncie teologii nauki (szczególnie w~aspekcie aksjologicznym i~prakseologicznym). Wzbogaciłyby tym samym projektowaną dyscyplinę i~obok jej wymiaru intelektualistycznego, wprowadziłaby także elementy związane z~szeroko pojętą aktywnością badawczą.

Propozycją projektu teologii nauki Michał Heller zawiesił poprzeczkę jeszcze wyżej, aniżeli innymi, wcześniejszymi przedsięwzięciami, dotyczącymi relacji nauka-religia czy nauka-teologia. Powstałe zadanie jest niezwykle ambitne, wymagające, ale także obarczone pewnym ryzykiem. Ponoć jednak w~nauce, w~tym także w~teologii, ryzyko czasami bardzo się opłaca.




\selectlanguage{english}
\vspace{5mm}%
\begin{flushright}
{\chaptitleeng\color{black!50}{Theology of science --~doomed to success?}}
\end{flushright}

%\vspace{10mm}%
{\subsubsectit{\hfill Abstract}}\\
{The paper briefly presents a genesis of the research project called the theology of science initiated by Michał Heller in early 1980s, but also enriched by the remarks of Józef Życiński. Then the main features of the new Heller’s book \textit{Science and theology -- not necessarily exclusively on one planet} (Kraków, \cite*{heller_nauka_2019}) are discussed. The book develops and strengthens the early ideas of theology of science. The paper draws also attention to another author---C.B.~Kaiser who---since a dozen years---has been developing a similar project: theology of scientific endeavor. It is also suggested that the Heller’s project, especially in its axiological and praxeological aspects, could take advantage of the achievements of several Francophone authors (\textit{i.a.}, G.~Thils, M.-D.~Chenu) who once developed a field of the so-called theology of earthly realities.}\par%
\vspace{2mm}%
{\subsubsectit{\hfill Keywords}}\\%
{theology of science, rationality of science, theology of earthly realities.}%

\selectlanguage{polish}

\end{newrevplenv}
\label{rodzen_ende}