\begin{artengenv}{Anna Sarosiek}
	{The role of biosemiosis and semiotic scaffolding in the processes of developing intelligent behaviour}
	{The role of biosemiosis and semiotic scaffolding in the processes\ldots}
	{The role of biosemiosis and semiotic scaffolding in the processes of developing intelligent behaviour}
	{Pontifical University of John Paul II in Krakow}
	{Biosemiotics deals with the processes of signs in all dimensions of nature. Semiosis is the primary form of intelligence. Intelligent behaviour becomes immediately understandable in this approach because semiosis combines causality with the triadic structure of the semiotic sign. Intelligence is a~process created in a~given context. In the course of evolution organisms has learned to create increasingly sophisticated internal representations of external state. Semiosis is the precursor of the emergence of a~feature we consider intelligence. Biosemiotics also draws attention to the distributed intelligence, which relies on external semiotic scaffoldings as much as on the subject’s abilities and knowledge.}
	{biosemiotics, semiosis, biosemiosis, semiotic scaffolding, intelligence, cognitive activity, functional circles, Jakob von Uexküll.}





\section*{Introduction }
Biosemiotics is a~field of science connecting biology and semiotics. The primary purpose is to show that semiosis is an essential aspect of life. Biosemiotics aims to build a~bridge between biology, philosophy, linguistics and communication research. The main challenge of biosemiotics is to increase knowledge of biological information: recognition and interpretation of organic codes as fundamental elements of the living world
%\label{ref:RNDdqbtpjMK0T}(Barbieri, 2005).
\parencite[][]{barbieri_life_2005}. %
 Thomas Sebeok applied biosemiotics to investigate the biological roots of human semiosis. He tried to understand the organic world as a~world of signals, signs, communication and language. Some biologists have begun to see how many phenomena and functions at the centre of organic life (genetic code, cell metabolism, ecosystem activities) are semiotic mechanisms 
%\label{ref:RNDhZlZ9AKzIY}(Anderson et al., 1984).
\parencite[][]{anderson_semiotic_1984}. %
 The signs and meanings play a~fundamental role in all activities of living systems. Semiosis is an indispensable feature of all life forms’ ability to accumulate, replicate, transmit and make sense of messages. The study of these communication processes and the meaning they produce can be considered a~life science discipline related to both nature and culture 
%\label{ref:RNDr7Hwb6LkCW}(Sebeok, 1991, p.22).
\parencite[][p.22]{sebeok_semiotic_1991}.%


Biosemiotics researchers accept the complexity of life processes as a~collection of data provided by biological sciences. Nevertheless, they consider also behavioural research. Biosemiotics deals with the transformation of signs in all dimensions studied, including the appearance of semiosis in nature, which can predict the features of living cells, the natural history of signs, aspects of semiosis in the ontogenesis of organisms, in plant and animal communication, and the functions of signs in immune and nervous systems as well as the semiotics of cognition and language
%\label{ref:RNDAID5Cb7So1}(Emmeche, 1992, p.78).
\parencite[][p.78]{emmeche_modeling_1992}. %
 Biosemiotics is perceived as a~contribution to the general theory of evolution, based on a~synthesis of various disciplines. The field proposes a~new approach to the phenomenon of life, considering the importance and the functions of a~single ribosome and ecosystem and the beginning of life and its meaning. The biosemiotic concept assumes the sphere of life is filled with sign processes, which result in the individual creation of meanings like food, the need to escape, and sexual reproduction. Jasper Hoffmeyer argues that semiotic requirements are a~condition for the success of species and that organic evolution is the evidence of the development of sophisticated semiotic means necessary for survival. The most visible feature of organic evolution is the plurality of morphological structures, as well as the development of ``semiotic freedom'', which means a~significant increase in ``richness or depth of meaning 
%\label{ref:RNDPentkhE2Nr}(Hoffmeyer, 1996, p.61).
\parencite[][p.61]{hoffmeyer_signs_1996}.%
''

In the research, the role of the observer of the system is equally important. One tries to understand the nature of observed phenomena to describe and conceptualize it. This position presupposes the possibility of conducting a~scientific experiment and observation and to measure various systems. These issues have been studied many times in the philosophy of science. Nevertheless, the research of living systems is developed by biologists and systems scientists
%\label{ref:RNDJUxAdYTWhF}(Kampis, 1991; Pattee, 1989, pp.63–78; Rosen, 1978; Uexküll, 1984).
\parencites[][]{kampis_self-modifying_1991}[][pp.63–78]{pattee_simulations_1989}[][]{rosen_fundamentals_1978}[][]{uexkull_semiotics_1984}. %
 The goal of biosemiotics is to identify assumptions, that are applied to the teleological concepts of biology: function, information, code, signal, cue and to provide them with a~theoretical grounding. Such terms in biology cannot be avoided or replaced by logical, chemical or mathematical notation. Therefore, biosemiotics aims to consolidate these terms in the physical and biological context. They try to define and relate them to evade the anthropomorphisms that are hidden assumptions of human science. Biosemiotics uses very carefully concepts that are regularly adopted to describe the evolutionarily complex product of semiotic processes as human culture. The efforts are made to avoid falsities of anthropocentrism or vitalism and distinguish which of these concepts are appropriate to the precise level of research 
%\label{ref:RNDHVKdE0q0Ey}(T?nnessen, 2013).
\parencite[][]{tonnessen_biosemiosis_2013}.%


The history of semiotics is deeply rooted in structuralism and linguistics. Biosemiotics is related to a~theoretical biology. Relations, meaning, wholeness and contextuality allow to perceive living creatures as active systems of sign production, sign mediation and sign interpretation. This approach explains the essence of behaviour and a~life filled by intentionality, self-awareness and sense, which are deeply connected with intelligence. Biosemiotics is the science of sign processes, which gives the instruments for scientific research and the study of these features.

The intelligence research is struggling with the problem of the lack of a~clear definition of the term ‘intelligence’. Consideration of intelligence is based on the properties of an organism that can be measured. Definitions of the term ‘intelligence’ are made by finding paradigmatic examples and creating lists of properties. These are characteristics that describe the properties and functions of living beings’ actions leading to effective action in the world. Intelligent behaviour is defined as adequate and effective behaviour leading to the achievement of a~specific goal. However, there is no equivalent definition, because the concept of intelligence is vague and inaccurate. Colloquial and scientific descriptions of intelligence are multiplying. However, in the current approach, it is defined as the ability to think, infer, make decisions, recognize patterns, predict the effects of one’s actions, the ability to remember, learn, problem-solving, perception, but also the ability to use language, motor skills, use of intuition, being creative and conscious, i.e. general coping with the world
%\label{ref:RNDBPtBfYkfR3}(Rosen, 1978).
\parencite[][]{rosen_fundamentals_1978}. %
 This work tries to justify the concept of intelligence in a~biosemiotic context, which is broader than human
%\label{ref:RND4UZRqvH0L3}(Byrne, 2004; Goldstein, Princiotta and Naglieri, 2014; Pfeifer and Bongard, 2006; Spearman, 1904; Thorndike, 2017)
\parencites[][]{byrne_2004}[][]{goldstein_2015}[][]{pfeifer_2006}[][]{spearman_1904}[][]{thorndike_animal_2017} %
 or artificial intelligence AI 
%\label{ref:RNDpjVwUVRBFp}(Brooks, 1986; Kurzweil et al., 1990).
\parencites[][]{brooks_1986}[][]{kurzweil_1990}. %
 %Brooks, 1986; Huw, n.d.; Kurzweil et al., 1990
Thus, the concept of ‘intelligence’ will be presented as a~result of semiosis.

\section*{The conceptual problem of intelligence}
The ways of describing intelligence are changing over time. The common search for characteristic properties and specific determinants tries to justify intelligent behaviour. Arthur Jensen, one of the leading psychologists studying human intelligence, hypothesized that all human beings share the same intellectual mechanisms, and that differences in intelligence are related to ``quantitative biochemical and physiological conditions
%\label{ref:RNDjbalXtcVjE}(Jensen, 1998).
\parencite[][]{hoffmeyer_unfolding_1998}.%
'' He meant skills such as attention, perception, generalization, learning, memory, language, thinking, and problem-solving. In addressing the subject of the intelligent behaviour of living organisms, we do indeed reflect on the mechanisms underlying such action. First, the properties that seem to be crucial for the phenomenon of intelligence are distinguished. It is not considered as a~whole of behaviour but attempts to isolate those features that seem important and decisive for intelligent action.

Intelligence as the body’s ability to react in complex ways to environmental stimuli (as in biosemiotics) is closely related to the ability to think. It is a~Cartesian legacy of Western culture. For Descartes, there were two separate substances: physical and mental
%\label{ref:RND3PSZmfafdq}(Descartes, 1637).
\parencite[][]{descartes_discours_1637}. %
 This division gave rise to the historical problem of how mutual relations between the two systems---body and mind---can proceed since each of them can exist without the other. One of the main challenges posed by this dilemma is how a~thought, or something that happens in an immaterial mind, can affect the material body. To this day, the problem of substantial dualism is considered in modern science as a~mind-body problem. Descartes also argued that the bodies of animals and humans are simple mechanisms because they are subjected to external factors and act by mechanical forces 
%\label{ref:RNDnLqvAmGGx2}(Descartes, 1641).
\parencite[][]{descartes_renati_1641}. %
 The premise of Cartesian dualism is that the material body has certain physical properties and their description is provided by mathematical natural science. Only the ability to think, which belongs only to man, allows for rational and conscious management of one's own body. Self-awareness, on the other hand, is the result of a~way of understanding oneself through causality or logic and a~condition of cognition. This view made the human mind perceived as a~peculiar value and all other beings are measured by its measure.

Most people would probably agree that mental phenomena such as thinking come from processes in the brain and that there is the mind that controls these kinds of actions. However, the description of mental activity presents some fundamental problems. So far, this type of mind operation has not been comprehensively described. While we all have a~fairly clear idea of the term ‘thinking,’ it remains poorly defined. Thinking is understood as the possibility of inducing, inferring, looking for analogies, performing calculations and recognizing patterns, largely coincides with intelligent behaviour in the description.

Deep Blue met the requirements described above and performed the expected actions, but it is still not called intelligent
%\label{ref:RND5FmPuOtJgo}(Deep Blue, 2012).
\parencite[][]{noauthor_deep_2012}. %
 Similarly, other programs that mimic the human cognitive mechanisms also do not get the term ‘intelligent’. Interesting examples are expert systems that discover patterns and find the best possible solutions. However, a~diagnostic program will not be considered smarter than the average doctor, even if its knowledge base is larger. Its work is faster and its diagnoses are more accurate. The ability to solve problems is always placed high among the characteristics of intelligent organisms. It is related to the previously acquired knowledge and the possibility of its creative use. Programs recognize emotions, chatbots build complex and grammatically correct sentences. Some programs create images and music. Moreover, the capabilities of artificial systems, especially those related to computational capabilities, often far exceed human skills.

The early AI research methods were based on the consideration of high-level symbolic representations. The ability of logical reasoning considered manifestations of intelligence: the ability to play chess, solving mathematical problems or deduction. Research on artificial intelligence was based on colloquial or psychological understanding. Colloquially, intelligence is usually considered to be the ability to solve practical problems, language skills or social competencies. In psychology, it was most often considered an efficiency that determines the effectiveness of actions, using cognitive processes, which is controlled by the superior ability of the mind, corresponding to the mental level of the individual, and special abilities, responsible for the efficiency of action in specific areas or types of tasks. The research program has always been based on the current knowledge and beliefs about the phenomenon of intelligence.

\section*{The fundamental importance of semiosis}
Semiosis distinguishes living systems from inanimate ones
%\label{ref:RND7hry0nIuRm}(Sebeok, 1988b).
\parencite[][]{sebeok_communication_1988}. %
 Friedrich Rothschild argues that ``living systems formed from the very beginning as signs systems 
%\label{ref:RNDbsTmIxbWoQ}(Rothschild, 1962).
\parencite[][]{rothschild_laws_1962}.%
'' The significant and necessary premise of biosemiotics is the appearance of meaningful communication in species other than \textit{Homo sapiens}. Semiosis is the process of creating, receiving, and interpreting characters. Any form of activity, behaviour or process that involves signs is an aspect of semiosis. Semiosis is a~process that produces and carries meaning. It also mediates the relationship of purposefulness and causality, because it allows to search and connect hidden meaning.

The idea of semiosis was originally developed to relate language to other sign systems, both human and non-human. Moreover, the prototype of semiotic theories is language. Processes of semiosis can be applied to other sign systems (but then sign systems already presuppose semiosis…)
%\label{ref:RNDxsCcfmjwwC}(Noth, 2002).
\parencite[][]{noth_semiotic_2002}. %
 Some hypotheses describe meta signal systems and interpret language as one of the various codes for communicating meaning. Such a~perspective is predefining semiosis as an act of the communication of meaning established within the relationship of signs 
%\label{ref:RNDoqfKrONq0d}(Bateson, 1966; Pavlov, 1927).
\parencites[][]{bateson_information_1966}[][]{pavlov_conditioned_1927}.%


Semiosis is a~broad phenomenon concerning social interaction systems and its significant aspect of information exchange. Greater or lesser extent semiotic exist in nature as the linguistic message, ways of thinking, emotional reactions, beliefs, motives and goals. In semiosis the processing of information associated with the emergence of meaning and causality is equally important. In the biological context, systems that transmit, acquire, assimilate, decode and manipulate information, is generating a~meaning. The emergence of information implies thinking. To obtain information begins a~cause-and-effect process. It is the causative factor that ensures that one event is linked to another in a~chain of thought
%\label{ref:RND432gOW9NnM}(Pharoah, 2020).
\parencite[][]{pharoah_2020}.%


All the aspects of semiosis described above are related to the conditions for the emergence of intelligent behaviour. Emotional states, feelings and beliefs encoded in the message are transmitted the same way as essential information about objects and situations. The transformation of signs develops dynamically in the process of communication. The reception, interpretation and production and their passing of signs are neither chaotic nor random. The answer is more than a~stimulus or a~response. Among the many possibilities of interpreting the sign, the body chooses the optimal one for itself. Signals are transmitted to evoke a~specific reaction in the process of semiosis, which some extent requires a~proper attitude to the received signals.

Observations of the behaviour of living organisms show that the simplest forms of life use a~system of signs. It is especially apparent among animals that inform other members of the herd where they found food, announce the time of fertility to other individuals of the same species, and warn themselves of threats. Messages take various forms: chemical, audible, visual or tactile transmitted as individual signals or a~complex of them
%\label{ref:RND6q7iEGUfQD}(Sebeok, 1969).
\parencite[][]{sebeok_semiotics_1969}. %
 Sebeok suggests that endosymbiosis, self-reference, receptor functions, autopoiesis, and other living system properties align with the definition of semiosis: something is alive and communicates meaning 
%\label{ref:RNDQD071qwpmb}(Sebeok, 1988a, p.72).
\parencite[][p.72]{sebeok_animal_1988}. %
 It does not convey discrete information but gives it a~deeper meaning. The emergence of meaning is the beginning of intelligent behaviour, because it evokes suitable actions in the particular context of environment.

The course of the semiosis process also requires evaluation. Organisms receive enormous amounts of sensory data from the environment as well as from their interior. Totally number of signs processes will not run at the same level due to the possibility of sensory overload. The brain cannot handle all the information simultaneously. Therefore, there must be a~classification of the signs in terms of their meaning (for example, related to experience or current need). The signs become operational when distinguished from background noise. Cognitive activity is triggered to interpret the input data and transform it into meaningful information. Kalevi Kull
%\label{ref:RNDgLADMGwMiW}(1998)
\parencite*[][]{kull_semiosis_1998} %
 argues that there is a~repetitive order of semiosis:

\begin{enumerate}
\item the cognitive entity filters environmental data and recognizes signs (based on a~pre-existing model in memory);
\item meaning arises in mind---as a~new structure, but the explicit isomorphism is visible within the original and the new sign;
\item the sign is interpreted as meaningful and matched with existing patterns and their meanings stored in memory;
\item the results of a~successful interpretation are an observable response to perceived stimuli.
\end{enumerate}
Thus, in the beginning, a~recognition process takes place. It begins the semiosis process and is necessary at other levels. The arising meaning is not only recalled from memory but re-created and compared with previous experiences. The intelligent reaction requires the suitable selection of the appropriate interpretation of the sign but can differ significantly from the existing ones.

The subject transforms signs and remains the result of the interpretation---the physical development and the way the memory works are the results of previous semiosis processes. In this perspective, intelligent action appears as a~continuous interpretation of meaning influenced by its historical circumstances. According to Yuri Lotman
%\label{ref:RNDGOxZNS0qW6}(1990, p.101),
\parencite*[][p.101]{lotman_universe_1990}, %
 ``the sign itself is a~program for the creative process.'' Each sign interpreted in mind is already previously interpreted and stored in the memory, and the result of the interpretation influences the process of passing it on as a~sign to subsequent recipients. The components of semiosis are continuously interpreted. It becomes possible to create new and forget old meanings. The creativity of mind can produce new relationships and some strength of ``rewriting'' meaning over pre-existing engrams. The emergence of meaning is the creative work of the organism. During semiosis, relationships arise between things that do not interact or influence each other through direct physical or chemical processes. Such semiotic phenomena do not belong to the physical, but the mental. They are evidence of the intelligent relationship to reality.

All forms of cognition emerge from the interpretation of signs. It is a~result of the structural coupling of the subject and its environment. Those relations must proceed to guarantee the integrity and durability of the organism’s world. The living system structures and the environment are changing as a~result of their interaction. The coupling that arises from their plasticity produces an autonomous and well-defined unit. The relation develops the history of the subject. Cognition is associated with an embodiment, embedding and situated experience because it always includes a~subject defined by physical and mental architecture
%\label{ref:RNDl9bdWwwtAg}(Wilson and Golonka, 2013; Lakoff and Johnson, 2008; Miles, Nind and Macrae, 2010).
\parencites[][]{wilson_embodied_2013}[][]{lakoff_metaphors_2008}[][]{miles_moving_2010}. %
 Cognition immersed in the world defines dynamics events taking place in specific contexts in space and time. Such an individual is capable of learning from experience and adapting his behaviour accordingly. Semiosis allows overcoming the dualism of mind and matter by studying the transformations of signs providing a~better approach to living systems than the dichotomies of mental and physical properties. Semiosis is a~dynamic process, and although the goal of a~signs’ communication accomplished, it can potentially last indefinitely.

Semiosis is a~process, which prefers signs that are meaningful for organisms. During semiosis, intelligent behaviour is an experienced application of functions as transmission and creation of new information. Not entirely predictable data, which not comply with existing patterns, get the ability to preserve and selectively recreate the knowledge. Intelligence does not appear as a~fuzzy concept, but as a~developed ability to use signs and properly use environmental signals and organize knowledge.

\section*{Biosemiosis theory of Jakob von Uexküll}
One of the pioneers of behavioural physiology, ethology and the precursor of biocybernetics, was Jakob von Uexküll (1864-1944). He devoted his work to the research of perception and actions of living creatures. Uexküll has introduced the term \textit{Umwelt} to describe the subjective world of the organism
%\label{ref:RND6WWnDIrGBs}(Uexküll, 1921).
\parencite[][]{uexkull_umwelt_1921}. %
 He developed a~distinctive method he called \textit{Umweltforschung}. The scientific goal of Uexküll was to investigate the behaviour of organisms as subjects in families and groups. Uexküll theory of signs and meaning has the place in all aspects of life processes. The concept of the functional circle (\textit{Funktionskreis}) is the interpretation of the general model of sign process---semiosis. A~description of the functional circle allows understanding how intelligence arises in operations of the living system.

Uexküll strongly emphasized the essential role of the physical architecture of the organism in shaping forms of interaction with the outside world. He tried to create a~new idea based on theoretical foundations. The new conception provided to organize the research and understand how living organisms function in the world
%\label{ref:RNDRParn16QgX}(Uexküll, 1920, p.7).
\parencite[][p.7]{uexkull_theoretische_1920}. %
 Uexküll’s research perspective was different due to the dedication of a~principle recognizing reality as a~``subjective insight'' of a~specific organism 
%\label{ref:RNDAaQkEoF4wd}(Uexküll, 1920, p.9).
\parencite[][p.9]{uexkull_theoretische_1920}. %
 In this way, he interpreted Immanuel Kant’s views presenting objects as phenomena that owe their structure to the subject 
%\label{ref:RNDcoXyfSBOEl}(Uexküll, 1920, p.8).
\parencite[][p.8]{uexkull_theoretische_1920}. %
 The biologist argued that the role of the sensory organs and the central nervous have an enormous role in Umwelt construction. Another goal would be to examine the relationship of animals with the experienced objects of the world. Uexküll was interested in studying the subjective reality of animals. He argued that different species with different sensory feelings experience the world differently.

The physical nature of specific sensory apparatus influences a~particular world view
%\label{ref:RNDvGDzstF58Q}(Uexküll, 1934, p.22).
\parencite[][p.22]{uexkull_streifzuge_1934}. %
 Uexküll believed that one could access different Umwelts via the study of the physical organization of organisms. According to him, this access required immersion in the anatomical structure of the examined organism, which is responsible for the way it interacts with the outside world. Cognition is a~subjective process that takes place in a~species-specific area. The semiotic processes are responsible for the dynamics of cognition by linking the subject with the contextual world. There is a~particular symbolic sphere that contains a~range of possible interactions according to the mode of interaction permitted by the physical architecture of the organisms’ forms. The type of perception also shapes the environment---Umwelt---the significant world.\footnote{German term ‘\textit{Umwelt}’ refers to the environment or surroundings. However, the term used by Uexküll is an interpretation of reality and distinguish from the environment---\textit{Umgebung}. The Umwelt of organisms is the projection of the world created inside them by the same. Umwelt is a~world experienced by an individual organism.} In this Umwelt, specific environmental features become essential and individualized. Furthermore, these peculiarities are perceiving as qualitatively distinct and assigned particular meanings. In this belief about the subjectivity of the experienced world, Uexküll was supported by Ernst Cassirer which argued that all living beings have their circle of action, which is both a~limitation and a~point of view a~specific world 
%\label{ref:RND7tBkh94GJj}(Cassirer, 1996).
\parencite[][]{cassirer_philosophy_1996}.%


Natural systems are pre-equipped with genetic information that guides the organism through different life cycles in different environmental contexts. The genetic data is a~kind of embodied pre-empirical knowledge and inherited memory that allows the system to identify and assign meaning to specific environmental signals and produce the appropriate behaviour as a~response to meet the demands of its \textit{Innenwelt}
%\label{ref:RNDe5nDlXoMyK}(Uexküll, 1921, p.46).
\parencite[][p.46]{uexkull_umwelt_1921}. %
 The concept of Innenwelt refers to the inner world and contrasts with Umwelt pointing to an experience coming from within the body. Umwelt is a~particular world view. Innenwelt is defined by internal states which characterize the physical state of an individual. The Innenwelt concept seems to be systemic. It is essential to explain why particular environmental features have greater importance than others.

Uexküll's theory makes one wonder how the world reality is described and what it means to be an animal. Not only does it multiply worlds in a~variety of environments, but it also tries to reject the understanding of the animal as a~soulless machine, a~mindless or impassive object. One can recognize here the biological interpretation of Kant. Uexküll states that the world owes its existence to the organism’s internal subjective organization, which turns sensory features into a~spatial form
%\label{ref:RNDPNpI1bWCXU}(Uexküll, 1920, p.12).
\parencite[][p.12]{uexkull_theoretische_1920}. %
 Uexküll had also introduced a~new way of thinking about reality as more than just the physical world. He was one of the first biologists which emphasize the subjective experience of an animal.

One sees the development of an animals’ intelligence in observing its behaviour in its environment. Uexküll postulated to treats animals as subjects and as entities whose primary activity is perceiving and acting. Everything that the subject perceives becomes its world of perception (\textit{Merkwelt})
%\label{ref:RND3G0utXH9GM}(Uexküll, 1934, pp.26–28).
\parencite[][pp.26–28]{uexkull_streifzuge_1934}. %
 Everything the subject does becomes its world of action (\textit{Wirkwelt}). Uexküll focused on the subjective environment of animal and its relationships which may not be noticed by the human-researcher. An animal in its environment deals with many objects with which it can interact. Some of the perceived things become carriers of meaning when they enter into a~relationship with the subject 
%\label{ref:RNDe2roScZjZy}(Uexküll, 1934, pp.105–107)
\parencite[][pp.105–107]{uexkull_streifzuge_1934}%
\footnote{Uexküll shows an example of a~stone-throwing at a~barking dog. The stone as a~physical object does not change, yet there is a~majority change in its meaning. As long as it lay on the road, it not grab the dog's attention. It turns into a~carrier of meaning as soon as it enters into a~relationship with the dog/subject. Now, the stone becomes a~bullet associated with a~feeling of pain. The subject produces meaning. This example shows the importance and primacy of functional relations.}. Objects are experienced because of their functional meanings to the subject. The study of perception cannot be isolated from other bodily functions. It should be regarded as a~phase of action related to motor and intellectual activity. Living organisms exhibit activity as long as some meaning to them derives from functional relationships with other objects, whether they are spatial, temporal, causal, or purposeful relationships.

To adequately explain theory, Uexküll presented a~concept of the functional circle. This concept is an influential conceptual tool that emphasizes the subject’s interaction with the environment. It offers an instrument to describe the sensorimotor activity. In the perspective of Uexküll’s research behaviours are not movements or tropisms as Jacques Loeb would have wished
%\label{ref:RND6t2Nh8l6mh}(Loeb, 1907, pp.151–156)
\parencite[][pp.151–156]{loeb_concerning_1907}%
\footnote{Jacques Loeb developed a~theory of animal behaviour based on tropism---involuntary, forced movement. He presented the animals’ response to a~stimulus as a~direct and automatic response. He believed that the behavioural response was forced by the stimulus and does not require an explanation in terms of the animal’s supposed consciousness.}, but consist of perception (\textit{Merken}) and action (\textit{Wirken}). Behaviour is not the result of mechanically regulated reflexes but it is organized according to the subject's body composition and sensory abilities 
%\label{ref:RNDsH9ZTqXBQZ}(Uexküll, 1982, p.26).
\parencite[][p.26]{uexkull_theory_1982}. %
 The functional circle describes the basic structure of the interaction between animal and objects appear in the Umwelt 
%\label{ref:RNDhV05hT2tOT}(Uexküll, 1934, p.27).
\parencite[][p.27]{uexkull_streifzuge_1934}. %
 Subject capture the neutral object from the environment as a~meaning carrier and a~perceiving organ or a~perceiving cell, modify it by an effector organ and use it to respond 
%\label{ref:RND4BoP1il7b7}(Uexküll, 1987, p.170).
\parencite[][p.170]{uexkull_sign_1987}. %
 Each organism simultaneously observes the world and changes it. The functional circle shows how the organism interacts as a~subject with the object of its action.

In every process in which the organism is involved, the carrier of meaning plays a~leading role, either as a~carrier of a~perceptual stimulus or as a~carrier of an effector stimulus
%\label{ref:RNDVUxL8A9wWu}(Uexküll, 1982).
\parencite[][]{uexkull_theory_1982}. %
 According to Uexküll, the functional circle transforms sensory data into meaning. The circle processes trigger the subject’s reaction. Meaning here is understood as a~structure that connects perception and action. Functional circles present the act of biosemiosis, which is the basis for the formation of cognitive processes. One can see how the organism and its environment feedback enables rational action in Umwelt. Object and subject in a~circle connect and form a~whole. An individual is formed through the plasticity of the semantic structure and experience. The subject can identify signs and gives meaning to specific environmental cues. It produces behaviour that the external observer perceives as adequate and intelligent, and for the organism conforms to its own needs and requirements 
%\label{ref:RNDXM7JxXuRx6}(Uexküll, 1934, pp.28–29).
\parencite[][pp.28–29]{uexkull_streifzuge_1934}.%


The subject uses his sensorimotor system to meet its needs effectively. Features of the subject are structurally related. Features with perceptual meaning affect features of an operational nature and vice versa. The subject’s state changes causing it to adapt to the environment and situation in the best possible way. One observes an action that bears the signs of intelligence. Even a~simple organism uses the sensory organs and the motor organs to process meaning to the best effect.

Depicting an animal’s senses and motor organs as these were machine parts ignores their actual functions and operation. The sensations and the acting will, are a~kind of perception of the operator built into these organs. Everything subject perceives and makes the result of the subjective perception of the world. The operator acts according to its own needs which form a~closed whole which is its Umwelt. A~simple functional cycle presents receptors and effectors signals as manifestations of the subject’s actions. Objects are merely carriers of meaning. The observation of animals behaviour leads to the conclusion that they act intelligently. No matter how simple the symptoms are. One cannot describe the intelligence in terms of one-sided human definition because intelligence is a~result of subjective perception and action.

The functional circle model contains all elements that are part of the meaning process. The connection of them depicts the process of semiosis. The subject interprets the external signals as the sign; the sign evoke the biological state of the organism; the biological state determines behaviour. On the other hand, there could be an object that is hard to interpret. Any object may have a~temporary existence as a~different semiotic object. That means anything has the possibility of a~meaning change. The functional circle connects perception and action that enables the understanding of reason and goal of the action. Objects can be associated with the other items through different functional circles. The more complicated organism has a~greater number of circles. Nevertheless, the organism deals with perception and action to imprint its meaning on the meaningless object and makes it a~carrier of meaning for the subject in Umwelt
%\label{ref:RND8x8dSdW7Pd}(Uexküll, 1934, p.110).
\parencite[][p.110]{uexkull_streifzuge_1934}.%


Umwelt of an animal is only a~fragment of the universe that the human observer perceives as the human world. One needs to recognize the perceptual signals of all stimuli in the environment to consider the intelligent operation of the system. That is the first condition to understand the Umwelt of the organism in which certain objects and situations become the reason why it takes actions. The organism can perform the number of functions which is equal to the number of objects. This number increases if the number of objects that fill the Umwelt increases too. Each new experience is associated with re-adaptation to new situations. Uexküll noticed that objects of cognition are transformed into perceptual signals and become real objects of Umwelt
%\label{ref:RNDbwAgtViSpi}(Uexküll, 1934).
\parencite[][]{uexkull_streifzuge_1934}. %
 Things do not gain value until they are transformed into the carrier of meaning. An thing without a~relationship remains meaningless. The meaning depends on the perception of the object and action dependent on the need. From Uexküll's point of view, properties of an object are a~perception marked as meaningful by the subject in a~relationship with it. Objects are initially neutral in the subject's universe. Objects become carriers of meanings by giving them a~function which depends on the mood and needs of the subject.

Animals like humans exhibit activities that are attempting to exceed biological constraints. Uexküll was aware that perception and action also depend on the experiences of the cognitive subject. New experiences give rise to generalizations that become the point for creating new levels of generality. The process of learning, adopting new patterns, looking for alternative explications can start again on a~new level. Jesper Hoffmeyer calls a~network of interactions that span living systems semethic interaction. Semethic interaction is the habit acquired by an individual, which is used (and interpreted) by others of the same species so it induces new habits in the particular group
%\label{ref:RNDGExh7UPE60}(Hoffmeyer, 1998).
\parencite[][]{hoffmeyer_unfolding_1998}. %
 That means that any regularity developed in a~living system (on any level) tends to create new interpretations and build a~set of new experiences.

The concept of the formation of meaning presented by Uexküll makes it possible to use biosemiotics to study intelligence. Biosemiotics is placing interpretation at the centre of attention. It shows that semiosis is an inevitable feature of life and argues that the process of meaning is a~fundamental form of intelligence. Uexküll proved that the concepts of biosemiotics, although started from the attributes of objects (their perception and actions) cannot depend on specific physical implementation. According to Peirce, semiotic communication includes the sign, the object of the sign, and the interpreter
%\label{ref:RNDfClkaMysma}(Peirce, 1998).
\parencite[][]{peirce_essential_1998}. %
 Anything must be understood to be a~sign. The signs require interpretation, otherwise, they may not even be considered signs. The basic principle of Peirce’s semiotics is that index and iconic signs have no meaning in isolation 
%\label{ref:RNDWErWIt9enB}(Peirce, 1998).
\parencite[][]{peirce_essential_1998}. %
 Biosemiotics takes these issues seriously. It tries to present an approach to the idea of [200B?][200B?]shaping mental states and the biological sources of this process as well. Considering Uexküll's theory cannot ignore semiotic interactions at any level. To capture the meaning, one has to perceive signification, representation and reference as distinctive on any occasion.

\section*{Embodiment and intelligence}
The body is an essential factor enabling cognition or thinking. The existence of the body is a~necessary condition for intelligence. Especially the embodiment of mental processes is a~leading topic in cognitive science and neurobiology
%\label{ref:RNDhO80G68YtS}(Clark, 1997; Damasio, 1994; Deely, 1990; Lakoff and Johnson, 1999; Varela, 1993).
\parencites[][]{clark_being_1997}[][]{damasio_descartes_1994}[][]{deely_basics_1990}[][]{lakoff_philosophy_1999}[][]{varela_embodied_1993}. %
 The body’s sensors deliver sensory signals to the brain and provide a~basis for action. For example: grasping small objects with your fingertips is simple because there are more sensors than in the finger joints or the metacarpus. The grasp of a~small hard object requires more limited control due to the deformable tissue of the finger. There is some of the neural control taken over by the morphological and material properties of the hand. The material properties of the musculoskeletal system allow making quick movements. Even then, the nervous system is too slow to control all details of the movement 
%\label{ref:RNDBbqi11VBD0}(Clark, 1997; Damasio, 1994; Deely, 1990; Lakoff and Johnson, 1999; Varela, 1993).
\parencites[][]{clark_being_1997}[][]{damasio_descartes_1994}[][]{deely_basics_1990}[][]{lakoff_philosophy_1999}[][]{varela_embodied_1993}. %
 One of the aspects of the brain works is proprioception---awareness of body mobility and position. Even most simple moves require constant feedback from the proprioceptive organs in the body. Proprioceptive organs measure muscle strain and cell layer displacement including the gravitational orientation. Maxine Sheets-Johnstone suggests that the proprioceptive sense is a~bodily awareness. Any self-moving creature feels his movement and his stillness 
%\label{ref:RND2oq6ptThjc}(Sheets-Johnstone, 1998).
\parencite[][]{sheets-johnstone_consciousness_1998}.%


Cognitive science assumes that concept formation is related to specific motor programs motivated by perception and action in an experimental context. Eleanor Rosch proves the basic concepts in the human mind are related to types of things or actions with which subject has motor experience
%\label{ref:RND6uJBCE7W1t}(Rosch, 1999).
\parencite[][]{rosch_principles_1999}. %
 Afterwards, one can create schematic representations of images: tables, walls, bicycles, buildings, talking, walking, sleeping, etc. Sensory-motor knowledge of the world determines the fundamental concepts of the world. Primary concepts have a~core, depending on the basic functions of the organism. The large part of them is related to the perception and practice of acting in a~settled environment. Embodied gestures become mental patterns used in perception and reason (part-whole, centre-periphery, goal-path, straight-curved and near-far, cycle, contiguity, movement, balance). The structures of these gestures are body goal-oriented. The essential consequence is that elementary bodily experiences are the starting point for mental activities. The result of experiencing states and things in the world arises as an abstract thought. George Lakoff and Mark Johnson constructed a~groundbreaking theory. They model a~bodily metaphor as a~relevant cognitive tool providing structural metaphors led on metaphorical expressions.\footnote{For example, the structured conceptual metaphor ``Knowledge is seeing'' experience in many continuous languages a~series of different expressions such as ``enlightenment'', ``Can’t you see what I~am explaining?'', ``Look at the problem'' 
%\label{ref:RNDMZkc53iwuI}(Lakoff and Johnson, 2008).
\parencite[][]{lakoff_metaphors_2008}.%
} Imagination turned out to be the substantial instrument using metaphors and building developed conceptual models in thought experiments: idealized cognitive models constructed from basic concepts, schemas and mappings between them.

Biosemioticians try to study the embodied nature of the mental realm. The statement that mental processes are embodied implies their natural history of formation and development. One cannot separate it from embodied life. Mental life grounds at bodily intentionality manifested in the cycles of perception-action, thus ultimately in biosemiosis. Neurosciences reveal numerous mechanisms that may explain the effects of the sensory system in behavioural development
%\label{ref:RNDhClIGKtAiP}(Purves, 1988, pp.19–20).
\parencite[][pp.19–20]{purves_body_1988}. %
 There is a~relationship between body structure and the central nervous system. The number of mechanisms: muscle trophic responses, processes caused by the activity of sensory and motor cells, and hormonal changes influence distant parts of the body. Informational signals are generated in the various sensory channels through the physical interaction of the living system with the environment. The feeling of moving is caused by seeing changes in the environment that correlate with a~muscle strain. Objects closer appear to move faster than those farther away. Consequently, the body is informed about the distance. There is a~relationship between the brain’s neural activity, body morphology (shape and material properties) and interaction with the environment used to achieve specific tasks. Various cellular processes lead to the survival of the relevant neurons and different forms of the nervous system.

Merleau-Ponty identified primal awareness as not ``I think'' but ``I can''
%\label{ref:RNDvuftmKLf11}(Merleau-Ponty, 2001, p.156).
\parencite[][p.156]{merleau-ponty_fenomenologia_2001}. %
 The initiation of a~movement coexists with the motivation to perform it and requires the species-specific mobility range. It is the base of the ``I can'' potential, which subordinates the proprioceptive states and possibilities of the subject’s action. The body---carrier of meaning---allows feeling the external features of the environment, sensations of movement and stimulus from the inside. The potential possibilities of the body turn ``I can'' into ``I do''. Merleau-Ponty points out that the behaviour of an organism in Umwelt is more primary than awareness, which is only one of the particular forms of this behaviour 
%\label{ref:RNDk80o8RLt0v}(Merleau-Ponty, 2001, p.239,393).
\parencite[][p.239,393]{merleau-ponty_fenomenologia_2001}. %
 The connection between the body and the environment is the fundamental condition for the emergence of conscious functioning. Feeling and movement (perception and action) enable the body to discover and understand Umwelt. The sense organs in the functional circles make possible the body to act more precisely. It means that the animal distinguishes its spatial position due to an inherent neural proprioception system that facilitates feedback control of behaviour in relation to the proper perceptual and behavioural worlds. The perceptual world becomes possible as the comprehended while the body perceives itself. The represented world of the mind is the clue of possibility and choice to act intelligently in it.

Merleau-Ponty’s interpretation of Edmund Husserl’s concept is interesting. Husserl’s problem, as Merleau-Ponty argues, is to find a~place for nature in the philosophy of reflection
%\label{ref:RNDYpoLAKtCa7}(Merleau-Ponty, 2003, p.72).
\parencite[][p.72]{merleau-ponty_nature_2003}. %
 The perception of nature allows for the correlation of the environment and consciousness in \textit{Lebenswelt}. Merleau-Ponty portrays Husserl’s ``body-world'' as an Umwelt sensory engine. He explains a~way of perceiving as the motor capabilities of the subject’s body: the function of the body 
%\label{ref:RNDUnbZXhWv4i}(Merleau-Ponty, 2003, p.74).
\parencite[][p.74]{merleau-ponty_nature_2003}. %
 The body is suited for being in the world. The world becomes a~part of the subject’s body. It guarantees the possibility of orientation, not only in space-time but in all normative scales. Merleau-Ponty’s phenomenological interpretation of Uexküll’s theory shows Umwelt as the actuality of existence. Behaviour in such Umwelt cannot be understood with the moment, but only as a~significant whole of existence in time. Symbols affect the animal in current orientation, respond to options and lead to future perception.

\section*{Fundamental role of semiotic scaffolding}
The Uexküll functional circle is the blueprint of the sensory-motor body. The body is conceived as a~semiotic device perceiving and acting in its surroundings through signs
%\label{ref:RND1UnYgD9VWX}(Heidegger, 1992, p.261).
\parencite[][p.261]{heidegger_grundbegriffe_1992}. %
 Uexküll admitted the existence of indefinite neutral objects in Umwelt which are necessary to increase the complexity of the Umwelt 
%\label{ref:RNDSnj3ixXzlm}(Uexküll, 1934, pp.92–93).
\parencite[][pp.92–93]{uexkull_streifzuge_1934}. %
 The perception of neutral objects is a~prerequisite condition for learning. Placing them in the action of existing functional circles makes it necessary to expand circles. The adaptability of circles presupposes the perception of neutral objects that are initially irrelevant but cause an imbalance in the alignment of the organism to the environment. The development of the functional circle is triggered by increasing semiotic complexity. The situation of growing environmental pressure results in the optimal tension in the previously defined functional circle by adjusting it to the occurring changes.

Due to the organism’s autonomic activity, some actions accomplish well when information is incomplete. An autonomously functioning system can fill knowledge gaps by using external data. It is possible by the network of semiotic interactions with which individual cells, organisms or populations control their actions
%\label{ref:RNDe6tQGQ86jH}(Kull, 2015).
\parencite[][]{kull_evolution_2015}. %
 Semiotic scaffolding provides organisms with precise actions by ensuring efficient interaction with key signs arising in dynamic situations (e.g. hunting or mating). This term is understood in a~general sense as an entity or process that supports another process and thus enhances the stability, functioning or range of possibilities of cognitive activity 
%\label{ref:RND1ecHE08Vtt}(Emmeche, Kull and Stjernfelt, 2002, p.29).
\parencite[][p.29]{emmeche_reading_2002}. %
 Patterns of action emerge as intelligent system behaviour of various autonomous activities depends on the environment. Semiotic scaffolding takes the place of a~centralized decision-making system based on internally represented directions of activities or goals. Horst Hendriks-Jansen stated that interactive behaviour could be explained in generative terms because there are not internal rules to predict the types of behaviour that occur.\footnote{The concept of generativity describes an autonomous system that creates and uses new unique behaviours without relying on external data of this system 
%\label{ref:RNDxptcDXaTqv}(Hendriks-Jansen, 1996, p.9).
\parencite[][p.9]{hendriks-jansen_catching_1996}.%
} Andy Clark 
%\label{ref:RNDBGUO37p9LU}(1997, p.46),
\parencite*[][p.46]{clark_being_1997},%
 on the other hand, suggested that intelligent creatures use environmental structures for their purpose instead of storing or expensively processing information.

The importance of scaffolding has already been emphasized by Lev Vygotsky, who described the child’s development as gaining experience with the support of external structures.\footnote{It can be physical support when learning to walk or swim, or language support when learning to speak
%\label{ref:RNDoTFfYGAY0m}(Vygotsky, 1964).
\parencite[][]{vygotsky_thought_1964}.%
} New skills can be socially transferred from caregiver to child through mimicry or scaffolding when a~more capable adult manipulates the pupil’s interactions with the environment to develop novel skills. Scaffolding reduces the distance, highlight the most considerable characteristics of the task, decrease the number of steps in the performed plan and allow the finish the action. Scaffolding is necessary to experience before the child gains an independent cognitive or physical ability to seek and achieve a~goal. The concept of scaffolding describes living organisms use external structures to simplify tasks. Natural language and information technology are an example of a~powerful semiotic human scaffold. Human knowledge can be written in books and passed on. Man can rely on what was already established and written down: ideas in one text rely (directly or indirectly) on other texts. The World Wide Web with text, pictures, sound and video made the web of knowledge and ideas clearer and more accessible 
%\label{ref:RNDIWQyrnwJiZ}(Clark, 2004).
\parencite[][]{clark_natural-born_2004}.%


Scaffolding is raised to make a~building but limits and determines the way a~skyscraper is built. The semiotic control of biological actions also limits and determines the time and manner in which activity takes place. Conceptualization and analysis of semiotic scaffold mechanisms operating at different natural systems levels are the core of biosemiotics research. Semiotic scaffolding obtains many forms, but their primary property remains the focus of an organism’s behaviour on a~limited repertoire of possibilities or guiding behaviour to implement a~sequence of actions. The cell receptor is tuned to open when and only it hits the appropriate stimulus, e.g. an amphibian’s eye is the result of formed chemical interactions between the newly formed optical vesicle and the embryonic layer of the ectoderm. The chemical inductor produced by the optical vesicle is used as the scaffold for this action
%\label{ref:RNDmInECjoP0B}(Kull, 2014).
\parencite[][]{kull_catalysis_2014}. %
 This example shows the important aspect of development as the ability of individual cells to alter their internal settings under the influence of external factors or new molecular cues.

Scaffold-based activities become more complicated as the life cycle of organisms becomes complex or subject begin to engage in social processes. Semiotic mechanisms depend on changes in interpretations that always assume error. They also rest on the ability to foresee and prepare for the events and life cycle incidents. Gene scaffoldings work by controlling and assembling proteins converted into patterns that reflect the organism’s needs
%\label{ref:RNDA6AUvdDFj2}(Hoffmeyer, 2014).
\parencite[][]{hoffmeyer_semiome_2014}. %
 These mechanisms work well as long as the behavioural repertoire is limited to instinctively triggered responses to predictable events. However, animals with large brains, such as birds or mammals, depend on learning processes in addition to their instinctive reflexes. Such processes are genetically assisted by the provided preferences, but the ability to learn must be an integral part of behavioural flexibility. The transfer of behavioural control from the genome level to the brain level introduces the need to use scaffolding mechanisms.

Organisms use scaffolds to refer to the environment, to symbolize, to reason, to use patterns. The benefits from the abilities are acquired through phylogenesis or ontogenesis to establish regularities in the environment and direct actions accordingly. A~Ringed Plover, which pretends to have a~broken wing to distract a~weasel from its nest takes advantage of the predator’s semiotic scaffolding to be fooled by false signs. The bird deceives the mammal because it has genetically or ontogenetically acquired knowledge of how the predator will interpret the apparent relationship. However, there may be times when a~weasel is not deceived, so it proves that its responses are not strictly deterministic. The predator may or may not misinterpret the sign. It also means that the sign essence is the relational nature of causing awareness of something that is not in itself. This fact implies a~full Peircean sign triad that can arise in any system capable of autonomous anticipatory activity.

Semiotic scaffoldings are related to the interaction of the body’s interior and surroundings. They occur in the semiosphere of living organisms, which rely on communication with the world around them by sounds, smells, movement, colour, shape, electric field, chemical signals, touch and provide them with cognitive activity. The semiosphere is the result and condition for culture development by analogy with the Vladimir Wiernadski’s term ``biosphere''. Semiosphere is an organic whole of living nature and also a~place for the continuation of life
%\label{ref:RNDBTZNG5SfoG}(Lotman, 1990, pp.12–126).
\parencite[][pp.12–126]{lotman_universe_1990}.%
\footnote{Lotman, who introduced the concept ``semiosphere'' used it as an analogy with the term ``biosphere'' by Vladimir Wiernadski. Although in Lotman’s writings, semiosphere is associated formerly with culture, he presents semiosis as the fundamental mechanism of action that connects the entire semiotic space. The concept of Wiernadski's biosphere did not outlast in this sense. Now, it is a~term ``ecosystem'' including the earth with the inhibited organisms.} The concept of semiotic scaffolding of the semiosphere gives a~semiotic dimension to life processes emphasizing their belonging to the world of sign activity. Uexküll presented the processes of building independent relations with the world as a~necessary condition for the autonomously functioning system.

Species have limited access to this semiosphere because they can interpret potential signals in its environment. They evolved to fit into a~specific ecological niche. This niche contains all information and directions that must be correctly interpreted by the organism. The number of features of the world becomes mattering indications of an organism’s behaviour. It is infinite and more than the number of traits with which the organism interacts physically. The bird must notice the possibility of food or shelter to obtain the best effect but use semiotic scaffolds that are patterns of sounds, directions and wind speed, differences in air or temperature, changes in the intensity and wavelength of light. There are many environmental changes during the animal’s life. Therefore, the semiosphere and semiotic scaffolding indicate an enormous amount of possibilities for action or adaptation. Organisms cannot react only passively or instinctively to states and events. Instead, they perceive, interpret and act in the environment in a~way that creatively and unpredictably alters all evolutionary and selective settings.

Intelligence is a~sophisticated activity traditionally seen as extending the biological behaviour of animals. Modern research, however, recognizes that animals possess intelligence
%\label{ref:RNDrkME9KRdd5}(Waal, 2016; Heinrich, 1999; Fischer and Menzel, 2011).
\parencites[][]{waal_are_2016}[][]{heinrich_mind_1999}[][]{fischer_animal_2011}. %
 Many biologists have long believed that instincts and intelligent behaviour are in opposition to each other. They argued that successful action is a~coincidence of instincts interaction 
%\label{ref:RND2TjEYo9od4}(Richter, 1927; Thorndike, 2017, p.150).
\parencites[][]{richter_animal_1927}[][p.150]{thorndike_animal_2017}. %
 It is worth emphasizing, that intelligence is not the elusive trait somewhere in mind, but is related to social abilities, the ability to use physical signs, and the ability to accumulate and organize knowledge. Peirce defined the term ``semiotic'' as characteristic for all signs used by an intelligence capable of learning through experience 
%\label{ref:RNDNFUp52vkZ1}(Peirce, 1994 par. 227).
\parencite[][]{peirce_collected_1994}. %
 Extending intelligence to all living systems is an attempt to bridge the ontological gap between humans and animals. It also allows you to look at machine intelligence without having to refer to the human cognitive system.

Semiosis plays a~central role in experiencing and gaining knowledge about the world. The semiotic process penetrates the beginning, middle and end of the development of every organism. In the semiotic system, plans and concepts without intelligent actions would not take place. Therefore, symbolization is the basis of intelligence. Biology and semiotics are intertwining more and more. Biology provides solid empirical support for the understanding that semiotic faculties are the essence of intelligence. Each biological or genetic code is a~semiotic system par excellence and can be interpreted as the basis of intelligence itself: the connection of the genetic codes contained in living cells with the very nature of these cells, the differentiation and integration of these cells into complex organisms, their development in various organs, and ultimately organization in the brain. Intelligence indicates the process of experiencing the physical world that makes sense for structures and the whole life of the organism.

\section*{Conclusion}
Animals exhibit intelligent behaviour, such as planning, hunting for a~particular type of prey, remembering to find or build a~shelter, participating in courtship, and using other strategies commonly thought to occur only in higher species. For Uexküll, it was related to the natural cycle of life, based on the principles regulating life activities. In the subjective Umwelt, the animals produce variants of actions, perceive gradations, anticipate failures and they are deceived by illusions. The animal perceives the world in an integrated way without breaking it down into parts, colours or sounds. Its perceptions and actions are intelligent, context-sensitive and deeply meaningful. Experiencing a~specific situation is determined by changing moods, current needs and intentions and sensory-motor skills. Skills develop through the world experience, practice and therefore appear to play a~pivotal role in the embodiment and situating of intelligent behaviour
%\label{ref:RNDEJ1KZ3QtFR}(Dreyfus, 1979; Searle, 1992).
\parencites[][]{dreyfus_what_1979}[][]{searle_rediscovery_1992}.%


Earlier attempts to deal with this knowledge resulted in the theory of mind-body dualism. The biosemiotic approach can help since perceive meaning (\textit{sema}) as inherent to the body (\textit{soma}). The body is involved in communication processes that coordinate the activity of cells, tissues and organs. The exchange of messages integrates the different levels of the hierarchy. In this standing, intelligence is the interface of the organism manages its relationship with the surrounding environment. Organisms are systems that cannot be interpreted as being independent of the environment but as adaptive systems.

Mental life is grounded in bodily intentionality manifested as the cycles of perception and action. This activity integrates organisms’ survival strategies. Above all, it is a~deliberate action---a~precursor of life dimension. At higher animals is perceived as intelligence and ultimately consciousness. Biosemiotics explores the fundamental processes of mental life arising and argues that the whole organism functions are semiotic processes that precede the emergence of authentic intelligence. The semiotic scaffolding ensures that appropriate cognitive level is obtained and improves the cognitive functions of the subject. Distributed cognition describes cognitive acts as the result of the system operation. It involves the subject, others and objects and the situational context in the environment. Distributed intelligence involves support to perform an action that would otherwise be error-prone, difficult or impossible to achieve. The activities based on external scaffolding can enable understanding of intelligent behaviour that is supported outside the mind. By the analogy of embodied and situated artificial agents to some animals this perspective could be useful for understanding of artificial agents’ intelligent behaviour.

\end{artengenv}