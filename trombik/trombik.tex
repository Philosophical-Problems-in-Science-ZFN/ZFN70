\begin{newrevplenv}{Paweł Polak, Kamil Trombik, Roman Krzanowski}
	{Internetowe szaty filozofii epoki informatycznej}
	{Internetowe szaty filozofii epoki informatycznej}
	{Internetowe szaty filozofii epoki informatycznej}
	{Uniwersytet Papieski Jana Pawła II w Krakowie}
	{Blog Cafe Aleph, red. Witold Marciszewski, Paweł Stacewicz, https://marciszewski.eu.}

\lettrine[loversize=0.13,lines=2,lraise=-0.03,nindent=0em,findent=0.2pt]%
{R}{}ecenzje książek to jedno z~ważnych narzędzi rozwoju filozofii w~epoce Gutenberga
%(więcej na temat historycznej roli recenzji zob. Pleshkov, Surman, 2021).
\parencite[więcej na temat historycznej roli recenzji zob.][]{pleshkov_book_2021}. %
Ukazują one filozofię \textit{in statu nascendi} i~choć to obraz mocno wybiórczy, to pozwala dobrze wyczuć niepokoje danego czasu i~zrozumieć problemy trapiące aktualnie myślicieli. Dziś, gdy żyjemy w~epoce Internetu, trudno nie zauważać nowych form publikacji i~niesionych przez nich nowych szans oraz wyzwań. Publikacja internetowa nie musi być jedynie publikacją pliku z~wydrukiem -- nowa technika daje o~wiele większe możliwości, dotychczas prawie w~ogóle nie eksplorowane przez filozofów. Ten konserwatyzm filozofii wobec nowych źródeł wyrazu towarzyszy filozofii od jej zarania (wszak u~Platona odnajdziemy narzekania na pomysł utrwalania myśli filozoficznej pismem). Jeśli miał rację Whitehead twierdząc, że historia filozofii zachodniej jest szeregiem przypisów do Platona, to sceptycyzm wobec nowych ,,technologii filozofowania'', powinien mieć również swą godną reprezentację. W~niniejszej recenzji chcielibyśmy zmierzyć się, choćby w~małym zakresie, z~tym metodologicznym sceptycyzmem wobec technologii.

Książka filozoficzna jako forma komunikacji naukowej ukształtowana została zarówno przez wrażliwość i~wyobraźnię kolejnych pokoleń nowożytnych filozofów, jak i~przez możliwości oraz ograniczenia techniki druku. W~niniejszej recenzji chcielibyśmy się przyjrzeć publikacji internetowej, która z~jednej strony nawiązuje do książkowej formy publikacji (wszak blog oznaczał pierwotnie rodzaj pamiętnika\footnote{Na potrzeby tej recenzji możemy pokrótce określić blog jako rodzaj strony internetowej, na której twórca -- przy użyciu tekstu, a~niekiedy także materiałów graficznych i~nagrań video -- zamieszcza odrębne, autorskie wpisy o~charakterze zbliżonym np. do notatek z~dzienników.}), ale znacząco poszerza jej możliwości przez dodanie możliwości interakcji i~dyskusji. Mamy na myśli blog filozoficzny ,,Cafe Aleph'',,który jest pierwszym polskim szeroko zakrojonym eksperymentem odnośnie nowych form publikacji filozoficznych\footnote{Warto przy okazji odnotować, że na przestrzeni ostatnich lat pojawiło się wiele poczytnych blogów o~tematyce filozoficznej, np. ,,Philosophy Talk'', (\textit{https://www.philosophytalk.org/blog-classic}) czy ,,Leiter Reports: A~Philosophy Blog'', (\textit{https://leiterreports.typepad.com/blog}). Blog o~tematyce filozoficznej prowadzi także jeden z~autorów niniejszej recenzji (blog ,,Filozoficzny Kraków'', autorstwa Pawła Polaka, dostępny pod adresem: \textit{filozoficznykrakow.wordpress.com}). Można odnieść wrażenie, że blogi filozoficzne stają się bardzo popularną formą filozofowania, przykuwając uwagę szerokiego grona czytelników. Zestawienia licznych, popularnych blogów o~tematyce filozoficznej można znaleźć m.in. pod adresami: \url{http://consc.net/philosophical-weblogs} oraz \url{https://blog.feedspot.com/philosophy_blogs} [dostęp: 24.05.2021].}.

Blog założony przez prekursora filozofii informatycznej prof. Witolda Marciszewskiego oraz jego współpracownika dra Pawła Stacewicza jest miejscem rozwijania, popularyzacji i~dyskusji wokół idei filozofii informatycznej i~światopoglądu informatycznego\footnote{Blog stanowi kontynuację prac nad wspólną książką Autorów pt. ,,Umysł–Komputer–Świat. O~zagadce umysłu z~informatycznego punktu widzenia'',
%(Marciszewski, Stacewicz, 2011).
\parencite{marciszewski_2011}.%
}. W~warstwie tekstowej blog zawiera wiele interesujących artykułów prezentujących wybrane zagadnienia filozoficzne, ale tym co go wyróżnia z~innych blogów, to aktywne dyskusje online prowadzone pomiędzy autorami a~szerokim gronem osób zainteresowanych z~całego kraju. Tym, co czyni wspomniany blog dla nas równie interesującym jest bliskość w~stosunku do koncepcji ,,filozofii w~nauce'',
(Heller, 2019; cf. Polak, 2019).
\parencites[][]{heller_how_2019}[cf.][]{polak_philosophy_2019}
Wymowny jest zresztą fakt, że wiele zagadnień poruszanych na blogu ściśle wiąże się z~problematyką poruszaną od kilku lat także na łamach ,,Zagadnień Filozoficznych w~Nauce'',
%(Leciejewski, 2018; Krzanowski, 2020, 2017, 2016; Kycia, Niemczynowicz, 2020; Kycia, 2021).
\parencites{leciejewski_structure_2018}{krzanowski_why_2020}{krzanowski_minimal_2017}{krzanowski_towards_2016}{kycia_information_2020}{kycia_information_2021}.

Działalność ,,Cafe Aleph'', została zapoczątkowana w~2011 roku. Blog szybko znalazł uznanie w~oczach społeczności internetowej, zwłaszcza w~kręgu osób zainteresowanych zagadnieniami o~charakterze interdyscyplinarnym. Dzięki regularnie pojawiającym się wpisom stał się z~czasem nie tylko źródłem wielu cennych informacji, ale także platformą wymiany myśli między użytkownikami. W~,,Cafe Aleph'', przeprowadzono 167 dyskusji\footnote{Stan z~dnia 2.02.2021 r.}, zainicjowanych eksperckimi wpisami dotyczącymi problematyki z~pogranicza filozofii, nauki i~techniki. Obszar tematyczny wpisów jest ściśle powiązany z~zainteresowaniami badawczymi samych autorów. Na blogu dominuje więc problematyka z~zakresu filozofii informatyki (m.in. pojęcia i~metody informatyki, zagadnienia filozoficzne w~informatyce, wpływ technologii informatycznych na współczesną cywilizację), ponadto poruszane są kwestie związane z~filozoficznymi zagadnieniami matematyki i~logiki (np. zagadnienie nieskończoności). Autorzy podejmują także pytania nawiązujące do tzw. światopoglądu informatycznego i~coraz szerzej oddziałującej koncepcji informatyzmu. Dyskusje na blogu prowadzone są przede wszystkim w~j. polskim, a~biorą w~nich udział zarówno naukowcy i~nauczyciele akademiccy, jak i~studenci oraz inne osoby zainteresowane problematyką znajdującą się w~obszarze eksplorowanym zawodowo przez prof. Witolda Marciszewskiego i~dr. Pawła Stacewicza.

Blog ma zasadniczo charakter naukowo-ekspercki. Od 2015 roku jest też powiązany z~międzyuczelnianym Seminarium z~Filozofii Nauki, organizowanym w~ramach Wydziału Administracji i~Nauk Społecznych Politechniki Warszawskiej\footnote{W~blogu działa podstrona dedykowana dla tego seminarium, na której zamieszczane i~archiwizowane są materiały seminaryjne: \url{https://marciszewski.eu/?page_id=8381}}. Oprócz twórców bloga, w~,,Cafe Aleph'', specjalistyczne wpisy publikują osoby o~uznanym dorobku naukowym, m.in. Marek Hetmański, Adam Olszewski, Jakub Jernajczyk, Paweł Polak, Marcin Miłkowski czy Marcin Koszowy. Należy zaznaczyć, że w~ostatnich latach w~literaturze naukowej pojawiły się już odwołania do treści publikowanych na łamach ,,Cafe Aleph'',
%(zob. np. Sokół, 2014; Polak, 2017; Sarosiek, 2016),
\parencites[zob. np.][]{Sokół2014}{polak_czy_2017}{sarosiek_biologiczne_2016}, %
co pokazuje, iż oddziaływanie bloga w~społeczności akademickiej można uznać za porównywalne do oddziaływania tradycyjnych publikacji naukowych.

W~tym kontekście warto również dodać, że społeczność odbiorców i~współtwórców ,,Cafe Aleph'', wciąż się rozrasta. Coraz więcej użytkowników włącza się w~dyskusje wokół tematów inicjowanych przez autorów wpisów. Do dnia 2.02.2021~r.na blogu ukazało się 1341 komentarzy (głosów w~dyskusjach) osób reprezentujących różne grupy zawodowe, a~w ciągu ubiegłego roku blog odwiedziło ponad 50 tysięcy użytkowników. Statystyki świadczą o~dużym zainteresowaniu treściami dostępnymi w~,,Cafe Aleph'',. Co więcej, blog znalazł także uznanie w~oczach innych popularyzatorów nauki i~filozofii\footnote{Warto przy tym wspomnieć, że blog bywa również wykorzystywany w~działalności dydaktycznej. Na przykład podczas zajęć z~,,Metodyki pracy naukowej i~obsługi komputera'',,jakie są prowadzone na Wydziale Filozoficznym UPJPII, blog jest prezentowany i~polecany studentom przy okazji omawiania wartościowych materiałów filozoficznych w~Internecie.}. W~działania promocyjne ,,Cafe Aleph'', włączyło się m.in. znane w~polskim środowisku filozoficznym czasopismo ,,Filozofuj'',
(Stacewicz, Jernajczyk, 2020).
\parencite{stacewicz_filozofuj_2020}. %
Redaktorzy bloga, dzięki swojej aktywności w~sieci, są także regularnie zapraszani na różne wydarzenia upowszechniające naukę oraz technikę, jak Festiwal Nauki czy Festiwal Myśli Abstrakcyjnej.

Zakrojone na tak szeroką skalę oddziaływanie bloga w~połączeniu z~możliwością komentowania wpisów przez każdego, kto jest tylko zainteresowany problematyką filozofii informatyki, skłania do pytań o~jakość prowadzonych tam dyskusji. W~tym kontekście należy zwrócić uwagę na bardzo zróżnicowane formy wypowiedzi komentujących, co sprawia, że poziom dyskusji na blogu bywa momentami nierówny. Z~jednej strony mamy tu do czynienia ze swobodnymi, często niepogłębionymi i~mało profesjonalnymi refleksjami, które właściwie niewiele wnoszą do poruszanej problematyki. Z~drugiej strony część komentujących argumentuje na bardzo wysokim poziomie, wykazując przy tym doskonałą znajomość omawianych zagadnień. W~związku z~tym warto odnotować istotny problem związany z~narracją tekstu: poprzez zróżnicowanie form wypowiedzi czytelnik musi bowiem nieustannie włączać się w~zupełnie inne toki myślenia i~uzasadniania. Sprawia to, że odbiór treści zamieszczonych na blogu może nastręczać pewnych trudności.

Tak zróżnicowany poziom dyskusji jest nie bez znaczenia, zwłaszcza jeśli pomyślimy o~odbiorcach mniej zorientowanych w~poruszanej na blogu tematyce. Jak odróżnić treści merytorycznie wartościowe od tych mniej zaawansowanych czy wręcz takich, które mogą wprowadzać w~błąd? Wydaje się, że jest to jeden tych problemów, dla których trudno znaleźć satysfakcjonujące rozwiązanie, nie godzące w~tak ważną w~przestrzeni internetowej swobodę wypowiedzi. Niezależnie od tego warto zwrócić uwagę na istotny, bardzo pozytywny aspekt działalności ,,Cafe Aleph'',,związany z~otwartością jego twórców na głosy komentujących. Śmiało można bowiem stwierdzić, że w~ten sposób blog realizuje ważne cele społeczne -- mianowicie przybliża szerszemu gronu filozofię informatyki i~stanowi zachętę dla inżynierów do podejmowania refleksji oraz dyskusji o~charakterze filozoficznym. Wydaje się to szczególnie istotne w~kontekście wyzwań, przed jakimi stoimy w~związku z~oddziaływaniem technologii na różne sfery współczesnej kultury. Na tym polu wciąż jest wiele do zrobienia, dlatego z~szacunkiem należy podchodzić do każdej próby przybliżania filozofii informatyki, zwłaszcza w~środowisku inżynierów.

Ocena działalności ,,Cafe Aleph'', z~pewnością nie jest łatwa. Mimo wskazanych kwestii problematycznych, związanych m.in. z~narracją tekstu i~różnym poziomem wypowiedzi komentujących, blog można zaliczyć do udanych przedsięwzięć naukowych. ,,Cafe Aleph'', można jednocześnie potraktować jako swoisty eksperyment filozoficzny i~ważną przestrzeń popularyzacji cennych idei związanych z~rozwijającą się filozofią informatyki. Jeśli tylko w~środowisku filozofów na szerszą skalę przełamany zostanie sceptycyzm wobec tego typu form publikacji, to być może w~przyszłości blogi filozoficzne będą częściej traktowane jako wartościowe źródła naukowe.

W~przypadku bloga mamy do czynienia z~bardzo specyficznym i~wciąż jeszcze niezbyt dobrze przebadanym fenomenem współczesnej kultury -- fenomenem stanowiącym coś pośredniego między książką a~otwartym forum dyskusyjnym. Zręczne poruszanie w~sferze naukowego blogowania z~pewnością nie należy do łatwych zadań. Warto jednak dodać, że twórcy ,,Cafe Aleph'', dokładają starań, aby treści pojawiające się na blogu spełniały najwyższe standardy. Zarazem umożliwiają każdemu zainteresowanemu udział w~dyskusjach. Wprawdzie dyskusje te nie zawsze dorównują poziomowi eksperckich wpisów, jednak mają one społeczną wartość -- szczególnie w~kontekście upowszechniania wiedzy z~zakresu filozofii informatyki.

Wydaje się więc, że blog pozytywnie wpisuje się w~proces cyfryzacji komunikacji naukowej, nosząc w~sobie duży potencjał popularyzatorski czy wręcz naukotwórczy. Jednocześnie można stwierdzić, iż ,,Cafe Aleph'', sprawdza się także jako jedna z~nowych ,,technologii filozofowania'',,prowokując przy tym do pytań dotyczących wyzwań i~perspektyw związanych z~taką formą wymiany filozoficznych myśli.


%-------------------------------


\selectlanguage{english}
\vspace{5mm}%
\begin{flushright}
{\chaptitleeng\color{black!50}{Internet clothes of the philosophy in the information technology era -- Lorem ipsum}}
\end{flushright}

%\vspace{10mm}%
{\subsubsectit{\hfill Abstract}}\\
{Lorem ipsum dolor sit amet, consectetur adipiscing elit. Proin a blandit augue. Suspendisse tortor sem, sollicitudin id sollicitudin eget, bibendum et augue. Etiam bibendum at sem sed sollicitudin. Vestibulum quis sagittis neque. Orci varius natoque penatibus et magnis dis parturient montes, nascetur ridiculus mus. Suspendisse ac augue vitae felis eleifend egestas non vel ante. Aliquam mauris lacus, vestibulum eget pretium suscipit, laoreet sit amet orci. Pellentesque id dolor non orci pellentesque suscipit. Sed tincidunt luctus ultrices. Pellentesque auctor molestie tempor. Nunc in commodo arcu. Pellentesque nec suscipit sem. Suspendisse venenatis, ex eu fringilla aliquam, urna augue auctor diam, sit amet blandit mi nisl id elit. Donec libero nunc, venenatis ac ultrices eget, hendrerit eget nunc. Donec laoreet lacus sit amet velit interdum, quis viverra nibh fermentum.}\par%
\vspace{2mm}%
{\subsubsectit{\hfill Keywords}}\\%
{lorem, ipsum, dolor, sit, amet.}%

\selectlanguage{polish}

\end{newrevplenv}